\documentclass[11pt]{article}
\usepackage[
ruled,
vlined,
%commentsnumbered,
linesnumbered
%,algosection
,spanish
]{algorithm2e}
\usepackage{amssymb}
\usepackage{amsmath}
\usepackage{amsfonts}
\usepackage{mathtools}
\usepackage{multicol}

\title{Método de Jacobi}

\begin{document}

\maketitle


\section{Algoritmo}

\begin{algorithm}[ht]
    \SetKwInOut{Input}{Entrada}
    \SetKwInOut{Output}{Salida}
    \SetKwFunction{Salida}{Salida}
    \Input{$n, A = [a_{ij}]$, donde $1\leq i,j\leq n$, $b=[b_i]$ donde $1\leq i\leq n$, $XO=x^{(0)}$,TOL y $N$}
    \Output{$x_1,x_2,\dots,x_n$}
    \Begin{
    $k=1$\\
    \While{$k\leq N$}{
        \For{$i=1$ \KwTo $n$}{
    $\displaystyle{x_i = \frac{1}{a_{ii}}\left[-\sum_{\substack{j=1\\j\not=i}}^n (a_{ij}XO_j) + b_i\right]}$
        }
        \If{$||\textbf{x}-\textbf{XO}||<TOL$}{
    \Salida{$x_1,x_2,\dots,x_n$}\\
    $fin$
        }
        $k=k+1$\\
        \For{$i=1$ \KwTo $n$}{
    $XO_i = x_i$
        } 
    }
    \Salida{Número máximo de iteraciones excedido.}
    }
\caption{Método iterativo de Jacobi}
\label{algo:jacobi}
\end{algorithm}


\section{Ejercicios}
  \begin{enumerate}
    \item Obtenga las dos primeras iteraciones del método de Jacobi para los siguientes sistemas lineales, use $\textbf{x}^{(0)} = \textbf{0}$.
      \begin{enumerate}
		\begin{multicols}{2}
		\item $\begin{array}{r}
				3x_1 - x_2 + x_3 = 1\\
				3x_1 + 6x_2 + 2x_3 = 0\\
				3x_1 + 3x_2 + 7x_3 = 4\\
			  \end{array}$
		\item $\begin{array}{r}
				10x_1 - x_2 = 9\\
				-x_1 + 10x_2 - 2x_3 = 7\\
				-2x_2 + 10x_3 = 6\\
			\end{array}$
		\item $\begin{array}{r}
				10x_1 + 5x_2 = 6\\
				5x_1 + 10x_2 - 4x_3 = 25\\
				-4x_2 + 8x_3 - x_4 = -11\\
				-x_3 + 5x_4 = -11\\
			   \end{array}$
		\item $\begin{array}{r}
				4x_1 + x_2 + x_3 + x_5 = 6\\
				-x_1 - 3x_2 + x_3 + x_4 = 6\\
				2x_1 + x_2 + 5x_3 - x_4 - x_5 = 6\\
				-x_1 - x_2 - x_3 + 4x_4 = 6\\
				2x_2 - x_3 + x_4 + 4x_5 = 6
			\end{array}$
		\end{multicols}
      \end{enumerate}		
    \item Obtenga las dos primeras iteraciones del método de Jacobi para los siguientes sistemas lineales, use $\textbf{x}^{(0)} = \textbf{0}$.
      \begin{enumerate}
		\begin{multicols}{2}
		\item $\begin{array}{r}
				4x_1 + x_2 - x_3 = 5\\
				-x_1 + 3x_2 + x_3 = -4\\
				2x_1 + 2x_2 + 5x_3 = 1
			  \end{array}$
		\item $\begin{array}{r}
				-2x_1 + x_2 + \frac{1}{2}x_3 = 4\\
				x_1 - 2x_2 - \frac{1}{2}x_3 = -4\\
				x_2 + 2x_3 = 0
			\end{array}$
		\item $\begin{array}{r}
				4x_1 + x_2 - x_3 + x_4 = -2\\
				x_1 + 4x_2 - x_3 - x_4 = -1\\
				-x_1 - x_2 + 5x_3 + x_4 = 0\\
				x_1 - x_2 + x_3 + 3x_4 = 1
			   \end{array}$
		\item $\begin{array}{r}
				4x_1 - x_2 - x_4 = 0\\
				-x_1 + 4x_2 - x_3 - x_5 = 5\\
				-x_2 + 4x_3 - x_6 = 0\\
				-x_1 + 4x_4 - x_5 = 6\\
				-x_2 - x_4 + 4x_5 - x_6 = -2\\
				-x_3 - x_5 + 4x_6 = 6
			\end{array}$
		\end{multicols}		
      \end{enumerate}
\end{enumerate}


\end{document}