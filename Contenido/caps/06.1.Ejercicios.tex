\section{Ejercicios}

\subsection{map, filter, lambda}

\begin{exercise} 
Escribe una función lambda que tome dos números y devuelva su producto.
\begin{Shaded}
\begin{Highlighting}[]
\NormalTok{product }\OperatorTok{=} \KeywordTok{lambda}\NormalTok{ x, y: x }\OperatorTok{*}\NormalTok{ y}
\BuiltInTok{print}\NormalTok{(product(}\DecValTok{4}\NormalTok{, }\DecValTok{5}\NormalTok{))}
\end{Highlighting}
\end{Shaded}

\begin{verbatim}
20

\end{verbatim}
\end{exercise}

\begin{exercise}
Usa una función lambda dentro de map para sumar 10 a cada número en una lista.
\begin{Shaded}
\begin{Highlighting}[]
\NormalTok{numbers }\OperatorTok{=}\NormalTok{ [}\DecValTok{1}\NormalTok{, }\DecValTok{2}\NormalTok{, }\DecValTok{3}\NormalTok{, }\DecValTok{4}\NormalTok{, }\DecValTok{5}\NormalTok{]}
\NormalTok{result }\OperatorTok{=} \BuiltInTok{list}\NormalTok{(}\BuiltInTok{map}\NormalTok{(}\KeywordTok{lambda}\NormalTok{ x: x }\OperatorTok{+} \DecValTok{10}\NormalTok{, numbers))}
\BuiltInTok{print}\NormalTok{(result)}
\end{Highlighting}
\end{Shaded}

\begin{verbatim}
[11, 12, 13, 14, 15]

\end{verbatim}
\end{exercise}


\begin{exercise}
Usa una función lambda para ordenar una lista de tuplas en función del segundo elemento de cada tupla.
\begin{Shaded}
\begin{Highlighting}[]
\NormalTok{tuples }\OperatorTok{=}\NormalTok{ [(}\DecValTok{1}\NormalTok{, }\DecValTok{2}\NormalTok{), (}\DecValTok{3}\NormalTok{, }\DecValTok{1}\NormalTok{), (}\DecValTok{5}\NormalTok{, }\DecValTok{4}\NormalTok{)]}
\NormalTok{sorted\_tuples }\OperatorTok{=} \BuiltInTok{sorted}\NormalTok{(tuples, key}\OperatorTok{=}\KeywordTok{lambda}\NormalTok{ x: x[}\DecValTok{1}\NormalTok{])}
\BuiltInTok{print}\NormalTok{(sorted\_tuples)  }\CommentTok{\# Salida esperada: [(3, 1), (1, 2), (5, 4)]}
\end{Highlighting}
\end{Shaded}

\begin{verbatim}
[(3, 1), (1, 2), (5, 4)]

\end{verbatim}
\end{exercise}

\begin{exercise}
Usa una función lambda dentro de sorted para ordenar una lista de cadenas en función de su longitud.
\begin{Shaded}
\begin{Highlighting}[]
\NormalTok{strings }\OperatorTok{=}\NormalTok{ [}\StringTok{"Hola"}\NormalTok{, }\StringTok{"Mundo"}\NormalTok{, }\StringTok{"Python"}\NormalTok{, }\StringTok{"es"}\NormalTok{, }\StringTok{"genial"}\NormalTok{]}
\NormalTok{sorted\_strings }\OperatorTok{=} \BuiltInTok{sorted}\NormalTok{(strings, key}\OperatorTok{=}\KeywordTok{lambda}\NormalTok{ s: }\BuiltInTok{len}\NormalTok{(s))}
\BuiltInTok{print}\NormalTok{(sorted\_strings)}
\end{Highlighting}
\end{Shaded}

\begin{verbatim}
['es', 'Hola', 'Mundo', 'Python', 'genial']

\end{verbatim}
\end{exercise}

\begin{exercise}
Filtrar y transformar una lista de diccionarios. Escribe un programa
que tome una lista de diccionarios, donde cada diccionario representa
a una persona con llaves nombre y edad. Devuelve una nueva lista de
nombres de personas que tengan al menos 18 años.

\begin{Shaded}
\begin{Highlighting}[]
\KeywordTok{def}\NormalTok{ mayores\_de\_edad(personas):}
    \ControlFlowTok{return} \BuiltInTok{list}\NormalTok{(}
                \BuiltInTok{map}\NormalTok{(}
                    \KeywordTok{lambda}\NormalTok{ persona: persona[}\StringTok{\textquotesingle{}nombre\textquotesingle{}}\NormalTok{], }
                    \BuiltInTok{filter}\NormalTok{(}
                        \KeywordTok{lambda}\NormalTok{ persona: persona[}\StringTok{\textquotesingle{}edad\textquotesingle{}}\NormalTok{] }\OperatorTok{\textgreater{}=} \DecValTok{18}\NormalTok{, }
\NormalTok{                        personas}
\NormalTok{                    )}
\NormalTok{                )}
\NormalTok{            )}

\CommentTok{\# Ejemplo de uso}
\NormalTok{personas }\OperatorTok{=}\NormalTok{ [}
\NormalTok{    \{}\StringTok{"nombre"}\NormalTok{: }\StringTok{"Ana"}\NormalTok{, }\StringTok{"edad"}\NormalTok{: }\DecValTok{22}\NormalTok{\},}
\NormalTok{    \{}\StringTok{"nombre"}\NormalTok{: }\StringTok{"Luis"}\NormalTok{, }\StringTok{"edad"}\NormalTok{: }\DecValTok{17}\NormalTok{\},}
\NormalTok{    \{}\StringTok{"nombre"}\NormalTok{: }\StringTok{"Marta"}\NormalTok{, }\StringTok{"edad"}\NormalTok{: }\DecValTok{19}\NormalTok{\},}
\NormalTok{    \{}\StringTok{"nombre"}\NormalTok{: }\StringTok{"Carlos"}\NormalTok{, }\StringTok{"edad"}\NormalTok{: }\DecValTok{15}\NormalTok{\}}
\NormalTok{]}
\NormalTok{nombres\_mayores }\OperatorTok{=}\NormalTok{ mayores\_de\_edad(personas)}
\BuiltInTok{print}\NormalTok{(nombres\_mayores)  }\CommentTok{\# Output: [\textquotesingle{}Ana\textquotesingle{}, \textquotesingle{}Marta\textquotesingle{}]}
\end{Highlighting}
\end{Shaded}

\begin{verbatim}
['Ana', 'Marta']

\end{verbatim}
\end{exercise}

\begin{exercise}
Aplicar múltiples funciones a una lista de números. Escribe un
programa que tome una lista de números y aplique dos funciones
diferentes a cada número: una que calcule el cuadrado y otra que
calcule el cubo. Devuelve una lista de tuplas donde cada tupla
contiene el resultado de ambas funciones.

\begin{Shaded}
\begin{Highlighting}[]
\KeywordTok{def}\NormalTok{ cuadrado\_y\_cubo(lista):}
    \ControlFlowTok{return} \BuiltInTok{list}\NormalTok{(}
        \BuiltInTok{map}\NormalTok{(}
            \KeywordTok{lambda}\NormalTok{ x: (x }\OperatorTok{**} \DecValTok{2}\NormalTok{, x }\OperatorTok{**} \DecValTok{3}\NormalTok{), }
\NormalTok{            lista}
\NormalTok{        )}
\NormalTok{    )}

\CommentTok{\# Ejemplo de uso}
\NormalTok{numeros }\OperatorTok{=}\NormalTok{ [}\DecValTok{1}\NormalTok{, }\DecValTok{2}\NormalTok{, }\DecValTok{3}\NormalTok{, }\DecValTok{4}\NormalTok{]}
\NormalTok{resultados }\OperatorTok{=}\NormalTok{ cuadrado\_y\_cubo(numeros)}
\BuiltInTok{print}\NormalTok{(resultados)  }\CommentTok{\# Output: [(1, 1), (4, 8), (9, 27), (16, 64)]}
\end{Highlighting}
\end{Shaded}

\begin{verbatim}
[(1, 1), (4, 8), (9, 27), (16, 64)]

\end{verbatim}
\end{exercise}


\begin{exercise}
Ordenar una lista de tuplas basada en la suma de sus elementos.
Escribe un programa que tome una lista de tuplas, donde cada tupla
contiene dos números. Devuelve una nueva lista de tuplas ordenadas por
la suma de sus elementos.

\begin{Shaded}
\begin{Highlighting}[]
\KeywordTok{def}\NormalTok{ ordenar\_por\_suma(lista):}
    \ControlFlowTok{return} \BuiltInTok{list}\NormalTok{(}
        \BuiltInTok{map}\NormalTok{(}
            \KeywordTok{lambda}\NormalTok{ x: x, }
            \BuiltInTok{sorted}\NormalTok{(}
\NormalTok{                lista, }
\NormalTok{                key }\OperatorTok{=} \KeywordTok{lambda}\NormalTok{ x: x[}\DecValTok{0}\NormalTok{] }\OperatorTok{+}\NormalTok{ x[}\DecValTok{1}\NormalTok{]}
\NormalTok{            )}
\NormalTok{        )}
\NormalTok{    )}

\CommentTok{\# Ejemplo de uso}
\NormalTok{tuplas }\OperatorTok{=}\NormalTok{ [(}\DecValTok{1}\NormalTok{, }\DecValTok{2}\NormalTok{), (}\DecValTok{3}\NormalTok{, }\DecValTok{4}\NormalTok{), (}\DecValTok{1}\NormalTok{, }\DecValTok{1}\NormalTok{), (}\DecValTok{2}\NormalTok{, }\DecValTok{2}\NormalTok{)]}
\NormalTok{tuplas\_ordenadas }\OperatorTok{=}\NormalTok{ ordenar\_por\_suma(tuplas)}
\BuiltInTok{print}\NormalTok{(tuplas\_ordenadas)  }\CommentTok{\# Output: [(1, 1), (1, 2), (2, 2), (3, 4)]}
\end{Highlighting}
\end{Shaded}

\begin{verbatim}
[(1, 1), (1, 2), (2, 2), (3, 4)]

\end{verbatim}
\end{exercise}

\begin{exercise}
Convertir una lista de tuplas a un diccionario. Escribe un programa
que tome una lista de tuplas, donde cada tupla contiene una llave y un
valor. Devuelve un diccionario construido a partir de estas tuplas.

\begin{Shaded}
\begin{Highlighting}[]
\KeywordTok{def}\NormalTok{ tuplas\_a\_diccionario(lista):}
    \ControlFlowTok{return} \BuiltInTok{dict}\NormalTok{(}\BuiltInTok{map}\NormalTok{(}\KeywordTok{lambda}\NormalTok{ x: (x[}\DecValTok{0}\NormalTok{], x[}\DecValTok{1}\NormalTok{]), lista))}

\CommentTok{\# Ejemplo de uso}
\NormalTok{tuplas }\OperatorTok{=}\NormalTok{ [(}\StringTok{"llave1"}\NormalTok{, }\StringTok{"valor1"}\NormalTok{), (}\StringTok{"llave2"}\NormalTok{, }\StringTok{"valor2"}\NormalTok{), (}\StringTok{"llave3"}\NormalTok{, }\StringTok{"valor3"}\NormalTok{)]}
\NormalTok{diccionario }\OperatorTok{=}\NormalTok{ tuplas\_a\_diccionario(tuplas)}
\BuiltInTok{print}\NormalTok{(diccionario)  }\CommentTok{\# Output: \{\textquotesingle{}llave1\textquotesingle{}: \textquotesingle{}valor1\textquotesingle{}, \textquotesingle{}llave2\textquotesingle{}: \textquotesingle{}valor2\textquotesingle{}, \textquotesingle{}llave3\textquotesingle{}: \textquotesingle{}valor3\textquotesingle{}\}}
\end{Highlighting}
\end{Shaded}

\begin{verbatim}
{'llave1': 'valor1', 'llave2': 'valor2', 'llave3': 'valor3'}

\end{verbatim}
\end{exercise}

\begin{exercise}
Aplicar una función a una lista de listas. Escribe un programa que
tome una lista de listas de números y aplique una función que calcule
el promedio de cada lista interna. Devuelve una lista de promedios.

\begin{Shaded}
\begin{Highlighting}[]
\KeywordTok{def}\NormalTok{ promedios\_de\_listas(lista):}
    \ControlFlowTok{return} \BuiltInTok{list}\NormalTok{(}
        \BuiltInTok{map}\NormalTok{(}
            \KeywordTok{lambda}\NormalTok{ x: }\BuiltInTok{sum}\NormalTok{(x) }\OperatorTok{/} \BuiltInTok{len}\NormalTok{(x) }\ControlFlowTok{if} \BuiltInTok{len}\NormalTok{(x) }\OperatorTok{\textgreater{}} \DecValTok{0} \ControlFlowTok{else} \DecValTok{0}\NormalTok{, }
\NormalTok{            lista}
\NormalTok{        )}
\NormalTok{    )}

\CommentTok{\# Ejemplo de uso}
\NormalTok{listas }\OperatorTok{=}\NormalTok{ [[}\DecValTok{1}\NormalTok{, }\DecValTok{2}\NormalTok{, }\DecValTok{3}\NormalTok{], [}\DecValTok{4}\NormalTok{, }\DecValTok{5}\NormalTok{, }\DecValTok{6}\NormalTok{, }\DecValTok{7}\NormalTok{], [}\DecValTok{8}\NormalTok{, }\DecValTok{9}\NormalTok{], []]}
\NormalTok{promedios }\OperatorTok{=}\NormalTok{ promedios\_de\_listas(listas)}
\BuiltInTok{print}\NormalTok{(promedios)  }\CommentTok{\# Output: [2.0, 5.5, 8.5, 0]}
\end{Highlighting}
\end{Shaded}

\begin{verbatim}
[2.0, 5.5, 8.5, 0]

\end{verbatim}
\end{exercise}

\subsection{Reduce}

\begin{exercise}
Usa una función lambda dentro de reduce para calcular el producto de
todos los números en una lista.

\begin{Shaded}
\begin{Highlighting}[]
\ImportTok{from}\NormalTok{ functools }\ImportTok{import} \BuiltInTok{reduce}

\NormalTok{numbers }\OperatorTok{=}\NormalTok{ [}\DecValTok{1}\NormalTok{, }\DecValTok{2}\NormalTok{, }\DecValTok{3}\NormalTok{, }\DecValTok{4}\NormalTok{, }\DecValTok{5}\NormalTok{]}
\NormalTok{product }\OperatorTok{=} \BuiltInTok{reduce}\NormalTok{(}\KeywordTok{lambda}\NormalTok{ x, y: x }\OperatorTok{*}\NormalTok{ y, numbers)}
\BuiltInTok{print}\NormalTok{(product)  }\CommentTok{\# Salida esperada: 120}
\end{Highlighting}
\end{Shaded}

\begin{verbatim}
120

\end{verbatim}
\end{exercise}

\begin{exercise}
Suma de una lista de números. Escribe un programa que tome una lista
de números y devuelva la suma de todos los números en la lista.

\begin{Shaded}
\begin{Highlighting}[]
\ImportTok{from}\NormalTok{ functools }\ImportTok{import} \BuiltInTok{reduce}

\KeywordTok{def}\NormalTok{ suma(lista):}
    \ControlFlowTok{return} \BuiltInTok{reduce}\NormalTok{(}\KeywordTok{lambda}\NormalTok{ x, y: x }\OperatorTok{+}\NormalTok{ y, lista)}

\CommentTok{\# Ejemplo de uso}
\NormalTok{numeros }\OperatorTok{=}\NormalTok{ [}\DecValTok{1}\NormalTok{, }\DecValTok{2}\NormalTok{, }\DecValTok{3}\NormalTok{, }\DecValTok{4}\NormalTok{, }\DecValTok{5}\NormalTok{]}
\NormalTok{resultado }\OperatorTok{=}\NormalTok{ suma(numeros)}
\BuiltInTok{print}\NormalTok{(resultado)  }\CommentTok{\# Output: 15}
\end{Highlighting}
\end{Shaded}

\begin{verbatim}
15

\end{verbatim}
\end{exercise}

\begin{exercise}
Encontrar el máximo en una lista de números. Escribe un programa que
tome una lista de números y devuelva el número más grande en la lista.

\begin{Shaded}
\begin{Highlighting}[]
\ImportTok{from}\NormalTok{ functools }\ImportTok{import} \BuiltInTok{reduce}

\KeywordTok{def}\NormalTok{ maximo(lista):}
    \ControlFlowTok{return} \BuiltInTok{reduce}\NormalTok{(}\KeywordTok{lambda}\NormalTok{ x, y: x }\ControlFlowTok{if}\NormalTok{ x }\OperatorTok{\textgreater{}}\NormalTok{ y }\ControlFlowTok{else}\NormalTok{ y, lista)}

\CommentTok{\# Ejemplo de uso}
\NormalTok{numeros }\OperatorTok{=}\NormalTok{ [}\DecValTok{1}\NormalTok{, }\DecValTok{7}\NormalTok{, }\DecValTok{3}\NormalTok{, }\DecValTok{9}\NormalTok{, }\DecValTok{5}\NormalTok{]}
\NormalTok{resultado }\OperatorTok{=}\NormalTok{ maximo(numeros)}
\BuiltInTok{print}\NormalTok{(resultado)  }\CommentTok{\# Output: 9}
\end{Highlighting}
\end{Shaded}

\begin{verbatim}
9

\end{verbatim}
\end{exercise}


\begin{exercise}
Contar la frecuencia de elementos en una lista. Escribe un programa
que tome una lista de elementos y devuelva un diccionario con la
frecuencia de cada elemento en la lista.

\begin{Shaded}
\begin{Highlighting}[]
\ImportTok{from}\NormalTok{ functools }\ImportTok{import} \BuiltInTok{reduce}

\KeywordTok{def}\NormalTok{ frecuencia(lista):}
    \ControlFlowTok{return} \BuiltInTok{reduce}\NormalTok{(}\KeywordTok{lambda}\NormalTok{ acc, x: \{}\OperatorTok{**}\NormalTok{acc, x: acc.get(x, }\DecValTok{0}\NormalTok{) }\OperatorTok{+} \DecValTok{1}\NormalTok{\}, lista, \{\})}

\CommentTok{\# Ejemplo de uso}
\NormalTok{elementos }\OperatorTok{=}\NormalTok{ [}\StringTok{\textquotesingle{}a\textquotesingle{}}\NormalTok{, }\StringTok{\textquotesingle{}b\textquotesingle{}}\NormalTok{, }\StringTok{\textquotesingle{}a\textquotesingle{}}\NormalTok{, }\StringTok{\textquotesingle{}c\textquotesingle{}}\NormalTok{, }\StringTok{\textquotesingle{}b\textquotesingle{}}\NormalTok{, }\StringTok{\textquotesingle{}a\textquotesingle{}}\NormalTok{]}
\NormalTok{resultado }\OperatorTok{=}\NormalTok{ frecuencia(elementos)}
\BuiltInTok{print}\NormalTok{(resultado)  }\CommentTok{\# Output: \{\textquotesingle{}a\textquotesingle{}: 3, \textquotesingle{}b\textquotesingle{}: 2, \textquotesingle{}c\textquotesingle{}: 1\}}
\end{Highlighting}
\end{Shaded}

\begin{verbatim}
{'a': 3, 'b': 2, 'c': 1}

\end{verbatim}
\end{exercise}

\begin{exercise}
Calcular la diferencia entre el número más grande y el más pequeño en
una lista. Escribe un programa que tome una lista de números y
devuelva la diferencia entre el número más grande y el más pequeño en
la lista.

\begin{Shaded}
\begin{Highlighting}[]
\ImportTok{from}\NormalTok{ functools }\ImportTok{import} \BuiltInTok{reduce}

\KeywordTok{def}\NormalTok{ diferencia\_max\_min(lista):}
\NormalTok{    maximo }\OperatorTok{=} \BuiltInTok{reduce}\NormalTok{(}\KeywordTok{lambda}\NormalTok{ x, y: x }\ControlFlowTok{if}\NormalTok{ x }\OperatorTok{\textgreater{}}\NormalTok{ y }\ControlFlowTok{else}\NormalTok{ y, lista)}
\NormalTok{    minimo }\OperatorTok{=} \BuiltInTok{reduce}\NormalTok{(}\KeywordTok{lambda}\NormalTok{ x, y: x }\ControlFlowTok{if}\NormalTok{ x }\OperatorTok{\textless{}}\NormalTok{ y }\ControlFlowTok{else}\NormalTok{ y, lista)}
    \ControlFlowTok{return}\NormalTok{ maximo }\OperatorTok{{-}}\NormalTok{ minimo}

\CommentTok{\# Ejemplo de uso}
\NormalTok{numeros }\OperatorTok{=}\NormalTok{ [}\DecValTok{1}\NormalTok{, }\DecValTok{7}\NormalTok{, }\DecValTok{3}\NormalTok{, }\DecValTok{9}\NormalTok{, }\DecValTok{5}\NormalTok{]}
\NormalTok{resultado }\OperatorTok{=}\NormalTok{ diferencia\_max\_min(numeros)}
\BuiltInTok{print}\NormalTok{(resultado)  }\CommentTok{\# Output: 8}
\end{Highlighting}
\end{Shaded}

\begin{verbatim}
8

\end{verbatim}
\end{exercise}


\begin{exercise}
Encontrar el número de palabras en una lista de cadenas. Escribe un
programa que tome una lista de cadenas y devuelva el número total de
palabras en todas las cadenas.

\begin{Shaded}
\begin{Highlighting}[]
\ImportTok{from}\NormalTok{ functools }\ImportTok{import} \BuiltInTok{reduce}

\KeywordTok{def}\NormalTok{ contar\_palabras(lista):}
    \ControlFlowTok{return} \BuiltInTok{reduce}\NormalTok{(}\KeywordTok{lambda}\NormalTok{ acc, x: acc }\OperatorTok{+} \BuiltInTok{len}\NormalTok{(x.split()), lista, }\DecValTok{0}\NormalTok{)}

\CommentTok{\# Ejemplo de uso}
\NormalTok{cadenas }\OperatorTok{=}\NormalTok{ [}\StringTok{"hola mundo"}\NormalTok{, }\StringTok{"python es genial"}\NormalTok{, }\StringTok{"reduce es útil"}\NormalTok{]}
\NormalTok{resultado }\OperatorTok{=}\NormalTok{ contar\_palabras(cadenas)}
\BuiltInTok{print}\NormalTok{(resultado)  }\CommentTok{\# Output: 8}
\end{Highlighting}
\end{Shaded}

\begin{verbatim}
8
\end{verbatim}
\end{exercise}