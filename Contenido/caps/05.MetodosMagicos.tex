\chapter{Métodos Mágicos} 

\section{Atributos de clase protegidos}

Los atributos dentro de una clase pueden ser modificados por el usuario
directamente. Volvamos al ejemplo de la clase \texttt{Persona}.

\begin{tcolorbox}[breakable, size=fbox, boxrule=1pt, pad at break*=1mm,colback=cellbackground, colframe=cellborder]
\prompt{In}{incolor}{5.1}{\boxspacing}
\begin{Verbatim}[commandchars=\\\{\}]
\PY{k}{class}\PY{+w}{ }\PY{n+nc}{Persona}\PY{p}{:}
    \PY{k}{def}\PY{+w}{ }\PY{n+nf+fm}{\PYZus{}\PYZus{}init\PYZus{}\PYZus{}}\PY{p}{(}\PY{n+nb+bp}{self}\PY{p}{,} \PY{n}{nombre}\PY{p}{,} \PY{n}{edad}\PY{p}{,} \PY{n}{num\PYZus{}cuenta}\PY{p}{)}\PY{p}{:}
        \PY{n+nb+bp}{self}\PY{o}{.}\PY{n}{nombre}\PY{p}{,} \PY{n+nb+bp}{self}\PY{o}{.}\PY{n}{edad}\PY{p}{,} \PY{n+nb+bp}{self}\PY{o}{.}\PY{n}{num\PYZus{}cuenta} \PY{o}{=} \PY{n}{nombre}\PY{p}{,} \PY{n}{edad}\PY{p}{,} \PY{n}{num\PYZus{}cuenta}

    \PY{k}{def}\PY{+w}{ }\PY{n+nf}{mostrarInformación}\PY{p}{(}\PY{n+nb+bp}{self}\PY{p}{)}\PY{p}{:}
        \PY{n+nb}{print}\PY{p}{(}\PY{l+s+sa}{f}\PY{l+s+s1}{\PYZsq{}}\PY{l+s+si}{\PYZob{}}\PY{n+nb+bp}{self}\PY{o}{.}\PY{n}{nombre}\PY{l+s+si}{\PYZcb{}}\PY{l+s+s1}{ \PYZhy{}\PYZgt{} }\PY{l+s+si}{\PYZob{}}\PY{n+nb+bp}{self}\PY{o}{.}\PY{n}{edad}\PY{l+s+si}{\PYZcb{}}\PY{l+s+s1}{\PYZsq{}}\PY{p}{)}
\end{Verbatim}
\end{tcolorbox}

Con esta clase \texttt{Persona} se crea un objeto llamado \texttt{cliente1}.

\begin{tcolorbox}[breakable, size=fbox, boxrule=1pt, pad at break*=1mm,colback=cellbackground, colframe=cellborder]
\prompt{In}{incolor}{5.2}{\boxspacing}
\begin{Verbatim}[commandchars=\\\{\}]
\PY{n}{cliente1} \PY{o}{=} \PY{n}{Persona}\PY{p}{(}\PY{l+s+s2}{\PYZdq{}}\PY{l+s+s2}{Juan}\PY{l+s+s2}{\PYZdq{}}\PY{p}{,} \PY{l+m+mi}{34}\PY{p}{,} \PY{l+m+mi}{123456789}\PY{p}{)}
\PY{n}{cliente1}\PY{o}{.}\PY{n}{mostrarInformación}\PY{p}{(}\PY{p}{)}
\end{Verbatim}
\end{tcolorbox}

\begin{Verbatim}[commandchars=\\\{\}]
Juan -> 34
\end{Verbatim}

El objeto \texttt{cliente1} tiene los valores se sus propiedades pasadas
por el constructor al momento de ser creado, y no existe limitante
alguna para ser modificados directamente. Por ejemplo:

\begin{tcolorbox}[breakable, size=fbox, boxrule=1pt, pad at break*=1mm,colback=cellbackground, colframe=cellborder]
\prompt{In}{incolor}{5.3}{\boxspacing}
\begin{Verbatim}[commandchars=\\\{\}]
\PY{n}{cliente1}\PY{o}{.}\PY{n}{nombre} \PY{o}{=} \PY{l+s+s2}{\PYZdq{}}\PY{l+s+s2}{Juan Perez}\PY{l+s+s2}{\PYZdq{}}
\PY{n}{cliente1}\PY{o}{.}\PY{n}{mostrarInformación}\PY{p}{(}\PY{p}{)}
\end{Verbatim}
\end{tcolorbox}

\begin{Verbatim}[commandchars=\\\{\}]
Juan Perez -> 34
\end{Verbatim}

Aunque esto es permitido por el intérprete en \texttt{Python}, es
considerado como una mala práctica de programación. Lo que se debe hacer
es escribir un módulo dentro de la clase que reciba el nuevo nombre y
haga el cambio del valor de la propiedad. De esta manera se tiene un
control y certeza sobre los cambios en las propiedades. Considere además
el caso de que en la propiedad se almacene información sensible que no
se desea revelar, sería necesario limitar el acceso a la propiedad que
contenga esa información.

\begin{tcolorbox}[breakable, size=fbox, boxrule=1pt, pad at break*=1mm,colback=cellbackground, colframe=cellborder]
\prompt{In}{incolor}{5.4}{\boxspacing}
\begin{Verbatim}[commandchars=\\\{\}]
\PY{n}{cliente1}\PY{o}{.}\PY{n}{num\PYZus{}cuenta}
\end{Verbatim}
\end{tcolorbox}

\begin{tcolorbox}[breakable, size=fbox, boxrule=.5pt, pad at break*=1mm, opacityfill=0]
\prompt{Out}{outcolor}{5.5}{\boxspacing}
\begin{Verbatim}[commandchars=\\\{\}]
123456789
\end{Verbatim}
\end{tcolorbox}
        
Dado que \texttt{Python} no posee mecanismo para evitar el acceso o que
se modifiquen los valores de las propiedades, existe una convención para
marcar los atributos como protegidos. Para ello se utiliza el prefijo
guión bajo \_ en el nombre de la propiedad. Esto indica a cualquier otro
programador que dicha propiedad no debería ser accedida o modificada
fuera de la clase.

    \begin{tcolorbox}[breakable, size=fbox, boxrule=1pt, pad at break*=1mm,colback=cellbackground, colframe=cellborder]
\prompt{In}{incolor}{5.6}{\boxspacing}
\begin{Verbatim}[commandchars=\\\{\}]
\PY{k}{class}\PY{+w}{ }\PY{n+nc}{Persona}\PY{p}{:}
    \PY{k}{def}\PY{+w}{ }\PY{n+nf+fm}{\PYZus{}\PYZus{}init\PYZus{}\PYZus{}}\PY{p}{(}\PY{n+nb+bp}{self}\PY{p}{,} \PY{n}{nombre}\PY{p}{,} \PY{n}{edad}\PY{p}{,} \PY{n}{num\PYZus{}cuenta}\PY{p}{)}\PY{p}{:}
        \PY{n+nb+bp}{self}\PY{o}{.}\PY{n}{nombre}\PY{p}{,} \PY{n+nb+bp}{self}\PY{o}{.}\PY{n}{edad}\PY{p}{,} \PY{n+nb+bp}{self}\PY{o}{.}\PY{n}{\PYZus{}num\PYZus{}cuenta} \PY{o}{=} \PY{n}{nombre}\PY{p}{,} \PY{n}{edad}\PY{p}{,} \PY{n}{num\PYZus{}cuenta} 
        \PY{c+c1}{\PYZsh{} Nótese que num\PYZus{}cuenta se marcó como propiedad protegida}

    \PY{k}{def}\PY{+w}{ }\PY{n+nf}{mostrarInformación}\PY{p}{(}\PY{n+nb+bp}{self}\PY{p}{)}\PY{p}{:}
        \PY{n+nb}{print}\PY{p}{(}\PY{l+s+sa}{f}\PY{l+s+s1}{\PYZsq{}}\PY{l+s+si}{\PYZob{}}\PY{n+nb+bp}{self}\PY{o}{.}\PY{n}{nombre}\PY{l+s+si}{\PYZcb{}}\PY{l+s+s1}{ \PYZhy{}\PYZgt{} }\PY{l+s+si}{\PYZob{}}\PY{n+nb+bp}{self}\PY{o}{.}\PY{n}{edad}\PY{l+s+si}{\PYZcb{}}\PY{l+s+s1}{\PYZsq{}}\PY{p}{)}
\end{Verbatim}
\end{tcolorbox}

    \begin{tcolorbox}[breakable, size=fbox, boxrule=1pt, pad at break*=1mm,colback=cellbackground, colframe=cellborder]
\prompt{In}{incolor}{5.7}{\boxspacing}
\begin{Verbatim}[commandchars=\\\{\}]
\PY{n}{cliente1} \PY{o}{=} \PY{n}{Persona}\PY{p}{(}\PY{l+s+s2}{\PYZdq{}}\PY{l+s+s2}{Juan}\PY{l+s+s2}{\PYZdq{}}\PY{p}{,} \PY{l+m+mi}{34}\PY{p}{,} \PY{l+m+mi}{123456789}\PY{p}{)}
\PY{n}{cliente1}\PY{o}{.}\PY{n}{mostrarInformación}\PY{p}{(}\PY{p}{)}
\end{Verbatim}
\end{tcolorbox}

    \begin{Verbatim}[commandchars=\\\{\}]
Juan -> 34
    \end{Verbatim}

    \begin{tcolorbox}[breakable, size=fbox, boxrule=1pt, pad at break*=1mm,colback=cellbackground, colframe=cellborder]
\prompt{In}{incolor}{5.8}{\boxspacing}
\begin{Verbatim}[commandchars=\\\{\}]
\PY{n}{cliente1}\PY{o}{.}\PY{n}{\PYZus{}num\PYZus{}cuenta}
\end{Verbatim}
\end{tcolorbox}

            \begin{tcolorbox}[breakable, size=fbox, boxrule=.5pt, pad at break*=1mm, opacityfill=0]
\prompt{Out}{outcolor}{5.9}{\boxspacing}
\begin{Verbatim}[commandchars=\\\{\}]
123456789
\end{Verbatim}
\end{tcolorbox}
        
Aunque la propiedad \emph{\_num\_cuenta} aún puede ser impresa e incluso
modificada, marcarla como protegida indica al programador que hacerlo va
en contra de las buenas prácticas de codificación.

\section{Métodos especiales}

Los métodos especiales en Python, también conocidos como \textbf{métodos
mágicos} o \textbf{dunder methods}, son funciones integradas dentro de
las clases que permiten definir comportamientos específicos para
operaciones estándar. Tienen dos guiones bajos al principio y al final
de sus nombres, como \texttt{\_\_init\_\_} o \texttt{\_\_str\_\_}.

\subsection{Algunos métodos especiales comunes:}

\begin{itemize}
\item
  \textbf{\texttt{\_\_init\_\_}}: Inicializa una nueva instancia de una clase.
\item
  \textbf{\texttt{\_\_str\_\_}}: Define cómo representar un objeto como una cadena de texto, usado en \texttt{str(obj)} y \texttt{print(obj)}.
\item
  \textbf{\texttt{\_\_repr\_\_}}: Proporciona una representación oficial de un objeto, usada en \texttt{repr(obj)}.
\item
  \textbf{\texttt{\_\_len\_\_}}: Devuelve el tamaño o longitud de un objeto, usado en \texttt{len(obj)}.
\item
  \textbf{\texttt{\_\_getitem\_\_}, \texttt{\_\_setitem\_\_}, \texttt{\_\_delitem\_\_}}: Permiten acceder, modificar y eliminar
  elementos por índice.
\item
  \textbf{\texttt{\_\_iter\_\_}, \texttt{\_\_next\_\_}}: Permiten que un objeto sea iterable, como en un bucle \texttt{for}.
\item
  \textbf{\texttt{\_\_eq\_\_}, \texttt{\_\_lt\_\_}, \texttt{\_\_gt\_\_}}: Implementan comparaciones (igualdad, menor que,
  mayor que, etc.).
\item
  \textbf{\texttt{\_\_add\_\_}, \texttt{\_\_sub\_\_}, \texttt{\_\_mul\_\_}, etc.}: Definen el comportamiento de operadores
  aritméticos.
\end{itemize}

Ya hemos utilizado el método \texttt{\_\_init\_\_} que es el
constructor, y se ejecuta automáticamente cada vez que se crea un
objeto/instancia de la clase.

El nombre \textbf{método especial} o más aún \textbf{método mágico}
puede ser engañoso, ya que técnicamente no hay algo especial o mágico en
ellos. Lo único especial acerca de ellos es el nombre, el cual asegura
que serán llamados en situaciones especiales. Por ejemplo, el método
\texttt{\_\_init\_\_} que se ejecuta al crear un objeto.

Por ejemplo, considere la \emph{fórmula general barométrica} (\ref{eq:formulaGeneralBarométrica}) para
calcular la presión atmosférica \(p\) dada la altura \(h\).

\begin{equation}
   p = p_0e^{-Mgh/RT} 
   \label{eq:formulaGeneralBarométrica}
\end{equation}


donde \(M\) es la masa molar del aire, \(g\) es la constante
gravitacional, \(R\) es la constante del gas, \(T\) la temperatura y
\(p_0\) la presión del aire a nivel del mar. Se define además
\(h_0 = \dfrac{RT}{Mg}\).

\begin{equation}
    p = p_0e^{-h/h_0}
    \label{eq:formulaGeneralBarométrica2}
\end{equation}


Ahora se define una clase para el cálculo barométrico.

\begin{tcolorbox}[breakable, size=fbox, boxrule=1pt, pad at break*=1mm,colback=cellbackground, colframe=cellborder]
\prompt{In}{incolor}{5.10}{\boxspacing}
\begin{Verbatim}[commandchars=\\\{\}]
\PY{k+kn}{import}\PY{+w}{ }\PY{n+nn}{math} 

\PY{k}{class}\PY{+w}{ }\PY{n+nc}{Barometric}\PY{p}{:}
    \PY{k}{def}\PY{+w}{ }\PY{n+nf+fm}{\PYZus{}\PYZus{}init\PYZus{}\PYZus{}}\PY{p}{(}\PY{n+nb+bp}{self}\PY{p}{,} \PY{n}{T}\PY{p}{)}\PY{p}{:}
        \PY{n}{g} \PY{o}{=} \PY{l+m+mf}{9.81}        \PY{c+c1}{\PYZsh{} m/s²}
        \PY{n}{R} \PY{o}{=} \PY{l+m+mf}{8.314}       \PY{c+c1}{\PYZsh{} J/(K*mol)}
        \PY{n}{M} \PY{o}{=} \PY{l+m+mf}{0.02896}     \PY{c+c1}{\PYZsh{} kg/mol}
        \PY{n+nb+bp}{self}\PY{o}{.}\PY{n}{h0} \PY{o}{=} \PY{n}{R}\PY{o}{*}\PY{n}{T}\PY{o}{/}\PY{n}{M}\PY{o}{/}\PY{n}{g}
        \PY{n+nb+bp}{self}\PY{o}{.}\PY{n}{p0} \PY{o}{=} \PY{l+m+mi}{100}        \PY{c+c1}{\PYZsh{} kPa}

    \PY{k}{def}\PY{+w}{ }\PY{n+nf}{value}\PY{p}{(}\PY{n+nb+bp}{self}\PY{p}{,} \PY{n}{h}\PY{p}{)}\PY{p}{:}
        \PY{k}{return} \PY{n+nb+bp}{self}\PY{o}{.}\PY{n}{p0} \PY{o}{*} \PY{n}{math}\PY{o}{.}\PY{n}{exp}\PY{p}{(}\PY{o}{\PYZhy{}}\PY{n}{h}\PY{o}{/}\PY{n+nb+bp}{self}\PY{o}{.}\PY{n}{h0}\PY{p}{)}
\end{Verbatim}
\end{tcolorbox}

\begin{tcolorbox}[breakable, size=fbox, boxrule=1pt, pad at break*=1mm,colback=cellbackground, colframe=cellborder]
\prompt{In}{incolor}{5.11}{\boxspacing}
\begin{Verbatim}[commandchars=\\\{\}]
\PY{n}{bar1} \PY{o}{=} \PY{n}{Barometric}\PY{p}{(}\PY{l+m+mf}{292.15}\PY{p}{)}
\PY{n}{bar1}\PY{o}{.}\PY{n}{value}\PY{p}{(}\PY{l+m+mi}{2200}\PY{p}{)}
\end{Verbatim}
\end{tcolorbox}

\begin{tcolorbox}[breakable, size=fbox, boxrule=.5pt, pad at break*=1mm, opacityfill=0]
\prompt{Out}{outcolor}{5.12}{\boxspacing}
\begin{Verbatim}[commandchars=\\\{\}]
77.31204126500637
\end{Verbatim}
\end{tcolorbox}
        
Esta forma de la clase permite obtener el valor de la presión para
cierto valor de \(T\) y \(h\). El valor de la presión de obtiene al
llamar al método \texttt{value} y pasarle el argumento \(h\).

Sería más simple de utilizar si se pudiese llamar directamente al objeto
sin necesidad de emplear el método intermedio. Para ello existe el
método especial \texttt{\_\_call\_\_}.

Veamos una nueva versión de la clase empleando este método especial.

\begin{tcolorbox}[breakable, size=fbox, boxrule=1pt, pad at break*=1mm,colback=cellbackground, colframe=cellborder]
\prompt{In}{incolor}{5.13}{\boxspacing}
\begin{Verbatim}[commandchars=\\\{\}]
\PY{k+kn}{import}\PY{+w}{ }\PY{n+nn}{math} 

\PY{k}{class}\PY{+w}{ }\PY{n+nc}{Barometric}\PY{p}{:}
    \PY{k}{def}\PY{+w}{ }\PY{n+nf+fm}{\PYZus{}\PYZus{}init\PYZus{}\PYZus{}}\PY{p}{(}\PY{n+nb+bp}{self}\PY{p}{,} \PY{n}{T}\PY{p}{)}\PY{p}{:}
        \PY{n}{g} \PY{o}{=} \PY{l+m+mf}{9.81}        \PY{c+c1}{\PYZsh{} m/s²}
        \PY{n}{R} \PY{o}{=} \PY{l+m+mf}{8.314}       \PY{c+c1}{\PYZsh{} J/(K*mol)}
        \PY{n}{M} \PY{o}{=} \PY{l+m+mf}{0.02896}     \PY{c+c1}{\PYZsh{} kg/mol}
        \PY{n+nb+bp}{self}\PY{o}{.}\PY{n}{h0} \PY{o}{=} \PY{n}{R}\PY{o}{*}\PY{n}{T}\PY{o}{/}\PY{n}{M}\PY{o}{/}\PY{n}{g}
        \PY{n+nb+bp}{self}\PY{o}{.}\PY{n}{p0} \PY{o}{=} \PY{l+m+mi}{100}        \PY{c+c1}{\PYZsh{} kPa}

    \PY{k}{def}\PY{+w}{ }\PY{n+nf+fm}{\PYZus{}\PYZus{}call\PYZus{}\PYZus{}}\PY{p}{(}\PY{n+nb+bp}{self}\PY{p}{,} \PY{n}{h}\PY{p}{)}\PY{p}{:}
        \PY{k}{return} \PY{n+nb+bp}{self}\PY{o}{.}\PY{n}{p0} \PY{o}{*} \PY{n}{math}\PY{o}{.}\PY{n}{exp}\PY{p}{(}\PY{o}{\PYZhy{}}\PY{n}{h}\PY{o}{/}\PY{n+nb+bp}{self}\PY{o}{.}\PY{n}{h0}\PY{p}{)}
\end{Verbatim}
\end{tcolorbox}

    \begin{tcolorbox}[breakable, size=fbox, boxrule=1pt, pad at break*=1mm,colback=cellbackground, colframe=cellborder]
\prompt{In}{incolor}{5.14}{\boxspacing}
\begin{Verbatim}[commandchars=\\\{\}]
\PY{n}{bar2} \PY{o}{=} \PY{n}{Barometric}\PY{p}{(}\PY{l+m+mf}{292.15}\PY{p}{)}
\PY{n+nb}{print}\PY{p}{(}\PY{n}{bar2}\PY{p}{(}\PY{l+m+mi}{2200}\PY{p}{)}\PY{p}{)}

\PY{c+c1}{\PYZsh{} Es equivalente a esta forma de la llamada}
\PY{n+nb}{print}\PY{p}{(}\PY{n}{bar2}\PY{o}{.}\PY{n+nf+fm}{\PYZus{}\PYZus{}call\PYZus{}\PYZus{}}\PY{p}{(}\PY{l+m+mi}{2200}\PY{p}{)}\PY{p}{)}
\end{Verbatim}
\end{tcolorbox}

\begin{Verbatim}[commandchars=\\\{\}]
77.31204126500637
77.31204126500637
\end{Verbatim}

\subsection{Método especial para imprimir}

Es posible imprmir un objeto \texttt{a} empleando un \texttt{print(a)},
lo cual funciona bien para los objetos propios de Python como cadenas y
listas. Sin embrago, si nosotros creamos una clase, ese print no
necesariamente mostrará información útil. Por ello tendremos que
resolver ese problema definiendo el método \texttt{\_\_str\_\_} dentro
de la clase. El método \texttt{\_\_str\_\_} debe devolver de preferencia
una cadena y no debe recibir argumentos excepto por \texttt{self}.

Redefiniendo la clase \texttt{Barometric}, queda así:

\begin{tcolorbox}[breakable, size=fbox, boxrule=1pt, pad at break*=1mm,colback=cellbackground, colframe=cellborder]
\prompt{In}{incolor}{5.15}{\boxspacing}
\begin{Verbatim}[commandchars=\\\{\}]
\PY{k+kn}{import}\PY{+w}{ }\PY{n+nn}{math} 

\PY{k}{class}\PY{+w}{ }\PY{n+nc}{Barometric}\PY{p}{:}
    \PY{k}{def}\PY{+w}{ }\PY{n+nf+fm}{\PYZus{}\PYZus{}init\PYZus{}\PYZus{}}\PY{p}{(}\PY{n+nb+bp}{self}\PY{p}{,} \PY{n}{T}\PY{p}{)}\PY{p}{:}
        \PY{n}{g} \PY{o}{=} \PY{l+m+mf}{9.81}        \PY{c+c1}{\PYZsh{} m/s²}
        \PY{n}{R} \PY{o}{=} \PY{l+m+mf}{8.314}       \PY{c+c1}{\PYZsh{} J/(K*mol)}
        \PY{n}{M} \PY{o}{=} \PY{l+m+mf}{0.02896}     \PY{c+c1}{\PYZsh{} kg/mol}
        \PY{n+nb+bp}{self}\PY{o}{.}\PY{n}{h0} \PY{o}{=} \PY{n}{R}\PY{o}{*}\PY{n}{T}\PY{o}{/}\PY{n}{M}\PY{o}{/}\PY{n}{g}
        \PY{n+nb+bp}{self}\PY{o}{.}\PY{n}{p0} \PY{o}{=} \PY{l+m+mi}{100}        \PY{c+c1}{\PYZsh{} kPa}
        \PY{n+nb+bp}{self}\PY{o}{.}\PY{n}{T} \PY{o}{=} \PY{n}{T}

    \PY{k}{def}\PY{+w}{ }\PY{n+nf+fm}{\PYZus{}\PYZus{}call\PYZus{}\PYZus{}}\PY{p}{(}\PY{n+nb+bp}{self}\PY{p}{,} \PY{n}{h}\PY{p}{)}\PY{p}{:}
        \PY{k}{return} \PY{l+s+sa}{f}\PY{l+s+s1}{\PYZsq{}}\PY{l+s+s1}{p(h = }\PY{l+s+si}{\PYZob{}}\PY{n}{h}\PY{l+s+si}{\PYZcb{}}\PY{l+s+s1}{, T = }\PY{l+s+si}{\PYZob{}}\PY{n+nb+bp}{self}\PY{o}{.}\PY{n}{T}\PY{l+s+si}{\PYZcb{}}\PY{l+s+s1}{) = }\PY{l+s+si}{\PYZob{}}\PY{n+nb+bp}{self}\PY{o}{.}\PY{n}{p0}\PY{+w}{ }\PY{o}{*}\PY{+w}{ }\PY{n}{math}\PY{o}{.}\PY{n}{exp}\PY{p}{(}\PY{o}{\PYZhy{}}\PY{n}{h}\PY{o}{/}\PY{n+nb+bp}{self}\PY{o}{.}\PY{n}{h0}\PY{p}{)}\PY{l+s+si}{\PYZcb{}}\PY{l+s+s1}{ kPa}\PY{l+s+s1}{\PYZsq{}}
    
    \PY{k}{def}\PY{+w}{ }\PY{n+nf+fm}{\PYZus{}\PYZus{}str\PYZus{}\PYZus{}}\PY{p}{(}\PY{n+nb+bp}{self}\PY{p}{)}\PY{p}{:}
        \PY{k}{return} \PY{l+s+sa}{f}\PY{l+s+s1}{\PYZsq{}}\PY{l+s+s1}{p0 * exp(\PYZhy{}Mgh/(RT)) [kPa]; T = }\PY{l+s+si}{\PYZob{}}\PY{n+nb+bp}{self}\PY{o}{.}\PY{n}{T}\PY{l+s+si}{\PYZcb{}}\PY{l+s+s1}{ºK}\PY{l+s+s1}{\PYZsq{}}
\end{Verbatim}
\end{tcolorbox}

\begin{tcolorbox}[breakable, size=fbox, boxrule=1pt, pad at break*=1mm,colback=cellbackground, colframe=cellborder]
\prompt{In}{incolor}{5.16}{\boxspacing}
\begin{Verbatim}[commandchars=\\\{\}]
\PY{n}{bar3} \PY{o}{=} \PY{n}{Barometric}\PY{p}{(}\PY{l+m+mf}{292.15}\PY{p}{)}
\PY{n+nb}{print}\PY{p}{(}\PY{n}{bar3}\PY{p}{(}\PY{l+m+mi}{2200}\PY{p}{)}\PY{p}{)}
\PY{n+nb}{print}\PY{p}{(}\PY{n}{bar3}\PY{p}{)}
\end{Verbatim}
\end{tcolorbox}

\begin{Verbatim}[commandchars=\\\{\}]
p(h = 2200, T = 292.15) = 77.31204126500637 kPa
p0 * exp(-Mgh/(RT)) [kPa]; T = 292.15ºK
\end{Verbatim}

\subsection{Métodos especiales para operaciones matemáticas}

Hasta ahora hemos cubierto los métodos \texttt{\_\_init\_\_},
\texttt{\_\_call\_\_} y \texttt{\_\_str\_\_}, pero hay más de ellos. Por
ejemplo, los métodos \texttt{\_\_add\_\_}, \texttt{\_\_sub\_\_} y
\texttt{\_\_mul\_\_}. Definir estos métodos dentro de la clase nos
permite emplear expresiones como \(c = a + b\), donde \(a\) y \(b\) son
instancias de una clase.

\begin{Shaded}
\begin{Highlighting}[]
\NormalTok{c }\OperatorTok{=}\NormalTok{ a }\OperatorTok{+}\NormalTok{ b       }\CommentTok{\# c = a.\_\_add\_\_(b)}

\NormalTok{c }\OperatorTok{=}\NormalTok{ a }\OperatorTok{{-}}\NormalTok{ b       }\CommentTok{\# c = a.\_\_sub\_\_(b)}

\NormalTok{c }\OperatorTok{=}\NormalTok{ a }\OperatorTok{*}\NormalTok{ b       }\CommentTok{\# c = a.\_\_mul\_\_(b)}

\NormalTok{c }\OperatorTok{=}\NormalTok{ a }\OperatorTok{/}\NormalTok{ b       }\CommentTok{\# c = a.\_\_div\_\_(b)}

\NormalTok{c }\OperatorTok{=}\NormalTok{ a }\OperatorTok{**}\NormalTok{ b      }\CommentTok{\# c = a.\_\_pow\_\_(b)}
\end{Highlighting}
\end{Shaded}

Para la mayoría de los casos, cualquiera de estas operaciones devuelve
un objeto de la misma clase que los operandos.

De manera similar, también existen métodos especiales para comparar
objetos:

\begin{Shaded}
\begin{Highlighting}[]
\NormalTok{a }\OperatorTok{==}\NormalTok{ b          }\CommentTok{\# a.\_\_eq\_\_(b)}

\NormalTok{a }\OperatorTok{!=}\NormalTok{ b          }\CommentTok{\# a.\_\_ne\_\_(b)}

\NormalTok{a }\OperatorTok{\textless{}}\NormalTok{ b           }\CommentTok{\# a.\_\_lt\_\_(b)}

\NormalTok{a }\OperatorTok{\textless{}=}\NormalTok{ b          }\CommentTok{\# a.\_\_le\_\_(b)}

\NormalTok{a }\OperatorTok{\textgreater{}}\NormalTok{ b           }\CommentTok{\# a.\_\_gt\_\_(b)}

\NormalTok{a }\OperatorTok{\textgreater{}=}\NormalTok{ b          }\CommentTok{\# a.\_\_ge\_\_(b)}
\end{Highlighting}
\end{Shaded}

Estos métodos deben ser implementados para devolver un booleano, para
que sea consistente con el comportamiento de los operadores de
comparación.

El contenido de los métodos al momento de definirlos dependen del
desarrollador, lo único especial acerca de los métodos es su nombre, ya
que mediante este pueden ser llamados automáticamente por varios
operadores.

Por ejemplo, si se desea multiplicar dos objetos \(c = a * b\), Python
buscará el método llamado \texttt{\_\_mul\_\_} en la instancia \(a\). Si
el método existe, será llamado pasandole como argumento la instancia
\(b\) y cualquiera que sea la devolución del método \texttt{\_\_mul\_\_}
se asigna a \(c\).

\subsection{\texorpdfstring{Método especial
\texttt{\_\_repr\_\_}}{Método especial \_\_repr\_\_}}

Este método especial es semejante al método \texttt{\_\_str\_\_}, ya que
devuelve una cadena con información acerca del objeto. Por un lado la
cadena devuelta por \texttt{\_\_str\_\_} muestra información que es
fácilmente leíble y por otro lado la cadena devuelta por
\texttt{\_\_repr\_\_} contiene la información necesaria para recrear el
objeto.

Para un objeto \texttt{a} el método \texttt{\_\_repr\_\_} se puede
llamar mediante la función nativa de Python llamada \texttt{repr(a)}.

\begin{tcolorbox}[breakable, size=fbox, boxrule=1pt, pad at break*=1mm,colback=cellbackground, colframe=cellborder]
\prompt{In}{incolor}{5.17}{\boxspacing}
\begin{Verbatim}[commandchars=\\\{\}]
\PY{k+kn}{import}\PY{+w}{ }\PY{n+nn}{math} 

\PY{k}{class}\PY{+w}{ }\PY{n+nc}{Barometric}\PY{p}{:}
    \PY{k}{def}\PY{+w}{ }\PY{n+nf+fm}{\PYZus{}\PYZus{}init\PYZus{}\PYZus{}}\PY{p}{(}\PY{n+nb+bp}{self}\PY{p}{,} \PY{n}{T}\PY{p}{)}\PY{p}{:}
        \PY{n}{g} \PY{o}{=} \PY{l+m+mf}{9.81}        \PY{c+c1}{\PYZsh{} m/s²}
        \PY{n}{R} \PY{o}{=} \PY{l+m+mf}{8.314}       \PY{c+c1}{\PYZsh{} J/(K*mol)}
        \PY{n}{M} \PY{o}{=} \PY{l+m+mf}{0.02896}     \PY{c+c1}{\PYZsh{} kg/mol}
        \PY{n+nb+bp}{self}\PY{o}{.}\PY{n}{h0} \PY{o}{=} \PY{n}{R}\PY{o}{*}\PY{n}{T}\PY{o}{/}\PY{n}{M}\PY{o}{/}\PY{n}{g}
        \PY{n+nb+bp}{self}\PY{o}{.}\PY{n}{p0} \PY{o}{=} \PY{l+m+mi}{100}        \PY{c+c1}{\PYZsh{} kPa}
        \PY{n+nb+bp}{self}\PY{o}{.}\PY{n}{T} \PY{o}{=} \PY{n}{T}

    \PY{k}{def}\PY{+w}{ }\PY{n+nf+fm}{\PYZus{}\PYZus{}call\PYZus{}\PYZus{}}\PY{p}{(}\PY{n+nb+bp}{self}\PY{p}{,} \PY{n}{h}\PY{p}{)}\PY{p}{:}
        \PY{k}{return} \PY{l+s+sa}{f}\PY{l+s+s1}{\PYZsq{}}\PY{l+s+s1}{p(h = }\PY{l+s+si}{\PYZob{}}\PY{n}{h}\PY{l+s+si}{\PYZcb{}}\PY{l+s+s1}{, T = }\PY{l+s+si}{\PYZob{}}\PY{n+nb+bp}{self}\PY{o}{.}\PY{n}{T}\PY{l+s+si}{\PYZcb{}}\PY{l+s+s1}{) = }\PY{l+s+si}{\PYZob{}}\PY{n+nb+bp}{self}\PY{o}{.}\PY{n}{p0}\PY{+w}{ }\PY{o}{*}\PY{+w}{ }\PY{n}{math}\PY{o}{.}\PY{n}{exp}\PY{p}{(}\PY{o}{\PYZhy{}}\PY{n}{h}\PY{o}{/}\PY{n+nb+bp}{self}\PY{o}{.}\PY{n}{h0}\PY{p}{)}\PY{l+s+si}{\PYZcb{}}\PY{l+s+s1}{ kPa}\PY{l+s+s1}{\PYZsq{}}
    
    \PY{k}{def}\PY{+w}{ }\PY{n+nf+fm}{\PYZus{}\PYZus{}str\PYZus{}\PYZus{}}\PY{p}{(}\PY{n+nb+bp}{self}\PY{p}{)}\PY{p}{:}
        \PY{k}{return} \PY{l+s+sa}{f}\PY{l+s+s1}{\PYZsq{}}\PY{l+s+s1}{p0 * exp(\PYZhy{}Mgh/(RT)) [kPa]; T = }\PY{l+s+si}{\PYZob{}}\PY{n+nb+bp}{self}\PY{o}{.}\PY{n}{T}\PY{l+s+si}{\PYZcb{}}\PY{l+s+s1}{ºK}\PY{l+s+s1}{\PYZsq{}}

    \PY{k}{def}\PY{+w}{ }\PY{n+nf+fm}{\PYZus{}\PYZus{}repr\PYZus{}\PYZus{}}\PY{p}{(}\PY{n+nb+bp}{self}\PY{p}{)}\PY{p}{:}
\PY{+w}{        }\PY{l+s+sd}{\PYZdq{}\PYZdq{}\PYZdq{} Return code for regenerating this instance \PYZdq{}\PYZdq{}\PYZdq{}}
        \PY{k}{return} \PY{l+s+sa}{f}\PY{l+s+s1}{\PYZsq{}}\PY{l+s+s1}{Barometric(}\PY{l+s+si}{\PYZob{}}\PY{n+nb+bp}{self}\PY{o}{.}\PY{n}{T}\PY{l+s+si}{\PYZcb{}}\PY{l+s+s1}{)}\PY{l+s+s1}{\PYZsq{}}
\end{Verbatim}
\end{tcolorbox}

\begin{tcolorbox}[breakable, size=fbox, boxrule=1pt, pad at break*=1mm,colback=cellbackground, colframe=cellborder]
\prompt{In}{incolor}{5.18}{\boxspacing}
\begin{Verbatim}[commandchars=\\\{\}]
\PY{n}{b3} \PY{o}{=} \PY{n}{Barometric}\PY{p}{(}\PY{l+m+mf}{292.15}\PY{p}{)}
\PY{n+nb}{print}\PY{p}{(}\PY{n}{b3}\PY{p}{)}
\PY{n+nb}{repr}\PY{p}{(}\PY{n}{b3}\PY{p}{)}
\end{Verbatim}
\end{tcolorbox}

\begin{Verbatim}[commandchars=\\\{\}]
p0 * exp(-Mgh/(RT)) [kPa]; T = 292.15ºK
\end{Verbatim}

\begin{tcolorbox}[breakable, size=fbox, boxrule=.5pt, pad at break*=1mm, opacityfill=0]
\prompt{Out}{outcolor}{5.19}{\boxspacing}
\begin{Verbatim}[commandchars=\\\{\}]
'Barometric(292.15)'
\end{Verbatim}
\end{tcolorbox}
        
\begin{tcolorbox}[breakable, size=fbox, boxrule=1pt, pad at break*=1mm,colback=cellbackground, colframe=cellborder]
\prompt{In}{incolor}{5.20}{\boxspacing}
\begin{Verbatim}[commandchars=\\\{\}]
\PY{n}{b4} \PY{o}{=} \PY{n+nb}{eval}\PY{p}{(}\PY{n+nb}{repr}\PY{p}{(}\PY{n}{b3}\PY{p}{)}\PY{p}{)}
\PY{n+nb}{print}\PY{p}{(}\PY{n}{b4}\PY{p}{)}
\end{Verbatim}
\end{tcolorbox}

\begin{Verbatim}[commandchars=\\\{\}]
p0 * exp(-Mgh/(RT)) [kPa]; T = 292.15ºK
\end{Verbatim}

Estos resultados confirman que el método \texttt{\_\_repr\_\_} funciona
de acuerdo a lo esperado, dado que \texttt{eval(repr(b3))} devuelve un
objeto idéntico a \texttt{b3}.

Ambos métodos \texttt{\_\_str\_\_} y \texttt{\_\_repr\_\_} muestran
información acerca de un objeto, la diferencia es que una muestra
información legible para humanos y la segunda información legible para
Python.

\subsection{Mostrar contenido de una clase}

Algunas veces resulta útil mostrar el contenido de una clase, por
ejemplo para realizar debugging.

Considere la siguiente clase de ejemplo que solo contiene un comentario,
el constructor y una propiedad:

\begin{tcolorbox}[breakable, size=fbox, boxrule=1pt, pad at break*=1mm,colback=cellbackground, colframe=cellborder]
\prompt{In}{incolor}{5.21}{\boxspacing}
\begin{Verbatim}[commandchars=\\\{\}]
\PY{k}{class}\PY{+w}{ }\PY{n+nc}{A}\PY{p}{:}
\PY{+w}{    }\PY{l+s+sd}{\PYZdq{}\PYZdq{}\PYZdq{} Una clase de muestra \PYZdq{}\PYZdq{}\PYZdq{}}
    \PY{k}{def}\PY{+w}{ }\PY{n+nf+fm}{\PYZus{}\PYZus{}init\PYZus{}\PYZus{}}\PY{p}{(}\PY{n+nb+bp}{self}\PY{p}{,} \PY{n}{value}\PY{p}{)}\PY{p}{:}
        \PY{n+nb+bp}{self}\PY{o}{.}\PY{n}{v} \PY{o}{=} \PY{n}{value}
\end{Verbatim}
\end{tcolorbox}

Si se realiza un dir(A) se mostrarán varios métodos y propiedades que se
han definido automáticamente en la clase.

\begin{tcolorbox}[breakable, size=fbox, boxrule=1pt, pad at break*=1mm,colback=cellbackground, colframe=cellborder]
\prompt{In}{incolor}{5.22}{\boxspacing}
\begin{Verbatim}[commandchars=\\\{\}]
\PY{n+nb}{dir}\PY{p}{(}\PY{n}{A}\PY{p}{)}
\end{Verbatim}
\end{tcolorbox}

\begin{tcolorbox}[breakable, size=fbox, boxrule=.5pt, pad at break*=1mm, opacityfill=0]
\prompt{Out}{outcolor}{5.23}{\boxspacing}
\begin{Verbatim}[commandchars=\\\{\}]
['\_\_class\_\_',
 '\_\_delattr\_\_',
 '\_\_dict\_\_',
 '\_\_dir\_\_',
 '\_\_doc\_\_',
 '\_\_eq\_\_',
 '\_\_format\_\_',
 '\_\_ge\_\_',
 '\_\_getattribute\_\_',
 '\_\_getstate\_\_',
 '\_\_gt\_\_',
 '\_\_hash\_\_',
 '\_\_init\_\_',
 '\_\_init\_subclass\_\_',
 '\_\_le\_\_',
 '\_\_lt\_\_',
 '\_\_module\_\_',
 '\_\_ne\_\_',
 '\_\_new\_\_',
 '\_\_reduce\_\_',
 '\_\_reduce\_ex\_\_',
 '\_\_repr\_\_',
 '\_\_setattr\_\_',
 '\_\_sizeof\_\_',
 '\_\_str\_\_',
 '\_\_subclasshook\_\_',
 '\_\_weakref\_\_']
\end{Verbatim}
\end{tcolorbox}
        
Además, si se crea un objeto de la clase y se ejecuta el método
\texttt{dir(a)} observaremos la misma salida mostrado con la clase pero
también los valores creados por el constructor al momento de la creación
del objeto.

\begin{tcolorbox}[breakable, size=fbox, boxrule=1pt, pad at break*=1mm,colback=cellbackground, colframe=cellborder]
\prompt{In}{incolor}{5.24}{\boxspacing}
\begin{Verbatim}[commandchars=\\\{\}]
\PY{n}{a} \PY{o}{=} \PY{n}{A}\PY{p}{(}\PY{l+m+mi}{2}\PY{p}{)}
\PY{n+nb}{dir}\PY{p}{(}\PY{n}{a}\PY{p}{)}
\end{Verbatim}
\end{tcolorbox}

\begin{tcolorbox}[breakable, size=fbox, boxrule=.5pt, pad at break*=1mm, opacityfill=0]
\prompt{Out}{outcolor}{5.25}{\boxspacing}
\begin{Verbatim}[commandchars=\\\{\}]
['\_\_class\_\_',
 '\_\_delattr\_\_',
 '\_\_dict\_\_',
 '\_\_dir\_\_',
 '\_\_doc\_\_',
 '\_\_eq\_\_',
 '\_\_format\_\_',
 '\_\_ge\_\_',
 '\_\_getattribute\_\_',
 '\_\_getstate\_\_',
 '\_\_gt\_\_',
 '\_\_hash\_\_',
 '\_\_init\_\_',
 '\_\_init\_subclass\_\_',
 '\_\_le\_\_',
 '\_\_lt\_\_',
 '\_\_module\_\_',
 '\_\_ne\_\_',
 '\_\_new\_\_',
 '\_\_reduce\_\_',
 '\_\_reduce\_ex\_\_',
 '\_\_repr\_\_',
 '\_\_setattr\_\_',
 '\_\_sizeof\_\_',
 '\_\_str\_\_',
 '\_\_subclasshook\_\_',
 '\_\_weakref\_\_',
 'v']
\end{Verbatim}
\end{tcolorbox}
        
\subsection{\texorpdfstring{Método especial \texttt{\_\_doc\_\_}}{Método especial \_\_doc\_\_}}

El método especial \texttt{\_\_doc\_\_} muestra los comentarios que
existen dentro de la definición de la clase que han sido escritos dentro
de triples comillas dobles. Estos textos y este método son los que
sirven para construir la documentación del código posteriormente.

\begin{tcolorbox}[breakable, size=fbox, boxrule=1pt, pad at break*=1mm,colback=cellbackground, colframe=cellborder]
\prompt{In}{incolor}{5.26}{\boxspacing}
\begin{Verbatim}[commandchars=\\\{\}]
\PY{n}{a}\PY{o}{.}\PY{n+nv+vm}{\PYZus{}\PYZus{}doc\PYZus{}\PYZus{}}
\end{Verbatim}
\end{tcolorbox}

\begin{tcolorbox}[breakable, size=fbox, boxrule=.5pt, pad at break*=1mm, opacityfill=0]
\prompt{Out}{outcolor}{5.27}{\boxspacing}
\begin{Verbatim}[commandchars=\\\{\}]
' Una clase de muestra '
\end{Verbatim}
\end{tcolorbox}
        
\begin{tcolorbox}[breakable, size=fbox, boxrule=1pt, pad at break*=1mm,colback=cellbackground, colframe=cellborder]
\prompt{In}{incolor}{5.28}{\boxspacing}
\begin{Verbatim}[commandchars=\\\{\}]
\PY{n}{A}\PY{o}{.}\PY{n+nv+vm}{\PYZus{}\PYZus{}doc\PYZus{}\PYZus{}}
\end{Verbatim}
\end{tcolorbox}

\begin{tcolorbox}[breakable, size=fbox, boxrule=.5pt, pad at break*=1mm, opacityfill=0]
\prompt{Out}{outcolor}{5.29}{\boxspacing}
\begin{Verbatim}[commandchars=\\\{\}]
' Una clase de muestra '
\end{Verbatim}
\end{tcolorbox}
        
\begin{tcolorbox}[breakable, size=fbox, boxrule=1pt, pad at break*=1mm,colback=cellbackground, colframe=cellborder]
\prompt{In}{incolor}{5.30}{\boxspacing}
\begin{Verbatim}[commandchars=\\\{\}]
\PY{k+kn}{import}\PY{+w}{ }\PY{n+nn}{numpy}
\PY{n}{numpy}\PY{o}{.}\PY{n+nv+vm}{\PYZus{}\PYZus{}doc\PYZus{}\PYZus{}}
\end{Verbatim}
\end{tcolorbox}

\begin{tcolorbox}[breakable, size=fbox, boxrule=.5pt, pad at break*=1mm, opacityfill=0]
\prompt{Out}{outcolor}{5.31}{\boxspacing}
\begin{Verbatim}[commandchars=\\\{\}]
'\textbackslash{}nNumPy\textbackslash{}n=====\textbackslash{}n\textbackslash{}nProvides\textbackslash{}n  1. An array object of arbitrary homogeneous
items\textbackslash{}n  2. Fast mathematical operations over arrays\textbackslash{}n  3. Linear Algebra,
Fourier Transforms, Random Number Generation\textbackslash{}n\textbackslash{}nHow to use the
documentation\textbackslash{}n----------------------------\textbackslash{}nDocumentation is available in two
forms: docstrings provided\textbackslash{}nwith the code, and a loose standing reference guide,
available from\textbackslash{}n`the NumPy homepage <https://numpy.org>`\_.\textbackslash{}n\textbackslash{}nWe recommend
exploring the docstrings using\textbackslash{}n`IPython <https://ipython.org>`\_, an advanced
Python shell with\textbackslash{}nTAB-completion and introspection capabilities.  See below for
further\textbackslash{}ninstructions.\textbackslash{}n\textbackslash{}nThe docstring examples assume that `numpy` has been
imported as ``np``::\textbackslash{}n\textbackslash{}n  >>> import numpy as np\textbackslash{}n\textbackslash{}nCode snippets are indicated
by three greater-than signs::\textbackslash{}n\textbackslash{}n  >>> x = 42\textbackslash{}n  >>> x = x + 1\textbackslash{}n\textbackslash{}nUse the built-
in ``help`` function to view a function\textbackslash{}'s docstring::\textbackslash{}n\textbackslash{}n  >>> help(np.sort)\textbackslash{}n
{\ldots} \# doctest: +SKIP\textbackslash{}n\textbackslash{}nFor some objects, ``np.info(obj)`` may provide
additional help.  This is\textbackslash{}nparticularly true if you see the line "Help on ufunc
object:" at the top\textbackslash{}nof the help() page.  Ufuncs are implemented in C, not
Python, for speed.\textbackslash{}nThe native Python help() does not know how to view their
help, but our\textbackslash{}nnp.info() function does.\textbackslash{}n\textbackslash{}nTo search for documents containing a
keyword, do::\textbackslash{}n\textbackslash{}n  >>> np.lookfor(\textbackslash{}'keyword\textbackslash{}')\textbackslash{}n  {\ldots} \# doctest:
+SKIP\textbackslash{}n\textbackslash{}nGeneral-purpose documents like a glossary and help on the basic
concepts\textbackslash{}nof numpy are available under the ``doc`` sub-module::\textbackslash{}n\textbackslash{}n  >>> from
numpy import doc\textbackslash{}n  >>> help(doc)\textbackslash{}n  {\ldots} \# doctest: +SKIP\textbackslash{}n\textbackslash{}nAvailable
subpackages\textbackslash{}n---------------------\textbackslash{}nlib\textbackslash{}n    Basic functions used by several
sub-packages.\textbackslash{}nrandom\textbackslash{}n    Core Random Tools\textbackslash{}nlinalg\textbackslash{}n    Core Linear Algebra
Tools\textbackslash{}nfft\textbackslash{}n    Core FFT routines\textbackslash{}npolynomial\textbackslash{}n    Polynomial tools\textbackslash{}ntesting\textbackslash{}n
NumPy testing tools\textbackslash{}ndistutils\textbackslash{}n    Enhancements to distutils with support for\textbackslash{}n
Fortran compilers support and more  (for Python <=
3.11).\textbackslash{}n\textbackslash{}nUtilities\textbackslash{}n---------\textbackslash{}ntest\textbackslash{}n    Run numpy unittests\textbackslash{}nshow\_config\textbackslash{}n
Show numpy build configuration\textbackslash{}nmatlib\textbackslash{}n    Make everything
matrices.\textbackslash{}n\_\_version\_\_\textbackslash{}n    NumPy version string\textbackslash{}n\textbackslash{}nViewing documentation using
IPython\textbackslash{}n-----------------------------------\textbackslash{}n\textbackslash{}nStart IPython and import `numpy`
usually under the alias ``np``: `import\textbackslash{}nnumpy as np`.  Then, directly past or
use the ``\%cpaste`` magic to paste\textbackslash{}nexamples into the shell.  To see which
functions are available in `numpy`,\textbackslash{}ntype ``np.<TAB>`` (where ``<TAB>`` refers
to the TAB key), or use\textbackslash{}n``np.*cos*?<ENTER>`` (where ``<ENTER>`` refers to the
ENTER key) to narrow\textbackslash{}ndown the list.  To view the docstring for a function,
use\textbackslash{}n``np.cos?<ENTER>`` (to view the docstring) and ``np.cos??<ENTER>`` (to
view\textbackslash{}nthe source code).\textbackslash{}n\textbackslash{}nCopies vs. in-place
operation\textbackslash{}n-----------------------------\textbackslash{}nMost of the functions in `numpy`
return a copy of the array argument\textbackslash{}n(e.g., `np.sort`).  In-place versions of
these functions are often\textbackslash{}navailable as array methods, i.e. ``x =
np.array([1,2,3]); x.sort()``.\textbackslash{}nExceptions to this rule are documented.\textbackslash{}n\textbackslash{}n'
\end{Verbatim}
\end{tcolorbox}
        
\subsection{Otors métodos especiales}

El método \texttt{\_\_module\_\_} devuelve el nombre del módulo al cual
pertenece la clase, en el siguiente ejemplo se devuelve
\texttt{\_\_main\_\_} dado que el objeto fue creado dentro de ese
módulo.

\begin{tcolorbox}[breakable, size=fbox, boxrule=1pt, pad at break*=1mm,colback=cellbackground, colframe=cellborder]
\prompt{In}{incolor}{5.32}{\boxspacing}
\begin{Verbatim}[commandchars=\\\{\}]
\PY{n}{a}\PY{o}{.}\PY{n+nv+vm}{\PYZus{}\PYZus{}module\PYZus{}\PYZus{}}
\end{Verbatim}
\end{tcolorbox}

\begin{tcolorbox}[breakable, size=fbox, boxrule=.5pt, pad at break*=1mm, opacityfill=0]
\prompt{Out}{outcolor}{5.33}{\boxspacing}
\begin{Verbatim}[commandchars=\\\{\}]
'\_\_main\_\_'
\end{Verbatim}
\end{tcolorbox}
        
El método especial \texttt{\_\_dict\_\_} devuelve un diccionario con las
propiedades y los valores de un objeto.

\begin{tcolorbox}[breakable, size=fbox, boxrule=1pt, pad at break*=1mm,colback=cellbackground, colframe=cellborder]
\prompt{In}{incolor}{5.34}{\boxspacing}
\begin{Verbatim}[commandchars=\\\{\}]
\PY{n}{a}\PY{o}{.}\PY{n+nv+vm}{\PYZus{}\PYZus{}dict\PYZus{}\PYZus{}}
\end{Verbatim}
\end{tcolorbox}

\begin{tcolorbox}[breakable, size=fbox, boxrule=.5pt, pad at break*=1mm, opacityfill=0]
\prompt{Out}{outcolor}{5.35}{\boxspacing}
\begin{Verbatim}[commandchars=\\\{\}]
\{'v': 2\}
\end{Verbatim}
\end{tcolorbox}
        
Una instancia contiene todos los atributos de la clase creados
automáticamente por Python. Si se agregan nuevos valores al objeto,
éstos son añadidos al diccionario.

\begin{tcolorbox}[breakable, size=fbox, boxrule=1pt, pad at break*=1mm,colback=cellbackground, colframe=cellborder]
\prompt{In}{incolor}{5.36}{\boxspacing}
\begin{Verbatim}[commandchars=\\\{\}]
\PY{n}{a}\PY{o}{.}\PY{n}{myVar} \PY{o}{=} \PY{l+m+mi}{10}
\PY{n}{a}\PY{o}{.}\PY{n+nv+vm}{\PYZus{}\PYZus{}dict\PYZus{}\PYZus{}}
\end{Verbatim}
\end{tcolorbox}

\begin{tcolorbox}[breakable, size=fbox, boxrule=.5pt, pad at break*=1mm, opacityfill=0]
\prompt{Out}{outcolor}{5.37}{\boxspacing}
\begin{Verbatim}[commandchars=\\\{\}]
\{'v': 2, 'myVar': 10\}
\end{Verbatim}
\end{tcolorbox}
        
\begin{tcolorbox}[breakable, size=fbox, boxrule=1pt, pad at break*=1mm,colback=cellbackground, colframe=cellborder]
\prompt{In}{incolor}{5.38}{\boxspacing}
\begin{Verbatim}[commandchars=\\\{\}]
\PY{n}{a}\PY{o}{.}\PY{n+nf+fm}{\PYZus{}\PYZus{}getattribute\PYZus{}\PYZus{}}\PY{p}{(}\PY{l+s+s2}{\PYZdq{}}\PY{l+s+s2}{v}\PY{l+s+s2}{\PYZdq{}}\PY{p}{)}
\PY{n}{a}\PY{o}{.}\PY{n+nf+fm}{\PYZus{}\PYZus{}getattribute\PYZus{}\PYZus{}}\PY{p}{(}\PY{l+s+s2}{\PYZdq{}}\PY{l+s+s2}{myVar}\PY{l+s+s2}{\PYZdq{}}\PY{p}{)}
\end{Verbatim}
\end{tcolorbox}

\begin{tcolorbox}[breakable, size=fbox, boxrule=.5pt, pad at break*=1mm, opacityfill=0]
\prompt{Out}{outcolor}{5.39}{\boxspacing}
\begin{Verbatim}[commandchars=\\\{\}]
10
\end{Verbatim}
\end{tcolorbox}
        
\begin{example}
Crear una clase que calcule la evaluación de la derivada de una función.
Utilizar una definición genérica para el cálculo de una aproximación de
la derivada.

\[ f'(x) \approx \dfrac{f(x+h) - f(x)}{h}. \]

    \begin{tcolorbox}[breakable, size=fbox, boxrule=1pt, pad at break*=1mm,colback=cellbackground, colframe=cellborder]
\prompt{In}{incolor}{5.40}{\boxspacing}
\begin{Verbatim}[commandchars=\\\{\}]
\PY{k}{class}\PY{+w}{ }\PY{n+nc}{Derivada}\PY{p}{:}
    \PY{k}{def}\PY{+w}{ }\PY{n+nf+fm}{\PYZus{}\PYZus{}init\PYZus{}\PYZus{}}\PY{p}{(}\PY{n+nb+bp}{self}\PY{p}{,} \PY{n}{f}\PY{p}{,} \PY{n}{h}\PY{o}{=}\PY{l+m+mf}{1e\PYZhy{}5}\PY{p}{)}\PY{p}{:}
        \PY{n+nb+bp}{self}\PY{o}{.}\PY{n}{f} \PY{o}{=} \PY{n}{f}  \PY{c+c1}{\PYZsh{} Función f(x) a derivar}
        \PY{n+nb+bp}{self}\PY{o}{.}\PY{n}{h} \PY{o}{=} \PY{n}{h}
    
    \PY{k}{def}\PY{+w}{ }\PY{n+nf+fm}{\PYZus{}\PYZus{}call\PYZus{}\PYZus{}}\PY{p}{(}\PY{n+nb+bp}{self}\PY{p}{,} \PY{n}{x}\PY{p}{)}\PY{p}{:}
        \PY{n}{f}\PY{p}{,} \PY{n}{h} \PY{o}{=} \PY{n+nb+bp}{self}\PY{o}{.}\PY{n}{f}\PY{p}{,} \PY{n+nb+bp}{self}\PY{o}{.}\PY{n}{h}
        \PY{k}{return} \PY{p}{(}\PY{n}{f}\PY{p}{(}\PY{n}{x}\PY{o}{+}\PY{n}{h}\PY{p}{)} \PY{o}{\PYZhy{}} \PY{n}{f}\PY{p}{(}\PY{n}{x}\PY{p}{)}\PY{p}{)}\PY{o}{/}\PY{n}{h}
\end{Verbatim}
\end{tcolorbox}

    \begin{tcolorbox}[breakable, size=fbox, boxrule=1pt, pad at break*=1mm,colback=cellbackground, colframe=cellborder]
\prompt{In}{incolor}{5.41}{\boxspacing}
\begin{Verbatim}[commandchars=\\\{\}]
\PY{k}{def}\PY{+w}{ }\PY{n+nf}{f}\PY{p}{(}\PY{n}{x}\PY{p}{)}\PY{p}{:}
    \PY{k}{return} \PY{n}{x}\PY{o}{*}\PY{o}{*}\PY{l+m+mi}{3}

\PY{n}{dfdx} \PY{o}{=} \PY{n}{Derivada}\PY{p}{(}\PY{n}{f}\PY{p}{)}

\PY{n}{dfdx}\PY{p}{(}\PY{l+m+mi}{1}\PY{p}{)}
\end{Verbatim}
\end{tcolorbox}

            \begin{tcolorbox}[breakable, size=fbox, boxrule=.5pt, pad at break*=1mm, opacityfill=0]
\prompt{Out}{outcolor}{5.42}{\boxspacing}
\begin{Verbatim}[commandchars=\\\{\}]
3.000030000110953
\end{Verbatim}
\end{tcolorbox}
\end{example}


\begin{example}
Crear una clase que calcule la aproximación de una integral definida.
Para ello consideremos la regla compuesta del trapecio.

\[ A = \int_a^b f(x)dx \approx \frac{h}{2}\left[f(a) + 2\sum_{j=1}^{n-1}f(x_j) + f(b) \right] \]

considere:

\[ h = \dfrac{b-a}{n} \]

y

\[ x_j = a + j\cdot h, \]

donde \(n\) es la cantidad de trapecios y \(h\) la altura de cada uno de
ellos.

\begin{tcolorbox}[breakable, size=fbox, boxrule=1pt, pad at break*=1mm,colback=cellbackground, colframe=cellborder]
\prompt{In}{incolor}{5.43}{\boxspacing}
\begin{Verbatim}[commandchars=\\\{\}]
\PY{k}{class}\PY{+w}{ }\PY{n+nc}{TrapecioCompuesta}\PY{p}{:}
    \PY{k}{def}\PY{+w}{ }\PY{n+nf+fm}{\PYZus{}\PYZus{}init\PYZus{}\PYZus{}}\PY{p}{(}\PY{n+nb+bp}{self}\PY{p}{,} \PY{n}{f}\PY{p}{,} \PY{n}{n}\PY{o}{=}\PY{l+m+mi}{1}\PY{p}{)}\PY{p}{:}
        \PY{n+nb+bp}{self}\PY{o}{.}\PY{n}{f}\PY{p}{,} \PY{n+nb+bp}{self}\PY{o}{.}\PY{n}{n} \PY{o}{=} \PY{n}{f}\PY{p}{,} \PY{n}{n}
    
    \PY{k}{def}\PY{+w}{ }\PY{n+nf+fm}{\PYZus{}\PYZus{}call\PYZus{}\PYZus{}}\PY{p}{(}\PY{n+nb+bp}{self}\PY{p}{,} \PY{n}{a}\PY{p}{,} \PY{n}{b}\PY{p}{)}\PY{p}{:}
        \PY{n}{f}\PY{p}{,} \PY{n}{n} \PY{o}{=} \PY{n+nb+bp}{self}\PY{o}{.}\PY{n}{f}\PY{p}{,} \PY{n+nb+bp}{self}\PY{o}{.}\PY{n}{n}
        \PY{n}{h} \PY{o}{=} \PY{p}{(}\PY{n}{b}\PY{o}{\PYZhy{}}\PY{n}{a}\PY{p}{)}\PY{o}{/}\PY{n}{n}
        \PY{n}{suma} \PY{o}{=} \PY{l+m+mi}{0}
        \PY{k}{for} \PY{n}{j} \PY{o+ow}{in} \PY{n+nb}{range}\PY{p}{(}\PY{l+m+mi}{1}\PY{p}{,}\PY{n}{n}\PY{p}{)}\PY{p}{:}
            \PY{n}{suma} \PY{o}{+}\PY{o}{=} \PY{n}{f}\PY{p}{(}\PY{n}{a} \PY{o}{+} \PY{n}{j}\PY{o}{*}\PY{n}{h}\PY{p}{)}

        \PY{k}{return} \PY{l+s+sa}{f}\PY{l+s+s1}{\PYZsq{}}\PY{l+s+s1}{A = }\PY{l+s+si}{\PYZob{}}\PY{n}{h}\PY{o}{/}\PY{l+m+mi}{2}\PY{o}{*}\PY{p}{(}\PY{n}{f}\PY{p}{(}\PY{n}{a}\PY{p}{)}\PY{+w}{ }\PY{o}{+}\PY{+w}{ }\PY{l+m+mi}{2}\PY{o}{*}\PY{n}{suma}\PY{+w}{ }\PY{o}{+}\PY{+w}{ }\PY{n}{f}\PY{p}{(}\PY{n}{b}\PY{p}{)}\PY{p}{)}\PY{l+s+si}{\PYZcb{}}\PY{l+s+s1}{ u\PYZca{}2}\PY{l+s+s1}{\PYZsq{}}
\end{Verbatim}
\end{tcolorbox}

\begin{tcolorbox}[breakable, size=fbox, boxrule=1pt, pad at break*=1mm,colback=cellbackground, colframe=cellborder]
\prompt{In}{incolor}{5.44}{\boxspacing}
\begin{Verbatim}[commandchars=\\\{\}]
\PY{k+kn}{import}\PY{+w}{ }\PY{n+nn}{math}
\PY{k}{def}\PY{+w}{ }\PY{n+nf}{f}\PY{p}{(}\PY{n}{x}\PY{p}{)}\PY{p}{:}
    \PY{k}{return} \PY{n}{math}\PY{o}{.}\PY{n}{sin}\PY{p}{(}\PY{n}{x}\PY{p}{)}

\PY{n}{intf} \PY{o}{=} \PY{n}{TrapecioCompuesta}\PY{p}{(}\PY{n}{f}\PY{p}{,} \PY{l+m+mi}{10000}\PY{p}{)}

\PY{n}{intf}\PY{p}{(}\PY{n}{math}\PY{o}{.}\PY{n}{pi}\PY{p}{,} \PY{l+m+mi}{2}\PY{o}{*}\PY{n}{math}\PY{o}{.}\PY{n}{pi}\PY{p}{)}
\end{Verbatim}
\end{tcolorbox}

\begin{tcolorbox}[breakable, size=fbox, boxrule=.5pt, pad at break*=1mm, opacityfill=0]
\prompt{Out}{outcolor}{5.45}{\boxspacing}
\begin{Verbatim}[commandchars=\\\{\}]
'A = -1.9999999835506608 u\^{}2'
\end{Verbatim}
\end{tcolorbox}
\end{example}
        
\begin{example}
Crear una clase que permita construir un polinomio

\[ P(x) = a_0 + a_1x + a_2x^2 \]

La clase debe incluir la funcionalidad de evaluar un polinomio en un
valor dado, y sumar dos polinomios. Se deben utilizar los métodos
especiales para usarlos de la forma indicada:

\begin{itemize}
\item
  \texttt{\_\_init\_\_}: para constuir un polinomio de la forma
  \texttt{p\ =\ Polynomial({[}1,-1{]})}
\item
  \texttt{\_\_str\_\_}: para imprimir el polinomio
\item
  \texttt{\_\_call\_\_}: para evaluar el polinomio de la forma
  \texttt{p(2.0)}
\item
  \texttt{\_\_add\_\_}: para realizar la suma de polinomios
\item
  \texttt{\_\_mul\_\_}: para realizar la multiplicación de polinomios
\end{itemize}

Además se debe incluir un método para realizar la derivada del
polinomio.

\subsubsection{Solución}

Creación de la clase Polynomial, el constructor y el método ´\textbf{call}´

\begin{tcolorbox}[breakable, size=fbox, boxrule=1pt, pad at break*=1mm,colback=cellbackground, colframe=cellborder]
\prompt{In}{incolor}{5.46}{\boxspacing}
\begin{Verbatim}[commandchars=\\\{\}]
\PY{k}{class}\PY{+w}{ }\PY{n+nc}{Polynomial}\PY{p}{:}
    \PY{k}{def}\PY{+w}{ }\PY{n+nf+fm}{\PYZus{}\PYZus{}init\PYZus{}\PYZus{}}\PY{p}{(}\PY{n+nb+bp}{self}\PY{p}{,} \PY{n}{coefficients}\PY{p}{)}\PY{p}{:}
        \PY{n+nb+bp}{self}\PY{o}{.}\PY{n}{coeff} \PY{o}{=} \PY{n}{coefficients}

    \PY{k}{def}\PY{+w}{ }\PY{n+nf+fm}{\PYZus{}\PYZus{}call\PYZus{}\PYZus{}}\PY{p}{(}\PY{n+nb+bp}{self}\PY{p}{,} \PY{n}{x}\PY{p}{)}\PY{p}{:}
        \PY{n}{s} \PY{o}{=} \PY{l+m+mi}{0}
        \PY{k}{for} \PY{n}{i} \PY{o+ow}{in} \PY{n+nb}{range}\PY{p}{(}\PY{n+nb}{len}\PY{p}{(}\PY{n+nb+bp}{self}\PY{o}{.}\PY{n}{coeff}\PY{p}{)}\PY{p}{)}\PY{p}{:}
            \PY{n}{s} \PY{o}{+}\PY{o}{=} \PY{n+nb+bp}{self}\PY{o}{.}\PY{n}{coeff}\PY{p}{[}\PY{n}{i}\PY{p}{]}\PY{o}{*}\PY{n}{x}\PY{o}{*}\PY{o}{*}\PY{n}{i}
        \PY{k}{return} \PY{n}{s}
\end{Verbatim}
\end{tcolorbox}

\begin{tcolorbox}[breakable, size=fbox, boxrule=1pt, pad at break*=1mm,colback=cellbackground, colframe=cellborder]
\prompt{In}{incolor}{5.47}{\boxspacing}
\begin{Verbatim}[commandchars=\\\{\}]
\PY{n}{p1} \PY{o}{=} \PY{n}{Polynomial}\PY{p}{(}\PY{p}{[}\PY{l+m+mi}{1}\PY{p}{,}\PY{o}{\PYZhy{}}\PY{l+m+mi}{1}\PY{p}{]}\PY{p}{)}
\PY{n}{p1}\PY{p}{(}\PY{l+m+mi}{4}\PY{p}{)}
\end{Verbatim}
\end{tcolorbox}

\begin{tcolorbox}[breakable, size=fbox, boxrule=.5pt, pad at break*=1mm, opacityfill=0]
\prompt{Out}{outcolor}{5.48}{\boxspacing}
\begin{Verbatim}[commandchars=\\\{\}]
-3
\end{Verbatim}
\end{tcolorbox}
        
Implementación de la suma de polinomios

\begin{tcolorbox}[breakable, size=fbox, boxrule=1pt, pad at break*=1mm,colback=cellbackground, colframe=cellborder]
\prompt{In}{incolor}{5.49}{\boxspacing}
\begin{Verbatim}[commandchars=\\\{\}]
\PY{k}{class}\PY{+w}{ }\PY{n+nc}{Polynomial}\PY{p}{:}
    \PY{k}{def}\PY{+w}{ }\PY{n+nf+fm}{\PYZus{}\PYZus{}init\PYZus{}\PYZus{}}\PY{p}{(}\PY{n+nb+bp}{self}\PY{p}{,} \PY{n}{coefficients}\PY{p}{)}\PY{p}{:}
        \PY{n+nb+bp}{self}\PY{o}{.}\PY{n}{coeff} \PY{o}{=} \PY{n}{coefficients}

    \PY{k}{def}\PY{+w}{ }\PY{n+nf+fm}{\PYZus{}\PYZus{}call\PYZus{}\PYZus{}}\PY{p}{(}\PY{n+nb+bp}{self}\PY{p}{,} \PY{n}{x}\PY{p}{)}\PY{p}{:}
        \PY{n}{s} \PY{o}{=} \PY{l+m+mi}{0}
        \PY{k}{for} \PY{n}{i} \PY{o+ow}{in} \PY{n+nb}{range}\PY{p}{(}\PY{n+nb}{len}\PY{p}{(}\PY{n+nb+bp}{self}\PY{o}{.}\PY{n}{coeff}\PY{p}{)}\PY{p}{)}\PY{p}{:}
            \PY{n}{s} \PY{o}{+}\PY{o}{=} \PY{n+nb+bp}{self}\PY{o}{.}\PY{n}{coeff}\PY{p}{[}\PY{n}{i}\PY{p}{]}\PY{o}{*}\PY{n}{x}\PY{o}{*}\PY{o}{*}\PY{n}{i}
        \PY{k}{return} \PY{n}{s}
    
    \PY{k}{def}\PY{+w}{ }\PY{n+nf+fm}{\PYZus{}\PYZus{}add\PYZus{}\PYZus{}}\PY{p}{(}\PY{n+nb+bp}{self}\PY{p}{,} \PY{n}{other}\PY{p}{)}\PY{p}{:}
        \PY{c+c1}{\PYZsh{} return self + other}

        \PY{c+c1}{\PYZsh{} we start with longest list and add it to the other}
        \PY{k}{if} \PY{n+nb}{len}\PY{p}{(}\PY{n+nb+bp}{self}\PY{o}{.}\PY{n}{coeff}\PY{p}{)} \PY{o}{\PYZgt{}} \PY{n+nb}{len}\PY{p}{(}\PY{n}{other}\PY{o}{.}\PY{n}{coeff}\PY{p}{)}\PY{p}{:}
            \PY{n}{coeffsum} \PY{o}{=} \PY{n+nb+bp}{self}\PY{o}{.}\PY{n}{coeff}\PY{p}{[}\PY{p}{:}\PY{p}{]} \PY{c+c1}{\PYZsh{} copy list}
            \PY{k}{for} \PY{n}{i} \PY{o+ow}{in} \PY{n+nb}{range}\PY{p}{(}\PY{n+nb}{len}\PY{p}{(}\PY{n}{other}\PY{o}{.}\PY{n}{coeff}\PY{p}{)}\PY{p}{)}\PY{p}{:}
                \PY{n}{coeffsum}\PY{p}{[}\PY{n}{i}\PY{p}{]} \PY{o}{+}\PY{o}{=} \PY{n}{other}\PY{o}{.}\PY{n}{coeff}\PY{p}{[}\PY{n}{i}\PY{p}{]}

        \PY{k}{else}\PY{p}{:} 
            \PY{n}{coeffsum} \PY{o}{=} \PY{n}{other}\PY{o}{.}\PY{n}{coeff}\PY{p}{[}\PY{p}{:}\PY{p}{]} \PY{c+c1}{\PYZsh{} copy list}
            \PY{k}{for} \PY{n}{i} \PY{o+ow}{in} \PY{n+nb}{range}\PY{p}{(}\PY{n+nb}{len}\PY{p}{(}\PY{n+nb+bp}{self}\PY{o}{.}\PY{n}{coeff}\PY{p}{)}\PY{p}{)}\PY{p}{:}
                \PY{n}{coeffsum}\PY{p}{[}\PY{n}{i}\PY{p}{]} \PY{o}{+}\PY{o}{=} \PY{n+nb+bp}{self}\PY{o}{.}\PY{n}{coeff}\PY{p}{[}\PY{n}{i}\PY{p}{]}
            
        \PY{k}{return} \PY{n}{Polynomial}\PY{p}{(}\PY{n}{coeffsum}\PY{p}{)}
\end{Verbatim}
\end{tcolorbox}

\begin{tcolorbox}[breakable, size=fbox, boxrule=1pt, pad at break*=1mm,colback=cellbackground, colframe=cellborder]
\prompt{In}{incolor}{5.50}{\boxspacing}
\begin{Verbatim}[commandchars=\\\{\}]
\PY{n}{p1} \PY{o}{=} \PY{n}{Polynomial}\PY{p}{(}\PY{p}{[}\PY{o}{\PYZhy{}}\PY{l+m+mi}{3}\PY{p}{,} \PY{l+m+mi}{0}\PY{p}{,} \PY{l+m+mi}{2}\PY{p}{,} \PY{l+m+mi}{1}\PY{p}{]}\PY{p}{)}     \PY{c+c1}{\PYZsh{} x³ + 2x² \PYZhy{}3}
\PY{n}{p2} \PY{o}{=} \PY{n}{Polynomial}\PY{p}{(}\PY{p}{[}\PY{l+m+mi}{1}\PY{p}{,} \PY{l+m+mi}{1}\PY{p}{,} \PY{l+m+mi}{1}\PY{p}{]}\PY{p}{)}        \PY{c+c1}{\PYZsh{} x² + x + 1}
\PY{n}{p3} \PY{o}{=} \PY{n}{p1} \PY{o}{+} \PY{n}{p2}    \PY{c+c1}{\PYZsh{} x³ + 3x² + x \PYZhy{} 2}
\PY{n+nb}{print}\PY{p}{(}\PY{n}{p3}\PY{o}{.}\PY{n}{coeff}\PY{p}{)}
\end{Verbatim}
\end{tcolorbox}

\begin{Verbatim}[commandchars=\\\{\}]
[-2, 1, 3, 1]
\end{Verbatim}

Para la multiplicación se requiere realizar un proceso un poco más
complicado. Nos referiremos a la expresión matemática para la
multiplicación de polinomios.

\[ p_1 \cdot p_2 = \left( \sum_{i=0}^M c_ix^i \right) \left( \sum_{j=0}^N d_jx^j \right) = \sum_{i=0}^M \sum_{j=0}^N c_id_jx^{i+j}, \]

donde

\[p_1 = c_0 + c_1x + c_2x^2 + \dots + c_Mx^M \] y
\[p_2 = d_0 + d_1x + d_2x^2 + \dots + d_Nx^N \]

\begin{tcolorbox}[breakable, size=fbox, boxrule=1pt, pad at break*=1mm,colback=cellbackground, colframe=cellborder]
\prompt{In}{incolor}{5.51}{\boxspacing}
\begin{Verbatim}[commandchars=\\\{\}]
\PY{k}{class}\PY{+w}{ }\PY{n+nc}{Polynomial}\PY{p}{:}
    \PY{k}{def}\PY{+w}{ }\PY{n+nf+fm}{\PYZus{}\PYZus{}init\PYZus{}\PYZus{}}\PY{p}{(}\PY{n+nb+bp}{self}\PY{p}{,} \PY{n}{coefficients}\PY{p}{)}\PY{p}{:}
        \PY{n+nb+bp}{self}\PY{o}{.}\PY{n}{coeff} \PY{o}{=} \PY{n}{coefficients}

    \PY{k}{def}\PY{+w}{ }\PY{n+nf+fm}{\PYZus{}\PYZus{}call\PYZus{}\PYZus{}}\PY{p}{(}\PY{n+nb+bp}{self}\PY{p}{,} \PY{n}{x}\PY{p}{)}\PY{p}{:}
        \PY{n}{s} \PY{o}{=} \PY{l+m+mi}{0}
        \PY{k}{for} \PY{n}{i} \PY{o+ow}{in} \PY{n+nb}{range}\PY{p}{(}\PY{n+nb}{len}\PY{p}{(}\PY{n+nb+bp}{self}\PY{o}{.}\PY{n}{coeff}\PY{p}{)}\PY{p}{)}\PY{p}{:}
            \PY{n}{s} \PY{o}{+}\PY{o}{=} \PY{n+nb+bp}{self}\PY{o}{.}\PY{n}{coeff}\PY{p}{[}\PY{n}{i}\PY{p}{]}\PY{o}{*}\PY{n}{x}\PY{o}{*}\PY{o}{*}\PY{n}{i}
        \PY{k}{return} \PY{n}{s}
    
    \PY{k}{def}\PY{+w}{ }\PY{n+nf+fm}{\PYZus{}\PYZus{}add\PYZus{}\PYZus{}}\PY{p}{(}\PY{n+nb+bp}{self}\PY{p}{,} \PY{n}{other}\PY{p}{)}\PY{p}{:}
        \PY{c+c1}{\PYZsh{} return self + other}

        \PY{c+c1}{\PYZsh{} we start with longest list and add it to the other}
        \PY{k}{if} \PY{n+nb}{len}\PY{p}{(}\PY{n+nb+bp}{self}\PY{o}{.}\PY{n}{coeff}\PY{p}{)} \PY{o}{\PYZgt{}} \PY{n+nb}{len}\PY{p}{(}\PY{n}{other}\PY{o}{.}\PY{n}{coeff}\PY{p}{)}\PY{p}{:}
            \PY{n}{coeffsum} \PY{o}{=} \PY{n+nb+bp}{self}\PY{o}{.}\PY{n}{coeff}\PY{p}{[}\PY{p}{:}\PY{p}{]} \PY{c+c1}{\PYZsh{} copy list}
            \PY{k}{for} \PY{n}{i} \PY{o+ow}{in} \PY{n+nb}{range}\PY{p}{(}\PY{n+nb}{len}\PY{p}{(}\PY{n}{other}\PY{o}{.}\PY{n}{coeff}\PY{p}{)}\PY{p}{)}\PY{p}{:}
                \PY{n}{coeffsum}\PY{p}{[}\PY{n}{i}\PY{p}{]} \PY{o}{+}\PY{o}{=} \PY{n}{other}\PY{o}{.}\PY{n}{coeff}\PY{p}{[}\PY{n}{i}\PY{p}{]}

        \PY{k}{else}\PY{p}{:} 
            \PY{n}{coeffsum} \PY{o}{=} \PY{n}{other}\PY{o}{.}\PY{n}{coeff}\PY{p}{[}\PY{p}{:}\PY{p}{]} \PY{c+c1}{\PYZsh{} copy list}
            \PY{k}{for} \PY{n}{i} \PY{o+ow}{in} \PY{n+nb}{range}\PY{p}{(}\PY{n+nb}{len}\PY{p}{(}\PY{n+nb+bp}{self}\PY{o}{.}\PY{n}{coeff}\PY{p}{)}\PY{p}{)}\PY{p}{:}
                \PY{n}{coeffsum}\PY{p}{[}\PY{n}{i}\PY{p}{]} \PY{o}{+}\PY{o}{=} \PY{n+nb+bp}{self}\PY{o}{.}\PY{n}{coeff}\PY{p}{[}\PY{n}{i}\PY{p}{]}
            
        \PY{k}{return} \PY{n}{Polynomial}\PY{p}{(}\PY{n}{coeffsum}\PY{p}{)}

    \PY{k}{def}\PY{+w}{ }\PY{n+nf+fm}{\PYZus{}\PYZus{}mul\PYZus{}\PYZus{}}\PY{p}{(}\PY{n+nb+bp}{self}\PY{p}{,} \PY{n}{other}\PY{p}{)}\PY{p}{:}
        \PY{n}{M} \PY{o}{=} \PY{n+nb}{len}\PY{p}{(}\PY{n+nb+bp}{self}\PY{o}{.}\PY{n}{coeff}\PY{p}{)} \PY{o}{\PYZhy{}} \PY{l+m+mi}{1}
        \PY{n}{N} \PY{o}{=} \PY{n+nb}{len}\PY{p}{(}\PY{n}{other}\PY{o}{.}\PY{n}{coeff}\PY{p}{)} \PY{o}{\PYZhy{}} \PY{l+m+mi}{1}
        \PY{n}{coeff} \PY{o}{=} \PY{p}{[}\PY{l+m+mi}{0}\PY{p}{]}\PY{o}{*}\PY{p}{(}\PY{n}{M}\PY{o}{+}\PY{n}{N}\PY{o}{+}\PY{l+m+mi}{1}\PY{p}{)} \PY{c+c1}{\PYZsh{} [0 for i in range(10)]    \PYZsh{} List of M+N+1 zeros}
        \PY{k}{for} \PY{n}{i} \PY{o+ow}{in} \PY{n+nb}{range}\PY{p}{(}\PY{l+m+mi}{0}\PY{p}{,} \PY{n}{M}\PY{o}{+}\PY{l+m+mi}{1}\PY{p}{)}\PY{p}{:} 
            \PY{k}{for} \PY{n}{j} \PY{o+ow}{in} \PY{n+nb}{range}\PY{p}{(}\PY{l+m+mi}{0}\PY{p}{,} \PY{n}{N}\PY{o}{+}\PY{l+m+mi}{1}\PY{p}{)}\PY{p}{:}
                \PY{n}{coeff}\PY{p}{[}\PY{n}{i}\PY{o}{+}\PY{n}{j}\PY{p}{]} \PY{o}{+}\PY{o}{=} \PY{n+nb+bp}{self}\PY{o}{.}\PY{n}{coeff}\PY{p}{[}\PY{n}{i}\PY{p}{]} \PY{o}{*} \PY{n}{other}\PY{o}{.}\PY{n}{coeff}\PY{p}{[}\PY{n}{j}\PY{p}{]}
        
        \PY{k}{return} \PY{n}{Polynomial}\PY{p}{(}\PY{n}{coeff}\PY{p}{)}
\end{Verbatim}
\end{tcolorbox}

    \begin{tcolorbox}[breakable, size=fbox, boxrule=1pt, pad at break*=1mm,colback=cellbackground, colframe=cellborder]
\prompt{In}{incolor}{5.52}{\boxspacing}
\begin{Verbatim}[commandchars=\\\{\}]
\PY{n}{p1} \PY{o}{=} \PY{n}{Polynomial}\PY{p}{(}\PY{p}{[}\PY{o}{\PYZhy{}}\PY{l+m+mi}{3}\PY{p}{,} \PY{l+m+mi}{0}\PY{p}{,} \PY{l+m+mi}{2}\PY{p}{,} \PY{l+m+mi}{1}\PY{p}{]}\PY{p}{)}     \PY{c+c1}{\PYZsh{} x³ + 2x² \PYZhy{}3}
\PY{n}{p2} \PY{o}{=} \PY{n}{Polynomial}\PY{p}{(}\PY{p}{[}\PY{l+m+mi}{1}\PY{p}{,} \PY{l+m+mi}{1}\PY{p}{,} \PY{l+m+mi}{1}\PY{p}{]}\PY{p}{)}        \PY{c+c1}{\PYZsh{} x² + x + 1}
\PY{n}{p4} \PY{o}{=} \PY{n}{p1} \PY{o}{*} \PY{n}{p2}    \PY{c+c1}{\PYZsh{} x³ + 3x² + x \PYZhy{} 2}
\PY{n+nb}{print}\PY{p}{(}\PY{n}{p4}\PY{o}{.}\PY{n}{coeff}\PY{p}{)}
\end{Verbatim}
\end{tcolorbox}

\begin{Verbatim}[commandchars=\\\{\}]
[-3, -3, -1, 3, 3, 1]
\end{Verbatim}

Para el cálculo de la derivada del polinomio, se puede utilizar la regla:

\[ \dfrac{d}{dx}\sum_{i=0}^n c_ix^i = \sum_{i=1}^n ic_ix^{i-1} \]

Por lo tanto, si \(c\) es la lista de coeficientes del polinomio, la
derivada tiene una lista de coeficientes en \(dc\), donde
\(dc[i-1] = i\cdot c[i]\) para todos los valores de \(i\) desde 1 hasta
el índice mayor en \(c\). Recuerde que la derivada de un polinomio se
reduce en grado en 1, por lo tanto la lista \(dc\) tendrá un elemento
menos que la lista \(c\).

\begin{tcolorbox}[breakable, size=fbox, boxrule=1pt, pad at break*=1mm,colback=cellbackground, colframe=cellborder]
\prompt{In}{incolor}{5.53}{\boxspacing}
\begin{Verbatim}[commandchars=\\\{\}]
\PY{k}{class}\PY{+w}{ }\PY{n+nc}{Polynomial}\PY{p}{:}
    \PY{k}{def}\PY{+w}{ }\PY{n+nf+fm}{\PYZus{}\PYZus{}init\PYZus{}\PYZus{}}\PY{p}{(}\PY{n+nb+bp}{self}\PY{p}{,} \PY{n}{coefficients}\PY{p}{)}\PY{p}{:}
        \PY{n+nb+bp}{self}\PY{o}{.}\PY{n}{coeff} \PY{o}{=} \PY{n}{coefficients}

    \PY{k}{def}\PY{+w}{ }\PY{n+nf+fm}{\PYZus{}\PYZus{}call\PYZus{}\PYZus{}}\PY{p}{(}\PY{n+nb+bp}{self}\PY{p}{,} \PY{n}{x}\PY{p}{)}\PY{p}{:}
        \PY{n}{s} \PY{o}{=} \PY{l+m+mi}{0}
        \PY{k}{for} \PY{n}{i} \PY{o+ow}{in} \PY{n+nb}{range}\PY{p}{(}\PY{n+nb}{len}\PY{p}{(}\PY{n+nb+bp}{self}\PY{o}{.}\PY{n}{coeff}\PY{p}{)}\PY{p}{)}\PY{p}{:}
            \PY{n}{s} \PY{o}{+}\PY{o}{=} \PY{n+nb+bp}{self}\PY{o}{.}\PY{n}{coeff}\PY{p}{[}\PY{n}{i}\PY{p}{]}\PY{o}{*}\PY{n}{x}\PY{o}{*}\PY{o}{*}\PY{n}{i}
        \PY{k}{return} \PY{n}{s}
    
    \PY{k}{def}\PY{+w}{ }\PY{n+nf+fm}{\PYZus{}\PYZus{}add\PYZus{}\PYZus{}}\PY{p}{(}\PY{n+nb+bp}{self}\PY{p}{,} \PY{n}{other}\PY{p}{)}\PY{p}{:}
        \PY{c+c1}{\PYZsh{} return self + other}

        \PY{c+c1}{\PYZsh{} we start with longest list and add it to the other}
        \PY{k}{if} \PY{n+nb}{len}\PY{p}{(}\PY{n+nb+bp}{self}\PY{o}{.}\PY{n}{coeff}\PY{p}{)} \PY{o}{\PYZgt{}} \PY{n+nb}{len}\PY{p}{(}\PY{n}{other}\PY{o}{.}\PY{n}{coeff}\PY{p}{)}\PY{p}{:}
            \PY{n}{coeffsum} \PY{o}{=} \PY{n+nb+bp}{self}\PY{o}{.}\PY{n}{coeff}\PY{p}{[}\PY{p}{:}\PY{p}{]} \PY{c+c1}{\PYZsh{} copy list}
            \PY{k}{for} \PY{n}{i} \PY{o+ow}{in} \PY{n+nb}{range}\PY{p}{(}\PY{n+nb}{len}\PY{p}{(}\PY{n}{other}\PY{o}{.}\PY{n}{coeff}\PY{p}{)}\PY{p}{)}\PY{p}{:}
                \PY{n}{coeffsum}\PY{p}{[}\PY{n}{i}\PY{p}{]} \PY{o}{+}\PY{o}{=} \PY{n}{other}\PY{o}{.}\PY{n}{coeff}\PY{p}{[}\PY{n}{i}\PY{p}{]}

        \PY{k}{else}\PY{p}{:} 
            \PY{n}{coeffsum} \PY{o}{=} \PY{n}{other}\PY{o}{.}\PY{n}{coeff}\PY{p}{[}\PY{p}{:}\PY{p}{]} \PY{c+c1}{\PYZsh{} copy list}
            \PY{k}{for} \PY{n}{i} \PY{o+ow}{in} \PY{n+nb}{range}\PY{p}{(}\PY{n+nb}{len}\PY{p}{(}\PY{n+nb+bp}{self}\PY{o}{.}\PY{n}{coeff}\PY{p}{)}\PY{p}{)}\PY{p}{:}
                \PY{n}{coeffsum}\PY{p}{[}\PY{n}{i}\PY{p}{]} \PY{o}{+}\PY{o}{=} \PY{n+nb+bp}{self}\PY{o}{.}\PY{n}{coeff}\PY{p}{[}\PY{n}{i}\PY{p}{]}
            
        \PY{k}{return} \PY{n}{Polynomial}\PY{p}{(}\PY{n}{coeffsum}\PY{p}{)}

    \PY{k}{def}\PY{+w}{ }\PY{n+nf+fm}{\PYZus{}\PYZus{}mul\PYZus{}\PYZus{}}\PY{p}{(}\PY{n+nb+bp}{self}\PY{p}{,} \PY{n}{other}\PY{p}{)}\PY{p}{:}
        \PY{n}{M} \PY{o}{=} \PY{n+nb}{len}\PY{p}{(}\PY{n+nb+bp}{self}\PY{o}{.}\PY{n}{coeff}\PY{p}{)} \PY{o}{\PYZhy{}} \PY{l+m+mi}{1}
        \PY{n}{N} \PY{o}{=} \PY{n+nb}{len}\PY{p}{(}\PY{n}{other}\PY{o}{.}\PY{n}{coeff}\PY{p}{)} \PY{o}{\PYZhy{}} \PY{l+m+mi}{1}
        \PY{n}{coeff} \PY{o}{=} \PY{p}{[}\PY{l+m+mi}{0}\PY{p}{]}\PY{o}{*}\PY{p}{(}\PY{n}{M}\PY{o}{+}\PY{n}{N}\PY{o}{+}\PY{l+m+mi}{1}\PY{p}{)} \PY{c+c1}{\PYZsh{} [0 for i in range(10)]    \PYZsh{} List of M+N+1 zeros}
        \PY{k}{for} \PY{n}{i} \PY{o+ow}{in} \PY{n+nb}{range}\PY{p}{(}\PY{l+m+mi}{0}\PY{p}{,} \PY{n}{M}\PY{o}{+}\PY{l+m+mi}{1}\PY{p}{)}\PY{p}{:} 
            \PY{k}{for} \PY{n}{j} \PY{o+ow}{in} \PY{n+nb}{range}\PY{p}{(}\PY{l+m+mi}{0}\PY{p}{,} \PY{n}{N}\PY{o}{+}\PY{l+m+mi}{1}\PY{p}{)}\PY{p}{:}
                \PY{n}{coeff}\PY{p}{[}\PY{n}{i}\PY{o}{+}\PY{n}{j}\PY{p}{]} \PY{o}{+}\PY{o}{=} \PY{n+nb+bp}{self}\PY{o}{.}\PY{n}{coeff}\PY{p}{[}\PY{n}{i}\PY{p}{]} \PY{o}{*} \PY{n}{other}\PY{o}{.}\PY{n}{coeff}\PY{p}{[}\PY{n}{j}\PY{p}{]}
        
        \PY{k}{return} \PY{n}{Polynomial}\PY{p}{(}\PY{n}{coeff}\PY{p}{)}

    \PY{k}{def}\PY{+w}{ }\PY{n+nf}{differentiate}\PY{p}{(}\PY{n+nb+bp}{self}\PY{p}{)}\PY{p}{:}
        \PY{k}{for} \PY{n}{i} \PY{o+ow}{in} \PY{n+nb}{range}\PY{p}{(}\PY{l+m+mi}{1}\PY{p}{,} \PY{n+nb}{len}\PY{p}{(}\PY{n+nb+bp}{self}\PY{o}{.}\PY{n}{coeff}\PY{p}{)}\PY{p}{)}\PY{p}{:}
            \PY{n+nb+bp}{self}\PY{o}{.}\PY{n}{coeff}\PY{p}{[}\PY{n}{i}\PY{o}{\PYZhy{}}\PY{l+m+mi}{1}\PY{p}{]} \PY{o}{=} \PY{n}{i} \PY{o}{*} \PY{n+nb+bp}{self}\PY{o}{.}\PY{n}{coeff}\PY{p}{[}\PY{n}{i}\PY{p}{]}
        \PY{k}{del} \PY{n+nb+bp}{self}\PY{o}{.}\PY{n}{coeff}\PY{p}{[}\PY{o}{\PYZhy{}}\PY{l+m+mi}{1}\PY{p}{]}

    \PY{k}{def}\PY{+w}{ }\PY{n+nf}{derivative}\PY{p}{(}\PY{n+nb+bp}{self}\PY{p}{)}\PY{p}{:}
        \PY{n}{dpdx} \PY{o}{=} \PY{n}{Polynomial}\PY{p}{(}\PY{n+nb+bp}{self}\PY{o}{.}\PY{n}{coeff}\PY{p}{[}\PY{p}{:}\PY{p}{]}\PY{p}{)}
        \PY{n}{dpdx}\PY{o}{.}\PY{n}{differentiate}\PY{p}{(}\PY{p}{)}
        \PY{k}{return} \PY{n}{dpdx}
\end{Verbatim}
\end{tcolorbox}

\begin{tcolorbox}[breakable, size=fbox, boxrule=1pt, pad at break*=1mm,colback=cellbackground, colframe=cellborder]
\prompt{In}{incolor}{5.54}{\boxspacing}
\begin{Verbatim}[commandchars=\\\{\}]
\PY{n}{p1} \PY{o}{=} \PY{n}{Polynomial}\PY{p}{(}\PY{p}{[}\PY{o}{\PYZhy{}}\PY{l+m+mi}{3}\PY{p}{,} \PY{l+m+mi}{0}\PY{p}{,} \PY{l+m+mi}{2}\PY{p}{,} \PY{l+m+mi}{1}\PY{p}{]}\PY{p}{)}     \PY{c+c1}{\PYZsh{} x³ + 2x² \PYZhy{}3}
\PY{n}{p1}\PY{o}{.}\PY{n}{derivative}\PY{p}{(}\PY{p}{)}
\end{Verbatim}
\end{tcolorbox}

\begin{tcolorbox}[breakable, size=fbox, boxrule=.5pt, pad at break*=1mm, opacityfill=0]
\prompt{Out}{outcolor}{5.55}{\boxspacing}
\begin{Verbatim}[commandchars=\\\{\}]
<\_\_main\_\_.Polynomial at 0x7fa9cc5e0fd0>
\end{Verbatim}
\end{tcolorbox}
        
Por último, hace falta agregar la función \texttt{\_\_str\_\_} para
mostrar el polinomio de una forma legible. El método debe devolver una
representación del polinomio lo más cercana posible a como se escribe en
Matemáticas.

\begin{tcolorbox}[breakable, size=fbox, boxrule=1pt, pad at break*=1mm,colback=cellbackground, colframe=cellborder]
\prompt{In}{incolor}{5.56}{\boxspacing}
\begin{Verbatim}[commandchars=\\\{\}]
\PY{k}{class}\PY{+w}{ }\PY{n+nc}{Polynomial}\PY{p}{:}
    \PY{k}{def}\PY{+w}{ }\PY{n+nf+fm}{\PYZus{}\PYZus{}init\PYZus{}\PYZus{}}\PY{p}{(}\PY{n+nb+bp}{self}\PY{p}{,} \PY{n}{coefficients}\PY{p}{)}\PY{p}{:}
        \PY{n+nb+bp}{self}\PY{o}{.}\PY{n}{coeff} \PY{o}{=} \PY{n}{coefficients}

    \PY{k}{def}\PY{+w}{ }\PY{n+nf+fm}{\PYZus{}\PYZus{}call\PYZus{}\PYZus{}}\PY{p}{(}\PY{n+nb+bp}{self}\PY{p}{,} \PY{n}{x}\PY{p}{)}\PY{p}{:}
        \PY{n}{s} \PY{o}{=} \PY{l+m+mi}{0}
        \PY{k}{for} \PY{n}{i} \PY{o+ow}{in} \PY{n+nb}{range}\PY{p}{(}\PY{n+nb}{len}\PY{p}{(}\PY{n+nb+bp}{self}\PY{o}{.}\PY{n}{coeff}\PY{p}{)}\PY{p}{)}\PY{p}{:}
            \PY{n}{s} \PY{o}{+}\PY{o}{=} \PY{n+nb+bp}{self}\PY{o}{.}\PY{n}{coeff}\PY{p}{[}\PY{n}{i}\PY{p}{]}\PY{o}{*}\PY{n}{x}\PY{o}{*}\PY{o}{*}\PY{n}{i}
        \PY{k}{return} \PY{n}{s}
    
    \PY{k}{def}\PY{+w}{ }\PY{n+nf+fm}{\PYZus{}\PYZus{}add\PYZus{}\PYZus{}}\PY{p}{(}\PY{n+nb+bp}{self}\PY{p}{,} \PY{n}{other}\PY{p}{)}\PY{p}{:}
        \PY{c+c1}{\PYZsh{} return self + other}

        \PY{c+c1}{\PYZsh{} we start with longest list and add it to the other}
        \PY{k}{if} \PY{n+nb}{len}\PY{p}{(}\PY{n+nb+bp}{self}\PY{o}{.}\PY{n}{coeff}\PY{p}{)} \PY{o}{\PYZgt{}} \PY{n+nb}{len}\PY{p}{(}\PY{n}{other}\PY{o}{.}\PY{n}{coeff}\PY{p}{)}\PY{p}{:}
            \PY{n}{coeffsum} \PY{o}{=} \PY{n+nb+bp}{self}\PY{o}{.}\PY{n}{coeff}\PY{p}{[}\PY{p}{:}\PY{p}{]} \PY{c+c1}{\PYZsh{} copy list}
            \PY{k}{for} \PY{n}{i} \PY{o+ow}{in} \PY{n+nb}{range}\PY{p}{(}\PY{n+nb}{len}\PY{p}{(}\PY{n}{other}\PY{o}{.}\PY{n}{coeff}\PY{p}{)}\PY{p}{)}\PY{p}{:}
                \PY{n}{coeffsum}\PY{p}{[}\PY{n}{i}\PY{p}{]} \PY{o}{+}\PY{o}{=} \PY{n}{other}\PY{o}{.}\PY{n}{coeff}\PY{p}{[}\PY{n}{i}\PY{p}{]}

        \PY{k}{else}\PY{p}{:} 
            \PY{n}{coeffsum} \PY{o}{=} \PY{n}{other}\PY{o}{.}\PY{n}{coeff}\PY{p}{[}\PY{p}{:}\PY{p}{]} \PY{c+c1}{\PYZsh{} copy list}
            \PY{k}{for} \PY{n}{i} \PY{o+ow}{in} \PY{n+nb}{range}\PY{p}{(}\PY{n+nb}{len}\PY{p}{(}\PY{n+nb+bp}{self}\PY{o}{.}\PY{n}{coeff}\PY{p}{)}\PY{p}{)}\PY{p}{:}
                \PY{n}{coeffsum}\PY{p}{[}\PY{n}{i}\PY{p}{]} \PY{o}{+}\PY{o}{=} \PY{n+nb+bp}{self}\PY{o}{.}\PY{n}{coeff}\PY{p}{[}\PY{n}{i}\PY{p}{]}
            
        \PY{k}{return} \PY{n}{Polynomial}\PY{p}{(}\PY{n}{coeffsum}\PY{p}{)}

    \PY{k}{def}\PY{+w}{ }\PY{n+nf+fm}{\PYZus{}\PYZus{}mul\PYZus{}\PYZus{}}\PY{p}{(}\PY{n+nb+bp}{self}\PY{p}{,} \PY{n}{other}\PY{p}{)}\PY{p}{:}
        \PY{n}{M} \PY{o}{=} \PY{n+nb}{len}\PY{p}{(}\PY{n+nb+bp}{self}\PY{o}{.}\PY{n}{coeff}\PY{p}{)} \PY{o}{\PYZhy{}} \PY{l+m+mi}{1}
        \PY{n}{N} \PY{o}{=} \PY{n+nb}{len}\PY{p}{(}\PY{n}{other}\PY{o}{.}\PY{n}{coeff}\PY{p}{)} \PY{o}{\PYZhy{}} \PY{l+m+mi}{1}
        \PY{n}{coeff} \PY{o}{=} \PY{p}{[}\PY{l+m+mi}{0}\PY{p}{]}\PY{o}{*}\PY{p}{(}\PY{n}{M}\PY{o}{+}\PY{n}{N}\PY{o}{+}\PY{l+m+mi}{1}\PY{p}{)} \PY{c+c1}{\PYZsh{} [0 for i in range(10)]    \PYZsh{} List of M+N+1 zeros}
        \PY{k}{for} \PY{n}{i} \PY{o+ow}{in} \PY{n+nb}{range}\PY{p}{(}\PY{l+m+mi}{0}\PY{p}{,} \PY{n}{M}\PY{o}{+}\PY{l+m+mi}{1}\PY{p}{)}\PY{p}{:} 
            \PY{k}{for} \PY{n}{j} \PY{o+ow}{in} \PY{n+nb}{range}\PY{p}{(}\PY{l+m+mi}{0}\PY{p}{,} \PY{n}{N}\PY{o}{+}\PY{l+m+mi}{1}\PY{p}{)}\PY{p}{:}
                \PY{n}{coeff}\PY{p}{[}\PY{n}{i}\PY{o}{+}\PY{n}{j}\PY{p}{]} \PY{o}{+}\PY{o}{=} \PY{n+nb+bp}{self}\PY{o}{.}\PY{n}{coeff}\PY{p}{[}\PY{n}{i}\PY{p}{]} \PY{o}{*} \PY{n}{other}\PY{o}{.}\PY{n}{coeff}\PY{p}{[}\PY{n}{j}\PY{p}{]}
        
        \PY{k}{return} \PY{n}{Polynomial}\PY{p}{(}\PY{n}{coeff}\PY{p}{)}

    \PY{k}{def}\PY{+w}{ }\PY{n+nf}{differentiate}\PY{p}{(}\PY{n+nb+bp}{self}\PY{p}{)}\PY{p}{:}
        \PY{k}{for} \PY{n}{i} \PY{o+ow}{in} \PY{n+nb}{range}\PY{p}{(}\PY{l+m+mi}{1}\PY{p}{,} \PY{n+nb}{len}\PY{p}{(}\PY{n+nb+bp}{self}\PY{o}{.}\PY{n}{coeff}\PY{p}{)}\PY{p}{)}\PY{p}{:}
            \PY{n+nb+bp}{self}\PY{o}{.}\PY{n}{coeff}\PY{p}{[}\PY{n}{i}\PY{o}{\PYZhy{}}\PY{l+m+mi}{1}\PY{p}{]} \PY{o}{=} \PY{n}{i} \PY{o}{*} \PY{n+nb+bp}{self}\PY{o}{.}\PY{n}{coeff}\PY{p}{[}\PY{n}{i}\PY{p}{]}
        \PY{k}{del} \PY{n+nb+bp}{self}\PY{o}{.}\PY{n}{coeff}\PY{p}{[}\PY{o}{\PYZhy{}}\PY{l+m+mi}{1}\PY{p}{]}

    \PY{k}{def}\PY{+w}{ }\PY{n+nf}{derivative}\PY{p}{(}\PY{n+nb+bp}{self}\PY{p}{)}\PY{p}{:}
        \PY{n}{dpdx} \PY{o}{=} \PY{n}{Polynomial}\PY{p}{(}\PY{n+nb+bp}{self}\PY{o}{.}\PY{n}{coeff}\PY{p}{[}\PY{p}{:}\PY{p}{]}\PY{p}{)}
        \PY{n}{dpdx}\PY{o}{.}\PY{n}{differentiate}\PY{p}{(}\PY{p}{)}
        \PY{k}{return} \PY{n}{dpdx}

    \PY{k}{def}\PY{+w}{ }\PY{n+nf+fm}{\PYZus{}\PYZus{}str\PYZus{}\PYZus{}}\PY{p}{(}\PY{n+nb+bp}{self}\PY{p}{)}\PY{p}{:}
        \PY{n}{s} \PY{o}{=} \PY{l+s+s1}{\PYZsq{}}\PY{l+s+s1}{\PYZsq{}}
        \PY{k}{for} \PY{n}{i} \PY{o+ow}{in} \PY{n+nb}{range}\PY{p}{(}\PY{l+m+mi}{0}\PY{p}{,} \PY{n+nb}{len}\PY{p}{(}\PY{n+nb+bp}{self}\PY{o}{.}\PY{n}{coeff}\PY{p}{)}\PY{p}{)}\PY{p}{:}
            \PY{k}{if} \PY{n+nb+bp}{self}\PY{o}{.}\PY{n}{coeff}\PY{p}{[}\PY{n}{i}\PY{p}{]} \PY{o}{!=} \PY{l+m+mi}{0}\PY{p}{:}
                \PY{n}{s} \PY{o}{+}\PY{o}{=} \PY{l+s+sa}{f}\PY{l+s+s1}{\PYZsq{}}\PY{l+s+s1}{ + }\PY{l+s+si}{\PYZob{}}\PY{n+nb+bp}{self}\PY{o}{.}\PY{n}{coeff}\PY{p}{[}\PY{n}{i}\PY{p}{]}\PY{l+s+si}{:}\PY{l+s+s1}{g}\PY{l+s+si}{\PYZcb{}}\PY{l+s+s1}{*x\PYZca{}}\PY{l+s+si}{\PYZob{}}\PY{n}{i}\PY{l+s+si}{:}\PY{l+s+s1}{g}\PY{l+s+si}{\PYZcb{}}\PY{l+s+s1}{\PYZsq{}}
        \PY{c+c1}{\PYZsh{} fix layout}
        \PY{n}{s} \PY{o}{=} \PY{n}{s}\PY{o}{.}\PY{n}{replace}\PY{p}{(}\PY{l+s+s1}{\PYZsq{}}\PY{l+s+s1}{+ \PYZhy{} }\PY{l+s+s1}{\PYZsq{}}\PY{p}{,} \PY{l+s+s1}{\PYZsq{}}\PY{l+s+s1}{\PYZhy{} }\PY{l+s+s1}{\PYZsq{}}\PY{p}{)}
        \PY{n}{s} \PY{o}{=} \PY{n}{s}\PY{o}{.}\PY{n}{replace}\PY{p}{(}\PY{l+s+s1}{\PYZsq{}}\PY{l+s+s1}{ 1*}\PY{l+s+s1}{\PYZsq{}}\PY{p}{,} \PY{l+s+s1}{\PYZsq{}}\PY{l+s+s1}{ }\PY{l+s+s1}{\PYZsq{}}\PY{p}{)}
        \PY{n}{s} \PY{o}{=} \PY{n}{s}\PY{o}{.}\PY{n}{replace}\PY{p}{(}\PY{l+s+s1}{\PYZsq{}}\PY{l+s+s1}{x\PYZca{}0}\PY{l+s+s1}{\PYZsq{}}\PY{p}{,}\PY{l+s+s1}{\PYZsq{}}\PY{l+s+s1}{1}\PY{l+s+s1}{\PYZsq{}}\PY{p}{)}
        \PY{n}{s} \PY{o}{=} \PY{n}{s}\PY{o}{.}\PY{n}{replace}\PY{p}{(}\PY{l+s+s1}{\PYZsq{}}\PY{l+s+s1}{*1}\PY{l+s+s1}{\PYZsq{}}\PY{p}{,} \PY{l+s+s1}{\PYZsq{}}\PY{l+s+s1}{\PYZsq{}}\PY{p}{)}
        \PY{n}{s} \PY{o}{=} \PY{n}{s}\PY{o}{.}\PY{n}{replace}\PY{p}{(}\PY{l+s+s1}{\PYZsq{}}\PY{l+s+s1}{x\PYZca{}1}\PY{l+s+s1}{\PYZsq{}}\PY{p}{,} \PY{l+s+s1}{\PYZsq{}}\PY{l+s+s1}{x}\PY{l+s+s1}{\PYZsq{}}\PY{p}{)}
        \PY{k}{if} \PY{n}{s}\PY{p}{[}\PY{l+m+mi}{0}\PY{p}{:}\PY{l+m+mi}{3}\PY{p}{]} \PY{o}{==} \PY{l+s+s1}{\PYZsq{}}\PY{l+s+s1}{ + }\PY{l+s+s1}{\PYZsq{}}\PY{p}{:}
            \PY{n}{s} \PY{o}{=} \PY{n}{s}\PY{p}{[}\PY{l+m+mi}{3}\PY{p}{:}\PY{p}{]}
        \PY{k}{if} \PY{n}{s}\PY{p}{[}\PY{l+m+mi}{0}\PY{p}{:}\PY{l+m+mi}{3}\PY{p}{]} \PY{o}{==} \PY{l+s+s1}{\PYZsq{}}\PY{l+s+s1}{ \PYZhy{} }\PY{l+s+s1}{\PYZsq{}}\PY{p}{:}
            \PY{n}{s} \PY{o}{=} \PY{l+s+s1}{\PYZsq{}}\PY{l+s+s1}{\PYZhy{}}\PY{l+s+s1}{\PYZsq{}} \PY{o}{+} \PY{n}{s}\PY{p}{[}\PY{l+m+mi}{3}\PY{p}{:}\PY{p}{]}
        \PY{k}{return} \PY{n}{s}
\end{Verbatim}
\end{tcolorbox}

\[ p_1(x) = x^3 + 2x^2 - 3\]

\begin{tcolorbox}[breakable, size=fbox, boxrule=1pt, pad at break*=1mm,colback=cellbackground, colframe=cellborder]
\prompt{In}{incolor}{5.57}{\boxspacing}
\begin{Verbatim}[commandchars=\\\{\}]
\PY{n}{p1} \PY{o}{=} \PY{n}{Polynomial}\PY{p}{(}\PY{p}{[}\PY{o}{\PYZhy{}}\PY{l+m+mi}{3}\PY{p}{,} \PY{l+m+mi}{0}\PY{p}{,} \PY{l+m+mi}{2}\PY{p}{,} \PY{l+m+mi}{1}\PY{p}{]}\PY{p}{)}     \PY{c+c1}{\PYZsh{} x³ + 2x² \PYZhy{}3}
\PY{n+nb}{print}\PY{p}{(}\PY{n}{p1}\PY{p}{)}
\end{Verbatim}
\end{tcolorbox}

\begin{Verbatim}[commandchars=\\\{\}]
-3 + 2*x\^{}2 + x\^{}3
\end{Verbatim}

\[ \dfrac{dp_1}{dx} = 3x^2 + 4x \]

\begin{tcolorbox}[breakable, size=fbox, boxrule=1pt, pad at break*=1mm,colback=cellbackground, colframe=cellborder]
\prompt{In}{incolor}{5.58}{\boxspacing}
\begin{Verbatim}[commandchars=\\\{\}]
\PY{n}{p2} \PY{o}{=} \PY{n}{p1}\PY{o}{.}\PY{n}{derivative}\PY{p}{(}\PY{p}{)}
\PY{n+nb}{print}\PY{p}{(}\PY{n}{p2}\PY{p}{)}
\end{Verbatim}
\end{tcolorbox}

\begin{Verbatim}[commandchars=\\\{\}]
4*x + 3*x\^{}2
\end{Verbatim}
\end{example}

    % Add a bibliography block to the postdoc
    
    
    
