\chapter{Git}
\begin{figure}
\centering
\includegraphics{img/git-black.png}
\caption{Git}
\end{figure}

\href{https://git-scm.com/}{Git} es un sistema de control de versiones
de software libre diseñado para manejar desde proyectos pequeños hasta
muy grandes con rapidez y eficiencia.

Git es \href{https://git-scm.com/documentation}{facil de aprender} y es
liviano con rápido desempeño.
\href{https://git-scm.com/about/small-and-fast}{Supera a software
similares} como Subversion, CVS, Perforce y ClearCase gracias a
características como manejo de ramas locales, areas de "staging" y
múltiples flujos de trabajo.

Git realiza un control de versiones del código, esto quiere decir que
almacena el código escrito en todas sus veriones haciendo un registro de
los cambios realizados, cuando y quien los hizo. Es posible además
volver a un estado anterior del código. Git realiza un registro de los
cambios en el código y almacena un \texttt{snapshot} del código pudiendo
regresar a una versión previa con facilidad.

Aquí puede descargar la versión de Git que corresponda para su sistema:

\begin{itemize}
\item
  \href{https://git-scm.com/download/win}{Windows}
\item
  \href{https://git-scm.com/download/mac}{macOS}
\end{itemize}

\section{Mini Tutorial Git}

\subsection{Estados de Git}

Git tiene tres estados para el código:

\begin{enumerate}
\item
  \textbf{Working Directory}: El directorio de trabajo, lugar donde se
  encuentra el código que estamos escribiendo.
\item
  \textbf{Staging Area}: Archivos de código listos para ser llevados al
  repositorio.
\item
  \textbf{Respository}: Archivos dentro del repositorio.
\end{enumerate}

\subsection{Configuración inicial de Git}

\href{https://git-scm.com/}{Git} está disponible para instalar en
Windows, Linux y macOS. Descarge e instale la versión que corresponda
para su sistema operativo.

Una vez instalado, puede acceder a Git a través de la terminal en Linux
y macOS o bien a través de GitBash en Windows.

Recién instalado Git, es necesario configurar el nombre y correo del
usuario. El siguiente comando asignará el nombre de usuario, este nombre
es el que quedará registrado en cada commit que se haga en el
repositorio.

\begin{Shaded}
\begin{Highlighting}[]
    \ExtensionTok{$}\NormalTok{ git config }\AttributeTok{{-}{-}global}\NormalTok{ user.name }\StringTok{"Juan Perez"}
\end{Highlighting}
\end{Shaded}

Si ejecuta el comando sin argumentos, mostrará el nombre que se
encuentra configurado actualmente:

\begin{Shaded}
\begin{Highlighting}[]
    \ExtensionTok{$}\NormalTok{ git config }\AttributeTok{{-}{-}global}\NormalTok{ user.name}
\end{Highlighting}
\end{Shaded}

Para registrar el correo del usuario el proceso es similar, el comando
es el siguiente:

\begin{Shaded}
\begin{Highlighting}[]
    \ExtensionTok{$}\NormalTok{ git config }\AttributeTok{{-}{-}global}\NormalTok{ user.email }\StringTok{"juan.perez@correo.com"}
\end{Highlighting}
\end{Shaded}

Si ejecuta el comando sin argumentos, mostrará el correo que se
encuentra configurado actualmente:

\begin{Shaded}
\begin{Highlighting}[]
    \ExtensionTok{$}\NormalTok{ git config }\AttributeTok{{-}{-}global}\NormalTok{ user.email}
\end{Highlighting}
\end{Shaded}

La terminal/WindowsBash puede mostrar los resultados de Git en colores,
esto facilita su lectura. Para habilitar esta característica basta con
ejecutar el comando:

\begin{Shaded}
\begin{Highlighting}[]
    \ExtensionTok{$}\NormalTok{ git config }\AttributeTok{{-}{-}global}\NormalTok{ color.ui true}
\end{Highlighting}
\end{Shaded}

Para ver la configuración actual de Git, ejecute el siguiente comando:

\begin{Shaded}
\begin{Highlighting}[]
    \ExtensionTok{$}\NormalTok{ git config }\AttributeTok{{-}{-}global} \AttributeTok{{-}{-}list}
\end{Highlighting}
\end{Shaded}

\begin{Shaded}
\begin{Highlighting}[]
    \ExtensionTok{$}\NormalTok{ cat .gitconfig}
\end{Highlighting}
\end{Shaded}

\subsection{Comenzando con Git}

El comando \texttt{git\ help} muestra información del manual de git para
algún comando específico.

\begin{Shaded}
\begin{Highlighting}[]
    \ExtensionTok{$}\NormalTok{ git help comando}
\end{Highlighting}
\end{Shaded}

El comando \texttt{git\ init} inicializa el proyecto. Indica a Git que
este es el "Working Directory", que este directorio tiene el código que
habrá de guardarse en el repositorio. Debe usar este comando cada que
comienza un proyecto nuevo.

\begin{Shaded}
\begin{Highlighting}[]
    \ExtensionTok{$}\NormalTok{ git init}
\end{Highlighting}
\end{Shaded}

El comando \texttt{git\ stauts} muestra el estado actual del
repositorio.

\begin{Shaded}
\begin{Highlighting}[]
    \ExtensionTok{$}\NormalTok{ git status}
\end{Highlighting}
\end{Shaded}

\subsection{Añadir archivos al Staging
Area}\label{auxf1adir-archivos-al-staging-area}

Para añadir archivos al Staging Area se usa el comando git add. Puede
agregarse un archivo o varios a la vez.

El siguiente comando añade el archivo file.txt al staging:

\begin{Shaded}
\begin{Highlighting}[]
    \ExtensionTok{$}\NormalTok{ git add file.txt}
\end{Highlighting}
\end{Shaded}

El siguiente comando añade todos los archivos en el directorio actual:

\begin{Shaded}
\begin{Highlighting}[]
    \ExtensionTok{$}\NormalTok{ git add }\AttributeTok{{-}A}
\end{Highlighting}
\end{Shaded}

\subsection{Añadir archivos al Repository}

Para añadir archivos al Repository se utiliza el comando
\texttt{git\ commit}. Este comando permite agregar un mensaje, el cual
nos permite especificar el cambio que hemos realizado en el código. Este
mensaje es para nosotros mismos como desarrolladores, ya que en un
futuro que consultemos los cambios, podremos saber con precisión que
cambio fue realizado en esa etapa del desarrollo del código. Por esa
razón se recomienda que el mensaje sea claro y conciso.

\begin{Shaded}
\begin{Highlighting}[]
    \ExtensionTok{$}\NormalTok{ git commit }\AttributeTok{{-}m} \StringTok{"Mensaje claro y conciso que describe el cambio en el código"}
\end{Highlighting}
\end{Shaded}

Una vez que se haga algún cambio en el código o se agreguen archivos,
haga un \texttt{git\ status} y este indicará que el código en el
\emph{Working Directory} difiere del que se encientra en el
\emph{Repository}. Este será el momento para hacer un \texttt{git\ add}
para los archivos modificados y un \texttt{git\ commit} nuevamente.

Es posible utilizar el comando \texttt{git\ commit} sin más argumentos.
Esto llevará los cambios del Staging Area al Repository, pero dado que
no se especificó un mensaje, se abrirá el editor por defecto en el
sistema (\texttt{vi} en los sistemas *NIX) para escribir el mensaje
correspondiente para el \emph{commit}.

\subsection{Bitácora de Cambios}

Uno de las grandes ventajas de utiliar Git es que guarda un registro de
los cambios y un \texttt{snapshot} del código en el momento del
\texttt{commit}. Para ver el registro se utiliza el comando
\texttt{git\ log}.

\begin{Shaded}
\begin{Highlighting}[]
    \ExtensionTok{$}\NormalTok{ git log}
\end{Highlighting}
\end{Shaded}

\subsection{Ver estados anteriores del
código}\label{ver-estados-anteriores-del-cuxf3digo}

El comando \texttt{git\ checkout} permite ver una versión específica del
código para la ocurrencia de un \emph{commit} específico. Se dice que
este comando permite "\emph{viajar en el tiempo}" del código. Este
comando requiere de un identificador SHA (Secure Hash Algorithm) del
\emph{commit}, el cual podemos ver en la bitácora. Por ejemplo:

\begin{Shaded}
\begin{Highlighting}[]
    \ExtensionTok{$}\NormalTok{ git checkout [}\OperatorTok{\textless{}}\NormalTok{commit}\OperatorTok{\textgreater{}}\NormalTok{]}
\end{Highlighting}
\end{Shaded}

Este comando llevará el código al estado en el que se encontraba al
momento de haber hecho el coommit correspondiente al ID. De esta forma
podremos examinar el código en ese momento y llevar a cabo cualquier
acción que deseemos con el.

Si desearamos viajar nuevamente a otro estado anterior del código
podríamos hacerlo con \texttt{git\ chechout\ ID}, de acuerdo al ID
específico a donde quisieramos ir. Por otro lado, si quisieramos
regresar al último commit realizado (antes del primer
\texttt{git\ checkout}), basta con escribir:

\begin{Shaded}
\begin{Highlighting}[]
    \ExtensionTok{$}\NormalTok{ git checkout master}
\end{Highlighting}
\end{Shaded}

\subsection{Regresar a estados anteriores del
código.}\label{regresar-a-estados-anteriores-del-cuxf3digo.}

Una de las funcionalidades de Git es que permite ver y regresar a
estados anteriores del código. Esto es útil por si algún commit tuviera
un error peligroso o cambios no deseados. Utilice esta instrucción con
precaución.

Existen cinco tipos de \texttt{reset}: \texttt{soft}, \texttt{mixed},
\texttt{hard}, \texttt{merge} y \texttt{keep}. A continuación se
describne los más comunes:

\begin{description}
\item [soft] Mantiene los cambios de nuestros archivos intacto,
  simplemente es para que Git tenga presente que está en otro
  \emph{commit}.
\item [mixed] Mantiene nuestros archivos, pero limpia el index de
  git de las cambios realizadas.
\item [hard] Elimina todo los cambios que tenemos en nuestros
  archivos para dejarlo exactamente igual que en el repositorio.
\end{description}

\section{GitHub}

\begin{figure}
\centering
\includegraphics[width=2.08333in,height=\textheight]{img/github.png}
\caption{GitHub}
\end{figure}

\href{https://github.com/}{GitHub} es una plataforma de desarrollo
colaborativo para alojar proyectos utilizando el sistema de control de
versiones Git. Utiliza el framework Ruby on Rails por GitHub,
Inc.~(anteriormente conocida como Logical Awesome).

GitHub es una plataforma de desarrollo inspirada en su forma de trabajo.
Usted puede almacenar y revisar código, administrar proyectos y
construir software desde código abierto hasta empresarial junto a
millones de desarrolladores.

Desde enero de 2010, opera bajo el nombre de GitHub, Inc.~El código se
almacena de forma pública, aunque también se puede hacer de forma
privada, creando una cuenta de pago.

GitHub es un sitio que crea una comunidad de desarrolladores, se puede
decir que es la red social de los desarrolladores.

\section{MiniTutorial Git + GitHub}

\subsection{Clonar un repositorio}

Si deseamos descargar todos los archivos de código de un proyecto que se
encuentra en GitHub, podemos hacer una copia a través del comando:

\begin{Shaded}
\begin{Highlighting}[]
    \ExtensionTok{$}\NormalTok{ git clone URL}
\end{Highlighting}
\end{Shaded}

Este comando es útil si nos interesa obtener el código y no
necesariamente hacer contribuciones.

\subsection{Manipular repositorios remotos} 

Para vincular un proyecto de código local a un repositorio remoto en
GitHub se utiliza el comando \emph{git remote}.

\begin{Shaded}
\begin{Highlighting}[]
    \ExtensionTok{$}\NormalTok{ git remote add origin URL\_Repo}
\end{Highlighting}
\end{Shaded}

Este comando establecerá un vinculo entre el código del repositorio
local \emph{origin} con el repositorio remoto que se encuentra en GitHub
a través de \emph{URL\_Repo}.

El siguiente comando mostrará si existe un vinculo entre el repositorio
local y uno remoto:

\begin{Shaded}
\begin{Highlighting}[]
    \ExtensionTok{$}\NormalTok{ git remote }\AttributeTok{{-}v}
\end{Highlighting}
\end{Shaded}

El siguiente comando permite eliminar el vinculo que exista entre el
repositorio local y remoto:

\begin{Shaded}
\begin{Highlighting}[]
    \ExtensionTok{$}\NormalTok{ git remote remove origin}
\end{Highlighting}
\end{Shaded}

Se puede comprobar el efecto de estas operaciones con el comando
\texttt{git\ remote\ -v}.

El comando \texttt{git\ push} permite subir el código que se encuentra
en el repositorio local al repositorio remoto en GitHub.

\begin{Shaded}
\begin{Highlighting}[]
    \ExtensionTok{$}\NormalTok{ git push origin master}
\end{Highlighting}
\end{Shaded}

El comando solicitará el \emph{username} y \emph{password} de la cuenta
de GitHib donde se encuentra el repositorio remoto.

El comando \texttt{git\ push\ origin\ master} puede ejecutarse cada vez
que se hagan cambios en el código, de manera que los repositorios local
y remoto se encuentren en sincronía.

Tenga en cuenta que este comando sincroniza la rama \emph{master} del
repositorio local con el repositorio remoto en GitHub.

\section{Referencias}

\begin{itemize}
\item
  \href{https://git-scm.com/}{Git}
\item
  \href{https://github.com/}{GitHub}
\item
  \href{https://gitlab.com/}{GitLab}
\end{itemize}
