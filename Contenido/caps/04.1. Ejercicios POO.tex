\section{Ejercicios de POO}
\begin{exercise}{\rm
Cree una clase llamada Persona, use la función \emph{\_\_init\_\_()}
para asignar valores para nombre y edad:

\begin{Shaded}
\begin{Highlighting}[]
\KeywordTok{class}\NormalTok{ Person:}
  \KeywordTok{def} \FunctionTok{\_\_init\_\_}\NormalTok{(}\VariableTok{self}\NormalTok{, name, age):}
    \VariableTok{self}\NormalTok{.name }\OperatorTok{=}\NormalTok{ name}
    \VariableTok{self}\NormalTok{.age }\OperatorTok{=}\NormalTok{ age}

\NormalTok{p1 }\OperatorTok{=}\NormalTok{ Person(}\StringTok{"John"}\NormalTok{, }\DecValTok{36}\NormalTok{)}

\BuiltInTok{print}\NormalTok{(p1.name)}
\BuiltInTok{print}\NormalTok{(p1.age)}
\end{Highlighting}
\end{Shaded}

\begin{verbatim}
John
36
\end{verbatim}

\begin{Shaded}
\begin{Highlighting}[]
\KeywordTok{class}\NormalTok{ Person:}
  \KeywordTok{def} \FunctionTok{\_\_init\_\_}\NormalTok{(}\VariableTok{self}\NormalTok{, name, age):}
    \VariableTok{self}\NormalTok{.name }\OperatorTok{=}\NormalTok{ name}
    \VariableTok{self}\NormalTok{.age }\OperatorTok{=}\NormalTok{ age}

\NormalTok{p1 }\OperatorTok{=}\NormalTok{ Person(}\StringTok{"John"}\NormalTok{, }\DecValTok{36}\NormalTok{)}

\BuiltInTok{print}\NormalTok{(p1)}
\end{Highlighting}
\end{Shaded}

\begin{verbatim}
<__main__.Person object at 0x7d8759ac9240>
\end{verbatim}

\begin{Shaded}
\begin{Highlighting}[]
\KeywordTok{class}\NormalTok{ Person:}
  \KeywordTok{def} \FunctionTok{\_\_init\_\_}\NormalTok{(mysillyobject, name, age):}
\NormalTok{    mysillyobject.name }\OperatorTok{=}\NormalTok{ name}
\NormalTok{    mysillyobject.age }\OperatorTok{=}\NormalTok{ age}

  \KeywordTok{def}\NormalTok{ myfunc(abc):}
    \BuiltInTok{print}\NormalTok{(}\StringTok{"Hello my name is "} \OperatorTok{+}\NormalTok{ abc.name)}

\NormalTok{p1 }\OperatorTok{=}\NormalTok{ Person(}\StringTok{"John"}\NormalTok{, }\DecValTok{36}\NormalTok{)}
\NormalTok{p1.myfunc()}
\end{Highlighting}
\end{Shaded}

\begin{verbatim}
Hello my name is John
\end{verbatim}

\begin{Shaded}
\begin{Highlighting}[]
\KeywordTok{class}\NormalTok{ Person:}
  \KeywordTok{def} \FunctionTok{\_\_init\_\_}\NormalTok{(}\VariableTok{self}\NormalTok{, name, age):}
    \VariableTok{self}\NormalTok{.name }\OperatorTok{=}\NormalTok{ name}
    \VariableTok{self}\NormalTok{.age }\OperatorTok{=}\NormalTok{ age}

  \KeywordTok{def} \FunctionTok{\_\_str\_\_}\NormalTok{(}\VariableTok{self}\NormalTok{):}
    \ControlFlowTok{return} \SpecialStringTok{f"}\SpecialCharTok{\{}\VariableTok{self}\SpecialCharTok{.}\NormalTok{name}\SpecialCharTok{\}}\SpecialStringTok{ is }\SpecialCharTok{\{}\VariableTok{self}\SpecialCharTok{.}\NormalTok{age}\SpecialCharTok{\}}\SpecialStringTok{ years old."}

\NormalTok{p1 }\OperatorTok{=}\NormalTok{ Person(}\StringTok{"John"}\NormalTok{, }\DecValTok{36}\NormalTok{)}

\BuiltInTok{print}\NormalTok{(p1)}
\end{Highlighting}
\end{Shaded}

\begin{verbatim}
John is 36 years old.

\end{verbatim}
}\end{exercise}



\begin{exercise}{\rm 
Crear una clase que reciba la parte real y la parte imaginaria de un
número complejo. Debe devolver la representación del número.

\[ 2 \pm 3i\] \[ -4 \pm 5i \]

\begin{Shaded}
\begin{Highlighting}[]
\KeywordTok{class}\NormalTok{ complex\_number:}
    \KeywordTok{def} \FunctionTok{\_\_init\_\_}\NormalTok{(}\VariableTok{self}\NormalTok{, real, imag):}
        \VariableTok{self}\NormalTok{.real }\OperatorTok{=}\NormalTok{ real}
        \VariableTok{self}\NormalTok{.imag }\OperatorTok{=}\NormalTok{ imag}

    \KeywordTok{def} \FunctionTok{\_\_str\_\_}\NormalTok{(}\VariableTok{self}\NormalTok{):}
        \ControlFlowTok{return} \SpecialStringTok{f\textquotesingle{}}\SpecialCharTok{\{}\VariableTok{self}\SpecialCharTok{.}\NormalTok{real}\SpecialCharTok{\}}\SpecialStringTok{ ± }\SpecialCharTok{\{}\VariableTok{self}\SpecialCharTok{.}\NormalTok{imag}\SpecialCharTok{\}}\SpecialStringTok{i\textquotesingle{}}

\NormalTok{num1 }\OperatorTok{=}\NormalTok{ complex\_number(}\DecValTok{2}\NormalTok{,}\DecValTok{3}\NormalTok{)}
\BuiltInTok{print}\NormalTok{(num1)}
\NormalTok{num2 }\OperatorTok{=}\NormalTok{ complex\_number(}\OperatorTok{{-}}\DecValTok{4}\NormalTok{,}\DecValTok{5}\NormalTok{)}
\BuiltInTok{print}\NormalTok{(num2)}
\end{Highlighting}
\end{Shaded}

\begin{verbatim}
2 ± 3i
-4 ± 5i

\end{verbatim}
}\end{exercise}


\begin{exercise}{\rm 
Inserte una función que imprima un saludo y ejecútelo en el objeto p1:

\begin{Shaded}
\begin{Highlighting}[]
\KeywordTok{class}\NormalTok{ Person:}
  \KeywordTok{def} \FunctionTok{\_\_init\_\_}\NormalTok{(}\VariableTok{self}\NormalTok{, name, age):}
    \VariableTok{self}\NormalTok{.name }\OperatorTok{=}\NormalTok{ name}
    \VariableTok{self}\NormalTok{.age }\OperatorTok{=}\NormalTok{ age}

  \KeywordTok{def}\NormalTok{ myfunc(}\VariableTok{self}\NormalTok{):}
    \BuiltInTok{print}\NormalTok{(}\StringTok{"Hello my name is "} \OperatorTok{+} \VariableTok{self}\NormalTok{.name }\OperatorTok{+} \StringTok{" and I\textquotesingle{}m "} \OperatorTok{+} \BuiltInTok{str}\NormalTok{(}\VariableTok{self}\NormalTok{.age) }\OperatorTok{+} \StringTok{" years old."}\NormalTok{)}

\NormalTok{p1 }\OperatorTok{=}\NormalTok{ Person(}\StringTok{"John"}\NormalTok{, }\DecValTok{36}\NormalTok{)}
\NormalTok{p1.myfunc()}
\end{Highlighting}
\end{Shaded}

\begin{verbatim}
Hello my name is John and I'm 36 years old.

\end{verbatim}

}\end{exercise}



\begin{exercise}{\rm 

Registro de perros

\begin{Shaded}
\begin{Highlighting}[]
\KeywordTok{class}\NormalTok{ Perro:}
    \KeywordTok{def} \FunctionTok{\_\_init\_\_}\NormalTok{(}\VariableTok{self}\NormalTok{, name, raza, talla, peso):}
        \VariableTok{self}\NormalTok{.name }\OperatorTok{=}\NormalTok{ name}
        \VariableTok{self}\NormalTok{.raza }\OperatorTok{=}\NormalTok{ raza}
        \VariableTok{self}\NormalTok{.talla }\OperatorTok{=}\NormalTok{ talla}
        \VariableTok{self}\NormalTok{.peso }\OperatorTok{=}\NormalTok{ peso}

    \KeywordTok{def}\NormalTok{ obtener\_peso(}\VariableTok{self}\NormalTok{):}
        \ControlFlowTok{return} \VariableTok{self}\NormalTok{.peso}

\NormalTok{miPerro }\OperatorTok{=}\NormalTok{ Perro(}\StringTok{"Firulais"}\NormalTok{, }\StringTok{"mestizo"}\NormalTok{, }\StringTok{"Mediano"}\NormalTok{, }\DecValTok{15}\NormalTok{)}
\BuiltInTok{print}\NormalTok{(}\SpecialStringTok{f\textquotesingle{}Mi perro pesa }\SpecialCharTok{\{}\NormalTok{miPerro}\SpecialCharTok{.}\NormalTok{obtener\_peso()}\SpecialCharTok{\}}\SpecialStringTok{ kilos\textquotesingle{}}\NormalTok{)}
\end{Highlighting}
\end{Shaded}

\begin{verbatim}
Mi perro pesa 15 kilos

\end{verbatim}
}\end{exercise}




\begin{exercise}{\rm 

Registro de perros con promedio de peso

\begin{Shaded}
\begin{Highlighting}[]
\KeywordTok{class}\NormalTok{ Perro:}
    \CommentTok{\# Peso promedio}
\NormalTok{    peso }\OperatorTok{=} \DecValTok{20}

    \KeywordTok{def} \FunctionTok{\_\_init\_\_}\NormalTok{(}\VariableTok{self}\NormalTok{, name, raza, talla, peso):}
        \VariableTok{self}\NormalTok{.name }\OperatorTok{=}\NormalTok{ name}
        \VariableTok{self}\NormalTok{.raza }\OperatorTok{=}\NormalTok{ raza}
        \VariableTok{self}\NormalTok{.talla }\OperatorTok{=}\NormalTok{ talla}
        \VariableTok{self}\NormalTok{.peso }\OperatorTok{=}\NormalTok{ peso}

    \AttributeTok{@classmethod} \CommentTok{\# Método de clase: Acceso a los valores de clase en lugar de instancia}
    \KeywordTok{def}\NormalTok{ obtener\_peso(}\VariableTok{self}\NormalTok{):}
        \ControlFlowTok{return} \VariableTok{self}\NormalTok{.peso}

\NormalTok{miPerro }\OperatorTok{=}\NormalTok{ Perro(}\StringTok{"Firulais"}\NormalTok{, }\StringTok{"mestizo"}\NormalTok{, }\StringTok{"Mediano"}\NormalTok{, }\DecValTok{12}\NormalTok{)}
\BuiltInTok{print}\NormalTok{(}\SpecialStringTok{f\textquotesingle{}Mi perro pesa }\SpecialCharTok{\{}\NormalTok{miPerro}\SpecialCharTok{.}\NormalTok{peso}\SpecialCharTok{\}}\SpecialStringTok{ kilos\textquotesingle{}}\NormalTok{)}
\BuiltInTok{print}\NormalTok{(}\SpecialStringTok{f\textquotesingle{}El peso promedio de los perros es }\SpecialCharTok{\{}\NormalTok{miPerro}\SpecialCharTok{.}\NormalTok{obtener\_peso()}\SpecialCharTok{\}}\SpecialStringTok{ kilos\textquotesingle{}}\NormalTok{)}
\end{Highlighting}
\end{Shaded}

\begin{verbatim}
Mi perro pesa 12 kilos
El peso promedio de los perros es 20 kilos

\end{verbatim}
}\end{exercise}


\texttt{@classmethod} convierte un método para que trabaje con la clase
en sí (y sus atributos de clase) y no con instancias específicas.

\begin{exercise}{\rm 

Calculadora

\begin{Shaded}
\begin{Highlighting}[]
\KeywordTok{class}\NormalTok{ calcu:}

    \AttributeTok{@staticmethod}
    \KeywordTok{def}\NormalTok{ sumar(num1, num2):}
        \ControlFlowTok{return}\NormalTok{ num1 }\OperatorTok{+}\NormalTok{ num2}

    \AttributeTok{@staticmethod}
    \KeywordTok{def}\NormalTok{ resta(num1, num2):}
        \ControlFlowTok{return}\NormalTok{ num1 }\OperatorTok{{-}}\NormalTok{ num2}

    \AttributeTok{@staticmethod}
    \KeywordTok{def}\NormalTok{ multiplicacion(num1, num2):}
        \ControlFlowTok{return}\NormalTok{ num1 }\OperatorTok{*}\NormalTok{ num2}

    \AttributeTok{@staticmethod}
    \KeywordTok{def}\NormalTok{ division(num1, num2):}
        \ControlFlowTok{return}\NormalTok{ num1 }\OperatorTok{/}\NormalTok{ num2}

\BuiltInTok{print}\NormalTok{(calcu.sumar(}\DecValTok{4}\NormalTok{,}\DecValTok{5}\NormalTok{))}
\BuiltInTok{print}\NormalTok{(calcu.resta(}\DecValTok{8}\NormalTok{,}\DecValTok{4}\NormalTok{))}
\BuiltInTok{print}\NormalTok{(calcu.multiplicacion(}\DecValTok{2}\NormalTok{,}\DecValTok{9}\NormalTok{))}
\BuiltInTok{print}\NormalTok{(calcu.division(}\DecValTok{6}\NormalTok{,}\DecValTok{3}\NormalTok{))}
\end{Highlighting}
\end{Shaded}

\begin{verbatim}
9
4
18
2.0
\end{verbatim}

\begin{Shaded}
\begin{Highlighting}[]
\NormalTok{x }\OperatorTok{=}\NormalTok{ calcu.sumar(}\DecValTok{4}\NormalTok{,}\DecValTok{5}\NormalTok{)}
\BuiltInTok{print}\NormalTok{(x)}
\end{Highlighting}
\end{Shaded}

\begin{verbatim}
9

\end{verbatim}
}\end{exercise}


\begin{exercise}{\rm 

Realizar una clase que reciba un par de argumentos numéricos y realice
las operaciones aritméticas básicas

\begin{Shaded}
\begin{Highlighting}[]
\KeywordTok{class}\NormalTok{ calculadora:}
    \KeywordTok{def} \FunctionTok{\_\_init\_\_}\NormalTok{(}\VariableTok{self}\NormalTok{, num1, num2):}
        \VariableTok{self}\NormalTok{.num1 }\OperatorTok{=}\NormalTok{ num1}
        \VariableTok{self}\NormalTok{.num2 }\OperatorTok{=}\NormalTok{ num2}

    \KeywordTok{def}\NormalTok{ suma(}\VariableTok{self}\NormalTok{):}
        \ControlFlowTok{return} \VariableTok{self}\NormalTok{.num1 }\OperatorTok{+} \VariableTok{self}\NormalTok{.num2}

    \KeywordTok{def}\NormalTok{ resta(}\VariableTok{self}\NormalTok{):}
        \ControlFlowTok{return} \VariableTok{self}\NormalTok{.num1 }\OperatorTok{{-}} \VariableTok{self}\NormalTok{.num2}

    \KeywordTok{def}\NormalTok{ multiplicacion(}\VariableTok{self}\NormalTok{):}
        \ControlFlowTok{return} \VariableTok{self}\NormalTok{.num1 }\OperatorTok{*} \VariableTok{self}\NormalTok{.num2}

    \KeywordTok{def}\NormalTok{ division(}\VariableTok{self}\NormalTok{):}
        \ControlFlowTok{return} \VariableTok{self}\NormalTok{.num1 }\OperatorTok{/} \VariableTok{self}\NormalTok{.num2}

\NormalTok{operacion }\OperatorTok{=}\NormalTok{ calculadora(}\DecValTok{5}\NormalTok{,}\DecValTok{9}\NormalTok{)}
\BuiltInTok{print}\NormalTok{(operacion.num1)}
\BuiltInTok{print}\NormalTok{(operacion.num2)}
\BuiltInTok{print}\NormalTok{(operacion.suma())}
\BuiltInTok{print}\NormalTok{(operacion.resta())}
\BuiltInTok{print}\NormalTok{(operacion.multiplicacion())}
\BuiltInTok{print}\NormalTok{(operacion.division())}
\end{Highlighting}
\end{Shaded}

\begin{verbatim}
5
9
14
-4
45
0.5555555555555556

\end{verbatim}
}\end{exercise}


\begin{exercise}{\rm 

Realice un clase \emph{Circulo} que reciba las coordenadas \((x,y)\)
donde se ubica el centro y el radio \(r\). La clase debe tener métodos
que determinen el perímetro, área, cuadrante donde se ubica el cirulo en
el plano cartesiano.

\begin{Shaded}
\begin{Highlighting}[]
\ImportTok{import}\NormalTok{ math}

\KeywordTok{class}\NormalTok{ Circulo:}
    \KeywordTok{def} \FunctionTok{\_\_init\_\_}\NormalTok{(}\VariableTok{self}\NormalTok{, x, y, r):}
        \VariableTok{self}\NormalTok{.x }\OperatorTok{=}\NormalTok{ x}
        \VariableTok{self}\NormalTok{.y }\OperatorTok{=}\NormalTok{ y}
        \VariableTok{self}\NormalTok{.r }\OperatorTok{=}\NormalTok{ r}

    \KeywordTok{def}\NormalTok{ perimetro(}\VariableTok{self}\NormalTok{):}
        \ControlFlowTok{return} \DecValTok{2} \OperatorTok{*}\NormalTok{ math.pi }\OperatorTok{*} \VariableTok{self}\NormalTok{.r}

    \KeywordTok{def}\NormalTok{ area(}\VariableTok{self}\NormalTok{):}
        \ControlFlowTok{return}\NormalTok{ math.pi }\OperatorTok{*} \VariableTok{self}\NormalTok{.r}\OperatorTok{**}\DecValTok{2}

    \KeywordTok{def}\NormalTok{ cuadrante(}\VariableTok{self}\NormalTok{):}
        \ControlFlowTok{if} \VariableTok{self}\NormalTok{.x }\OperatorTok{\textgreater{}} \DecValTok{0} \KeywordTok{and} \VariableTok{self}\NormalTok{.y }\OperatorTok{\textgreater{}} \DecValTok{0}\NormalTok{:}
            \ControlFlowTok{return} \StringTok{"I"}
        \ControlFlowTok{elif} \VariableTok{self}\NormalTok{.x }\OperatorTok{\textless{}} \DecValTok{0} \KeywordTok{and} \VariableTok{self}\NormalTok{.y }\OperatorTok{\textgreater{}} \DecValTok{0}\NormalTok{:}
            \ControlFlowTok{return} \StringTok{"II"}
        \ControlFlowTok{elif} \VariableTok{self}\NormalTok{.x }\OperatorTok{\textgreater{}} \DecValTok{0} \KeywordTok{and} \VariableTok{self}\NormalTok{.y }\OperatorTok{\textless{}} \DecValTok{0}\NormalTok{:}
            \ControlFlowTok{return} \StringTok{"IV"}
        \ControlFlowTok{else}\NormalTok{:}
            \ControlFlowTok{return} \StringTok{"III"}

\NormalTok{c1 }\OperatorTok{=}\NormalTok{ Circulo(}\OperatorTok{{-}}\DecValTok{6}\NormalTok{,}\OperatorTok{{-}}\DecValTok{7}\NormalTok{,}\DecValTok{2}\NormalTok{)}
\BuiltInTok{print}\NormalTok{(c1.perimetro())}
\BuiltInTok{print}\NormalTok{(c1.area())}
\BuiltInTok{print}\NormalTok{(c1.cuadrante())}
\end{Highlighting}
\end{Shaded}

\begin{verbatim}
12.566370614359172
12.566370614359172
III

\end{verbatim}
}\end{exercise}


\begin{exercise}{\rm 

Realice una clase que reciba la parte real y la parte imaginaria de un
número complejo. Debe tener métodos que determine el módulo del número
complejo, la suma y resta de dos números complejos.

\[ |2 +- 3i| = \sqrt{2^2 + 3^2} = \sqrt{13} \]
\[ (2+-3i) + (4+-5i) = 6 +- 8i \] \[ (2+-3i) - (4+-5i) = -2 +- 2i \]

\begin{Shaded}
\begin{Highlighting}[]
\ImportTok{import}\NormalTok{ math}

\KeywordTok{class}\NormalTok{ ComplexNumber:}
    \KeywordTok{def} \FunctionTok{\_\_init\_\_}\NormalTok{(}\VariableTok{self}\NormalTok{, r, i):}
        \VariableTok{self}\NormalTok{.r }\OperatorTok{=}\NormalTok{ r}
        \VariableTok{self}\NormalTok{.i }\OperatorTok{=}\NormalTok{ i}

    \KeywordTok{def}\NormalTok{ modulo(}\VariableTok{self}\NormalTok{):}
        \ControlFlowTok{return}\NormalTok{ math.sqrt(}\VariableTok{self}\NormalTok{.r}\OperatorTok{**}\DecValTok{2} \OperatorTok{+} \VariableTok{self}\NormalTok{.i}\OperatorTok{**}\DecValTok{2}\NormalTok{)}

    \KeywordTok{def} \FunctionTok{\_\_str\_\_}\NormalTok{(}\VariableTok{self}\NormalTok{):}
        \ControlFlowTok{return} \SpecialStringTok{f\textquotesingle{}}\SpecialCharTok{\{}\VariableTok{self}\SpecialCharTok{.}\NormalTok{r}\SpecialCharTok{\}}\SpecialStringTok{ ± }\SpecialCharTok{\{}\VariableTok{self}\SpecialCharTok{.}\NormalTok{i}\SpecialCharTok{\}}\SpecialStringTok{i\textquotesingle{}}

    \CommentTok{\# Métodos Mágicos (Dunder {-}\textgreater{} double under (score))}
    \KeywordTok{def} \FunctionTok{\_\_add\_\_}\NormalTok{(}\VariableTok{self}\NormalTok{, other):}
        \ControlFlowTok{return}\NormalTok{ ComplexNumber(}\VariableTok{self}\NormalTok{.r }\OperatorTok{+}\NormalTok{ other.r, }\VariableTok{self}\NormalTok{.i }\OperatorTok{+}\NormalTok{ other.i)}

    \KeywordTok{def} \FunctionTok{\_\_sub\_\_}\NormalTok{(}\VariableTok{self}\NormalTok{, other):}
        \ControlFlowTok{return}\NormalTok{ ComplexNumber(}\VariableTok{self}\NormalTok{.r }\OperatorTok{{-}}\NormalTok{ other.r, }\VariableTok{self}\NormalTok{.i }\OperatorTok{{-}}\NormalTok{ other.i)}

\CommentTok{\# Ejemplo}
\NormalTok{c1 }\OperatorTok{=}\NormalTok{ ComplexNumber(}\DecValTok{2}\NormalTok{,}\DecValTok{3}\NormalTok{)}
\BuiltInTok{print}\NormalTok{(c1)}
\BuiltInTok{print}\NormalTok{(c1.modulo())}

\NormalTok{c2 }\OperatorTok{=}\NormalTok{ ComplexNumber(}\DecValTok{4}\NormalTok{,}\DecValTok{5}\NormalTok{)}
\BuiltInTok{print}\NormalTok{(c2)}
\BuiltInTok{print}\NormalTok{(c2.modulo())}

\NormalTok{c3r }\OperatorTok{=}\NormalTok{ c1.r }\OperatorTok{+}\NormalTok{ c2.r}
\NormalTok{c3i }\OperatorTok{=}\NormalTok{ c1.i }\OperatorTok{+}\NormalTok{ c2.i}
\NormalTok{c3 }\OperatorTok{=}\NormalTok{ ComplexNumber(c3r, c3i)}
\BuiltInTok{print}\NormalTok{(c3)}

\NormalTok{c4 }\OperatorTok{=}\NormalTok{ c1 }\OperatorTok{+}\NormalTok{ c2}
\BuiltInTok{print}\NormalTok{(c4)}

\NormalTok{c5 }\OperatorTok{=}\NormalTok{ c1 }\OperatorTok{{-}}\NormalTok{ c2}
\BuiltInTok{print}\NormalTok{(c5)}
\end{Highlighting}
\end{Shaded}

\begin{verbatim}
2 ± 3i
3.605551275463989
4 ± 5i
6.4031242374328485
6 ± 8i
6 ± 8i
-2 ± -2i

\end{verbatim}
}\end{exercise}


En Python, los métodos \emph{\_\_add\_\_} y \emph{\_\_sub\_\_} son
métodos especiales que permiten definir el comportamiento de los
operadores de suma (+) y resta (-) para los objetos de una clase
personalizada. Estos métodos se denominan "\textbf{métodos mágicos}" o
"\textbf{métodos dunder}" (double underscore, por sus siglas en inglés).

\begin{exercise}{\rm \textbf{Clase Persona}.

Crea una clase \emph{Persona} con atributos nombre, edad y métodos para
mostrar esta información y calcular si la persona es mayor de edad.

\begin{Shaded}
\begin{Highlighting}[]
\KeywordTok{class}\NormalTok{ Persona:}
    \KeywordTok{def} \FunctionTok{\_\_init\_\_}\NormalTok{(}\VariableTok{self}\NormalTok{, nombre, edad):}
        \VariableTok{self}\NormalTok{.nombre }\OperatorTok{=}\NormalTok{ nombre}
        \VariableTok{self}\NormalTok{.edad }\OperatorTok{=}\NormalTok{ edad}

    \KeywordTok{def}\NormalTok{ mostrar\_info(}\VariableTok{self}\NormalTok{):}
        \ControlFlowTok{return}\NormalTok{(}\SpecialStringTok{f\textquotesingle{}Nombre: }\SpecialCharTok{\{}\VariableTok{self}\SpecialCharTok{.}\NormalTok{nombre}\SpecialCharTok{\}}\SpecialStringTok{, Edad: }\SpecialCharTok{\{}\VariableTok{self}\SpecialCharTok{.}\NormalTok{edad}\SpecialCharTok{\}}\SpecialStringTok{\textquotesingle{}}\NormalTok{)}

    \KeywordTok{def}\NormalTok{ mayor\_de\_edad(}\VariableTok{self}\NormalTok{):}
        \ControlFlowTok{if} \VariableTok{self}\NormalTok{.edad }\OperatorTok{\textgreater{}=} \DecValTok{18}\NormalTok{:}
            \ControlFlowTok{return} \VariableTok{True}
        \ControlFlowTok{else}\NormalTok{:}
            \ControlFlowTok{return} \VariableTok{False}

\CommentTok{\# Ejemplo}
\NormalTok{persona1 }\OperatorTok{=}\NormalTok{ Persona(}\StringTok{"Juan"}\NormalTok{, }\DecValTok{23}\NormalTok{)}
\BuiltInTok{print}\NormalTok{(persona1.mostrar\_info())}
\BuiltInTok{print}\NormalTok{(persona1.mayor\_de\_edad())}
\end{Highlighting}
\end{Shaded}

\begin{verbatim}
Nombre: Juan, Edad: 23
True
\end{verbatim}
}\end{exercise}


\begin{exercise}{\rm \textbf{Clase Rectángulo}. 
Crea una clase \emph{Rectángulo} con atributos largo y ancho. Incluye
métodos para calcular el área y el perímetro del rectángulo.

\begin{Shaded}
\begin{Highlighting}[]
\KeywordTok{class}\NormalTok{ Rectangulo:}
    \KeywordTok{def} \FunctionTok{\_\_init\_\_}\NormalTok{(}\VariableTok{self}\NormalTok{, largo, ancho):}
        \VariableTok{self}\NormalTok{.largo }\OperatorTok{=}\NormalTok{ largo}
        \VariableTok{self}\NormalTok{.ancho }\OperatorTok{=}\NormalTok{ ancho}

    \KeywordTok{def}\NormalTok{ perimetro(}\VariableTok{self}\NormalTok{):}
        \ControlFlowTok{return} \DecValTok{2} \OperatorTok{*}\NormalTok{ (}\VariableTok{self}\NormalTok{.largo }\OperatorTok{+} \VariableTok{self}\NormalTok{.ancho)}

    \KeywordTok{def}\NormalTok{ area(}\VariableTok{self}\NormalTok{):}
        \ControlFlowTok{return} \VariableTok{self}\NormalTok{.largo }\OperatorTok{*} \VariableTok{self}\NormalTok{.ancho}


\CommentTok{\# Ejemplo}
\NormalTok{r1 }\OperatorTok{=}\NormalTok{ Rectangulo(}\DecValTok{9}\NormalTok{, }\DecValTok{5}\NormalTok{)}
\BuiltInTok{print}\NormalTok{(r1.perimetro())}
\BuiltInTok{print}\NormalTok{(r1.area())}
\end{Highlighting}
\end{Shaded}

\begin{verbatim}
28
45

\end{verbatim}
}\end{exercise}

\begin{exercise}{\rm \textbf{Clase Círculo}. 
Crea una clase \emph{Círculo} con un atributo radio. Añade métodos para
calcular el área y la circunferencia del círculo.

\begin{Shaded}
\begin{Highlighting}[]
\ImportTok{import}\NormalTok{ math }

\KeywordTok{class}\NormalTok{ Circulo:}
    \KeywordTok{def} \FunctionTok{\_\_init\_\_}\NormalTok{(}\VariableTok{self}\NormalTok{, r):}
        \VariableTok{self}\NormalTok{.r }\OperatorTok{=}\NormalTok{ r}

    \KeywordTok{def}\NormalTok{ area(}\VariableTok{self}\NormalTok{):}
        \ControlFlowTok{return}\NormalTok{ math.pi }\OperatorTok{*} \VariableTok{self}\NormalTok{.r}\OperatorTok{**}\DecValTok{2}

    \KeywordTok{def}\NormalTok{ circunferencia(}\VariableTok{self}\NormalTok{):}
        \ControlFlowTok{return} \DecValTok{2}\OperatorTok{*}\NormalTok{math.pi}\OperatorTok{*}\VariableTok{self}\NormalTok{.r}
    
\CommentTok{\# Ejemplo}
\NormalTok{c1 }\OperatorTok{=}\NormalTok{ Circulo(}\DecValTok{4}\NormalTok{)}
\BuiltInTok{print}\NormalTok{(}\SpecialStringTok{f\textquotesingle{}Area = }\SpecialCharTok{\{}\NormalTok{c1}\SpecialCharTok{.}\NormalTok{area()}\SpecialCharTok{\}}\SpecialStringTok{ u²\textquotesingle{}}\NormalTok{)}
\BuiltInTok{print}\NormalTok{(}\SpecialStringTok{f\textquotesingle{}Perimetro = }\SpecialCharTok{\{}\NormalTok{c1}\SpecialCharTok{.}\NormalTok{circunferencia()}\SpecialCharTok{\}}\SpecialStringTok{ u\textquotesingle{}}\NormalTok{)}
\end{Highlighting}
\end{Shaded}

\begin{verbatim}
Area = 50.26548245743669 u²
Perimetro = 25.132741228718345 u
\end{verbatim}
}\end{exercise}

\begin{exercise}{\rm \textbf{Clase CuentaBancaria.}
Crea una clase \emph{CuentaBancaria} con atributos titular y saldo.
Incluye métodos para depositar, retirar y mostrar el saldo.

\begin{Shaded}
\begin{Highlighting}[]
\KeywordTok{class}\NormalTok{ CuentaBancaria:}
    \KeywordTok{def} \FunctionTok{\_\_init\_\_}\NormalTok{(}\VariableTok{self}\NormalTok{, titular, saldo}\OperatorTok{=}\DecValTok{0}\NormalTok{):}
        \VariableTok{self}\NormalTok{.titular }\OperatorTok{=}\NormalTok{ titular}
        \VariableTok{self}\NormalTok{.saldo }\OperatorTok{=}\NormalTok{ saldo}

    \KeywordTok{def}\NormalTok{ depositar(}\VariableTok{self}\NormalTok{, monto):}
        \VariableTok{self}\NormalTok{.saldo }\OperatorTok{+=}\NormalTok{ monto}

    \KeywordTok{def}\NormalTok{ retirar(}\VariableTok{self}\NormalTok{, monto):}
        \ControlFlowTok{if}\NormalTok{ monto }\OperatorTok{\textless{}} \VariableTok{self}\NormalTok{.saldo:}
            \VariableTok{self}\NormalTok{.saldo }\OperatorTok{=} \VariableTok{self}\NormalTok{.saldo }\OperatorTok{{-}}\NormalTok{ monto }\CommentTok{\# self.saldo {-}= monto}
        \ControlFlowTok{else}\NormalTok{:}
            \BuiltInTok{print}\NormalTok{(}\StringTok{"Saldo insuficiente"}\NormalTok{)}

    \KeywordTok{def}\NormalTok{ mostrarSaldo(}\VariableTok{self}\NormalTok{):}
        \BuiltInTok{print}\NormalTok{(}\SpecialStringTok{f\textquotesingle{}Saldo de }\SpecialCharTok{\{}\VariableTok{self}\SpecialCharTok{.}\NormalTok{titular}\SpecialCharTok{\}}\SpecialStringTok{: $ }\SpecialCharTok{\{}\VariableTok{self}\SpecialCharTok{.}\NormalTok{saldo}\SpecialCharTok{\}}\SpecialStringTok{\textquotesingle{}}\NormalTok{)}

\CommentTok{\# Ejemplo}
\NormalTok{cliente1 }\OperatorTok{=}\NormalTok{ CuentaBancaria(}\StringTok{"Juan Perez"}\NormalTok{, }\DecValTok{4500}\NormalTok{)}
\NormalTok{cliente1.depositar(}\DecValTok{430}\NormalTok{)}
\NormalTok{cliente1.mostrarSaldo()}
\NormalTok{cliente1.retirar(}\DecValTok{10000}\NormalTok{)}
\end{Highlighting}
\end{Shaded}

\begin{verbatim}
Saldo de Juan Perez: $ 4930
Saldo insuficiente

\end{verbatim}
}\end{exercise}

\begin{exercise}{\rm 
\textbf{Clase Estudiante.} 
Crea una clase Estudiante que herede de Persona. Añade un atributo
promedio y un método para determinar si el estudiante está aprobado
(promedio \textgreater= 6).

\begin{Shaded}
\begin{Highlighting}[]
\KeywordTok{class}\NormalTok{ Persona:}
    \KeywordTok{def} \FunctionTok{\_\_init\_\_}\NormalTok{(}\VariableTok{self}\NormalTok{, nombre, edad):}
        \VariableTok{self}\NormalTok{.nombre }\OperatorTok{=}\NormalTok{ nombre}
        \VariableTok{self}\NormalTok{.edad }\OperatorTok{=}\NormalTok{ edad}

    \KeywordTok{def}\NormalTok{ mostrarInfo(}\VariableTok{self}\NormalTok{):}
        \BuiltInTok{print}\NormalTok{(}\SpecialStringTok{f\textquotesingle{}Nombre: }\SpecialCharTok{\{}\VariableTok{self}\SpecialCharTok{.}\NormalTok{nombre}\SpecialCharTok{\}}\SpecialStringTok{, Edad: }\SpecialCharTok{\{}\VariableTok{self}\SpecialCharTok{.}\NormalTok{edad}\SpecialCharTok{\}}\SpecialStringTok{\textquotesingle{}}\NormalTok{)}

\KeywordTok{class}\NormalTok{ Estudiante(Persona):}
    \KeywordTok{def} \FunctionTok{\_\_init\_\_}\NormalTok{(}\VariableTok{self}\NormalTok{, nombre, edad, promedio):}
        \BuiltInTok{super}\NormalTok{().}\FunctionTok{\_\_init\_\_}\NormalTok{(nombre, edad)}
        \VariableTok{self}\NormalTok{.promedio }\OperatorTok{=}\NormalTok{ promedio}

    \KeywordTok{def}\NormalTok{ aprobado(}\VariableTok{self}\NormalTok{):}
        \ControlFlowTok{if} \VariableTok{self}\NormalTok{.promedio }\OperatorTok{\textgreater{}=} \DecValTok{6}\NormalTok{:}
            \ControlFlowTok{return} \VariableTok{True}
        \ControlFlowTok{else}\NormalTok{:}
            \ControlFlowTok{return} \VariableTok{False}

\CommentTok{\# Ejemplo}
\NormalTok{estudiante1 }\OperatorTok{=}\NormalTok{ Estudiante(}\StringTok{"Maria"}\NormalTok{, }\DecValTok{20}\NormalTok{, }\FloatTok{7.5}\NormalTok{)}
\BuiltInTok{print}\NormalTok{(estudiante1.aprobado())}
\NormalTok{estudiante1.mostrarInfo()}
\end{Highlighting}
\end{Shaded}

\begin{verbatim}
True
Nombre: Maria, Edad: 20

\end{verbatim}
}\end{exercise}

\begin{exercise}{\rm 

\textbf{Clase Libro.} Crea una clase Libro con atributos título, autor y año. Incluye un
método para mostrar la información del libro y otro para determinar si
es un libro antiguo (año \textless{} 2000).

\begin{Shaded}
\begin{Highlighting}[]
\KeywordTok{class}\NormalTok{ Libro:}
    \KeywordTok{def} \FunctionTok{\_\_init\_\_}\NormalTok{(}\VariableTok{self}\NormalTok{, titulo, autor, anio):}
        \VariableTok{self}\NormalTok{.titulo }\OperatorTok{=}\NormalTok{ titulo}
        \VariableTok{self}\NormalTok{.autor }\OperatorTok{=}\NormalTok{ autor}
        \VariableTok{self}\NormalTok{.anio }\OperatorTok{=}\NormalTok{ anio}

    \KeywordTok{def}\NormalTok{ info(}\VariableTok{self}\NormalTok{):}
        \ControlFlowTok{return} \SpecialStringTok{f\textquotesingle{}}\SpecialCharTok{\{}\VariableTok{self}\SpecialCharTok{.}\NormalTok{autor}\SpecialCharTok{\}}\SpecialStringTok{; }\SpecialCharTok{\{}\VariableTok{self}\SpecialCharTok{.}\NormalTok{titulo}\SpecialCharTok{\}}\SpecialStringTok{; }\SpecialCharTok{\{}\VariableTok{self}\SpecialCharTok{.}\NormalTok{anio}\SpecialCharTok{\}}\SpecialStringTok{\textquotesingle{}}

    \KeywordTok{def}\NormalTok{ esAntiguo(}\VariableTok{self}\NormalTok{):}
        \ControlFlowTok{return} \VariableTok{True} \ControlFlowTok{if} \VariableTok{self}\NormalTok{.anio }\OperatorTok{\textless{}} \DecValTok{2000} \ControlFlowTok{else} \VariableTok{False}

\NormalTok{libro1 }\OperatorTok{=}\NormalTok{ Libro(}\StringTok{"Algebra"}\NormalTok{, }\StringTok{"Baldor"}\NormalTok{, }\DecValTok{1980}\NormalTok{)}
\BuiltInTok{print}\NormalTok{(libro1.info())}
\BuiltInTok{print}\NormalTok{(libro1.esAntiguo())}
\end{Highlighting}
\end{Shaded}

\begin{verbatim}
Baldor; Algebra; 1980
True

\end{verbatim}
}\end{exercise}

\begin{exercise}{\rm \textbf{Clase Vehículo}. 
Crea una clase Vehículo con atributos marca, modelo y año. Añade un
método para mostrar la información del vehículo y otro para determinar
si es un vehículo clásico (año \textless{} 1980).

\begin{Shaded}
\begin{Highlighting}[]

\end{Highlighting}
\end{Shaded}
}\end{exercise}

\begin{exercise}{\rm 
\textbf{Clase Empleado.} 
Crea una clase Empleado con atributos nombre, salario y
años\_de\_experiencia. Añade un método para calcular un aumento salarial
basado en los años de experiencia (5\%).

\begin{Shaded}
\begin{Highlighting}[]
\KeywordTok{class}\NormalTok{ Empleado:}
    \KeywordTok{def} \FunctionTok{\_\_init\_\_}\NormalTok{(}\VariableTok{self}\NormalTok{, nombre, salario, exp):}
        \VariableTok{self}\NormalTok{.nombre }\OperatorTok{=}\NormalTok{ nombre}
        \VariableTok{self}\NormalTok{.salario }\OperatorTok{=}\NormalTok{ salario}
        \VariableTok{self}\NormalTok{.exp }\OperatorTok{=}\NormalTok{ exp}

    \KeywordTok{def} \FunctionTok{\_\_str\_\_}\NormalTok{(}\VariableTok{self}\NormalTok{):}
        \ControlFlowTok{return} \SpecialStringTok{f\textquotesingle{}Nombre: }\SpecialCharTok{\{}\VariableTok{self}\SpecialCharTok{.}\NormalTok{nombre}\SpecialCharTok{\}}\SpecialStringTok{, Salario: }\SpecialCharTok{\{}\VariableTok{self}\SpecialCharTok{.}\NormalTok{salario}\SpecialCharTok{\}}\SpecialStringTok{, Años de Experiencia: }\SpecialCharTok{\{}\VariableTok{self}\SpecialCharTok{.}\NormalTok{exp}\SpecialCharTok{\}}\SpecialStringTok{\textquotesingle{}}

    \KeywordTok{def}\NormalTok{ mostrarInfo(}\VariableTok{self}\NormalTok{):}
        \BuiltInTok{print}\NormalTok{(}\SpecialStringTok{f\textquotesingle{}Nombre: }\SpecialCharTok{\{}\VariableTok{self}\SpecialCharTok{.}\NormalTok{nombre}\SpecialCharTok{\}}\SpecialStringTok{, Salario: }\SpecialCharTok{\{}\VariableTok{self}\SpecialCharTok{.}\NormalTok{salario}\SpecialCharTok{\}}\SpecialStringTok{, Años de Experiencia: }\SpecialCharTok{\{}\VariableTok{self}\SpecialCharTok{.}\NormalTok{exp}\SpecialCharTok{\}}\SpecialStringTok{\textquotesingle{}}\NormalTok{)}

    \KeywordTok{def}\NormalTok{ aumento(}\VariableTok{self}\NormalTok{):}
\NormalTok{        aumento }\OperatorTok{=} \FloatTok{0.05}\OperatorTok{*}\VariableTok{self}\NormalTok{.exp}
        \VariableTok{self}\NormalTok{.salario }\OperatorTok{+=} \VariableTok{self}\NormalTok{.salario}\OperatorTok{*}\NormalTok{aumento}

\CommentTok{\# Ejemplo}
\NormalTok{empleado1 }\OperatorTok{=}\NormalTok{ Empleado(}\StringTok{"Carlos"}\NormalTok{, }\DecValTok{15000}\NormalTok{, }\DecValTok{10}\NormalTok{)}
\BuiltInTok{print}\NormalTok{(empleado1)}
\NormalTok{empleado1.mostrarInfo()}
\NormalTok{empleado1.aumento()}
\BuiltInTok{print}\NormalTok{(}\SpecialStringTok{f\textquotesingle{}Nuevo salario: }\SpecialCharTok{\{}\NormalTok{empleado1}\SpecialCharTok{.}\NormalTok{salario}\SpecialCharTok{\}}\SpecialStringTok{\textquotesingle{}}\NormalTok{)}
\end{Highlighting}
\end{Shaded}

\begin{verbatim}
Nombre: Carlos, Salario: 15000, Años de Experiencia: 10
Nombre: Carlos, Salario: 15000, Años de Experiencia: 10
Nuevo salario: 22500.0

\end{verbatim}
}\end{exercise}


\begin{exercise}{\rm \textbf{Clase CuentaDeAhorro.}
Crea una clase CuentaDeAhorro que herede de CuentaBancaria. Añade un
atributo interés y un método para aplicar el interés al saldo.

\begin{Shaded}
\begin{Highlighting}[]

\end{Highlighting}
\end{Shaded}
}\end{exercise}

\begin{exercise}{\rm \textbf{Clase Empresa.}
Crea una clase Empresa con atributos nombre y empleados (una lista de
objetos de la clase Empleado). Incluye métodos para agregar empleados,
eliminar empleados y calcular el salario total pagado por la empresa.

\begin{Shaded}
\begin{Highlighting}[]

\end{Highlighting}
\end{Shaded}
}\end{exercise}
