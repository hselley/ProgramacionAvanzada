\chapter{Sintáxis Básica de Python}
\section{Python}

\href{https://www.python.org/about/}{Python} es un lenguaje de
programación poderoso y rápido, se lleva bien con otros lenguajes, corre
en cualquier lugar, es amigable, fácil de aprender y es de software
libre.

\section{Sintáxis}

En la gran mayoría de los lenguajes de programación el compilador o
intérprete ignora los espacios que el usuario utilice para escribir su
código. En Python, los espacios se utilizan para formar grupos de
instrucciones y también para la sintaxis de algunas instrucciones.

\section{If-Elif-Else}

La sentencia de control \texttt{if-elif-else} tiene la siguiente
sintaxis:

Sintaxis:

\begin{Shaded}
\begin{Highlighting}[]
\ControlFlowTok{if}\NormalTok{ condición A:}
\NormalTok{    instrucción A1}
\NormalTok{    instrucción A2}
\NormalTok{    ...}
\NormalTok{    instrucción An}
\ControlFlowTok{elif}\NormalTok{ condición B:}
\NormalTok{    instrucción B1}
\NormalTok{    instrucción B2}
\NormalTok{    ...}
\NormalTok{    instrucción Bn}
\ControlFlowTok{else}\NormalTok{:}
\NormalTok{    instrucción C1}
\NormalTok{    instrucción C2}
\NormalTok{    ...}
\NormalTok{    instrucción Cn}

\end{Highlighting}
\end{Shaded}


\begin{code} if-else
\begin{Shaded}
\begin{Highlighting}[]
    \ControlFlowTok{if} \DecValTok{2}\OperatorTok{\textless{}}\DecValTok{3}\NormalTok{:}
        \BuiltInTok{print}\NormalTok{(}\StringTok{"verdadero"}\NormalTok{)}
    \ControlFlowTok{else}\NormalTok{:}
        \BuiltInTok{print}\NormalTok{(}\StringTok{"falso"}\NormalTok{)}
    \BuiltInTok{print}\NormalTok{(}\StringTok{"fin"}\NormalTok{)}

\end{Highlighting}
\end{Shaded}
\end{code}

\begin{code}
Solicite dos números x y \texttt{y}. Si \texttt{x}
es positivo \texttt{y} se multiplica por dos, si \texttt{x} es cero
\texttt{y} se multiplica por 3 y si \texttt{x} es negativo y se
multiplica por cuatro.

\begin{Shaded}
\begin{Highlighting}[]
\NormalTok{    x}\OperatorTok{=}\BuiltInTok{int}\NormalTok{(}\BuiltInTok{input}\NormalTok{(}\StringTok{"Ingrese el valor de x "}\NormalTok{))}
\NormalTok{    y}\OperatorTok{=}\BuiltInTok{int}\NormalTok{(}\BuiltInTok{input}\NormalTok{(}\StringTok{"Ingrese el valor de y "}\NormalTok{))}
    \ControlFlowTok{if}\NormalTok{ x}\OperatorTok{\textgreater{}}\DecValTok{0}\NormalTok{:}
\NormalTok{        z}\OperatorTok{=}\NormalTok{y}\OperatorTok{*}\DecValTok{2}
        \BuiltInTok{print}\NormalTok{(}\StringTok{"Positivo"}\NormalTok{)}
    \ControlFlowTok{elif}\NormalTok{ x}\OperatorTok{==}\DecValTok{0}\NormalTok{:}
\NormalTok{        z}\OperatorTok{=}\NormalTok{y}\OperatorTok{*}\DecValTok{3}
        \BuiltInTok{print}\NormalTok{(}\StringTok{"Cero"}\NormalTok{)}
    \ControlFlowTok{else}\NormalTok{:}
\NormalTok{        z}\OperatorTok{=}\NormalTok{y}\OperatorTok{*}\DecValTok{4}
        \BuiltInTok{print}\NormalTok{(}\StringTok{"Negativo"}\NormalTok{)}
    \BuiltInTok{print}\NormalTok{(z)}
\end{Highlighting}
\end{Shaded}
\end{code}

\section{For}

El ciclo \texttt{for} tiene la siguiente sintaxis:

\begin{Shaded}
\begin{Highlighting}[]
\ControlFlowTok{for}\NormalTok{ iterador }\KeywordTok{in}\NormalTok{ iterando:}
\NormalTok{    instrucción }\DecValTok{1}
\NormalTok{    instrucción }\DecValTok{2}
\NormalTok{    ...}
\NormalTok{    instrucción n}

\end{Highlighting}
\end{Shaded}

\begin{code} Un ciclo que imprime numeros enteros desde 0 hasta 9.

\begin{Shaded}
\begin{Highlighting}[]
\ControlFlowTok{for}\NormalTok{ i }\KeywordTok{in} \BuiltInTok{range}\NormalTok{(}\DecValTok{10}\NormalTok{):}
    \BuiltInTok{print}\NormalTok{(i)}
\end{Highlighting}
\end{Shaded}
    
\end{code}
 

La función \texttt{range} genera una secuencia de números enteros
comenzando desde 0, con un incremento unitario. Por esa razón es que se
generan los números enteros desde 0 hasta 9. Es posible cambiar el valor
inicial de la secuencia y el incremento.

\begin{code} Considere los primeros 10 números naturales, los
pimeros 5 urales se multiplican por 2 y los siguientes números naturales
se multiplican por 3.

\begin{Shaded}
\begin{Highlighting}[]
    \ControlFlowTok{for}\NormalTok{ x }\KeywordTok{in} \BuiltInTok{range}\NormalTok{(}\DecValTok{1}\NormalTok{,}\DecValTok{11}\NormalTok{):}
        \ControlFlowTok{if}\NormalTok{ x}\OperatorTok{\textless{}=}\DecValTok{5}\NormalTok{:}
            \BuiltInTok{print}\NormalTok{(}\SpecialStringTok{f\textquotesingle{}}\SpecialCharTok{\{}\NormalTok{x}\SpecialCharTok{\}}\SpecialStringTok{ {-}\textgreater{} }\SpecialCharTok{\{}\NormalTok{x}\OperatorTok{*}\DecValTok{2}\SpecialCharTok{\}}\SpecialStringTok{\textquotesingle{}}\NormalTok{)}
        \ControlFlowTok{else}\NormalTok{:}
            \BuiltInTok{print}\NormalTok{(}\SpecialStringTok{f\textquotesingle{}}\SpecialCharTok{\{}\NormalTok{x}\SpecialCharTok{\}}\SpecialStringTok{ {-}\textgreater{} }\SpecialCharTok{\{}\NormalTok{x}\OperatorTok{*}\DecValTok{3}\SpecialCharTok{\}}\SpecialStringTok{\textquotesingle{}}\NormalTok{)}
\end{Highlighting}
\end{Shaded}
\end{code}

\section{While} 
El ciclo \texttt{while} repite las instrucciones siempre y cuando la
condición sea verdadera. La sintaxis del ciclo \texttt{while} es la
siguiente:

\begin{Shaded}
\begin{Highlighting}[]
    \ControlFlowTok{while}\NormalTok{ condición:}
\NormalTok{        instrucción }\DecValTok{1}
\NormalTok{        instrucción }\DecValTok{2}
\NormalTok{        ...}
\NormalTok{        instrucción n}

\end{Highlighting}
\end{Shaded}

\begin{code} Se imprimen los primeros 10 numeros naturales.

\begin{Shaded}
\begin{Highlighting}[]
\NormalTok{    i }\OperatorTok{=} \DecValTok{1}
    \ControlFlowTok{while}\NormalTok{ i }\OperatorTok{\textless{}=} \DecValTok{10}\NormalTok{:}
        \BuiltInTok{print}\NormalTok{(i)}
\NormalTok{        i}\OperatorTok{=}\NormalTok{i}\OperatorTok{+}\DecValTok{1}
\end{Highlighting}
\end{Shaded}
\end{code}

\section{Listas}

\subsection{Creación}

Una lista es un conjunto de valores ordenados y modificables que permite
repeticiones. Los elementos en la lista son numerados comenzando desde
0, de manera que la localidad donde se encuentra el último elemento para
una lista de tamaño \texttt{n} será \texttt{n-1}.

\begin{Shaded}
\begin{Highlighting}[]
\NormalTok{    lista }\OperatorTok{=}\NormalTok{ [}\StringTok{"manzana"}\NormalTok{, }\StringTok{"plátano"}\NormalTok{, }\StringTok{"naranja"}\NormalTok{, }\StringTok{"mandarina"}\NormalTok{, }\StringTok{"maracuyá"}\NormalTok{]}
    \BuiltInTok{print}\NormalTok{(lista)}
\end{Highlighting}
\end{Shaded}

\subsection{Acceso}

El acceso a algún elemento en particular de la lista se hace utilizando
los corchetes, indicando dentro de ellos la posición que ocupa dentro de
la lista.

\begin{Shaded}
\begin{Highlighting}[]
\NormalTok{    lista[posición]}
\end{Highlighting}
\end{Shaded}

\subsection{Índices negativos}

Es posible usar numeración negativa para hacer referencia a los
elementos dentro de una lista pero en orden reverso. Es decir,
\texttt{-1} hace referencia al último elemento, \texttt{-2} al
penúltimo, \texttt{-3} al antepenúltimo y así sucesivamente.

\subsection{Rango de índices}

Es posible seleccionar un subconjunto de elementos contiguos contenidos
en la lista especificando el rango de posiciones que ocupan dentro de la
lista.

\begin{Shaded}
\begin{Highlighting}[]
\NormalTok{    lista }\OperatorTok{=}\NormalTok{ [}\StringTok{"manzana"}\NormalTok{, }\StringTok{"plátano"}\NormalTok{, }\StringTok{"naranja"}\NormalTok{, }\StringTok{"mandarina"}\NormalTok{, }\StringTok{"maracuyá"}\NormalTok{, }
            \StringTok{"toronja"}\NormalTok{, }\StringTok{"mango"}\NormalTok{, }\StringTok{"guayaba"}\NormalTok{]}
    \BuiltInTok{print}\NormalTok{(lista[}\DecValTok{2}\NormalTok{:}\DecValTok{6}\NormalTok{])}
\end{Highlighting}
\end{Shaded}

Por otro lado, si se omite el límite inferior del rango, python
considerará 0 como posición inicial.

\begin{Shaded}
\begin{Highlighting}[]
\NormalTok{    lista }\OperatorTok{=}\NormalTok{ [}\StringTok{"manzana"}\NormalTok{, }\StringTok{"plátano"}\NormalTok{, }\StringTok{"naranja"}\NormalTok{, }\StringTok{"mandarina"}\NormalTok{, }\StringTok{"maracuyá"}\NormalTok{, }
            \StringTok{"toronja"}\NormalTok{, }\StringTok{"mango"}\NormalTok{, }\StringTok{"guayaba"}\NormalTok{]}
    \BuiltInTok{print}\NormalTok{(lista[:}\DecValTok{6}\NormalTok{])}
\end{Highlighting}
\end{Shaded}

Finalmente, si se omite el límite superior del rango, python considerará
el último elemento como tal límite.

\begin{Shaded}
\begin{Highlighting}[]
\NormalTok{    lista }\OperatorTok{=}\NormalTok{ [}\StringTok{"manzana"}\NormalTok{, }\StringTok{"plátano"}\NormalTok{, }\StringTok{"naranja"}\NormalTok{, }\StringTok{"mandarina"}\NormalTok{, }\StringTok{"maracuyá"}\NormalTok{, }
            \StringTok{"toronja"}\NormalTok{, }\StringTok{"mango"}\NormalTok{, }\StringTok{"guayaba"}\NormalTok{]}
    \BuiltInTok{print}\NormalTok{(lista[}\DecValTok{2}\NormalTok{:])}
\end{Highlighting}
\end{Shaded}

\subsection{Tamaño de una lista}

Para determinar el tamaño de una lista se puede utilizar la función
\texttt{len()}

\begin{Shaded}
\begin{Highlighting}[]
    \BuiltInTok{print}\NormalTok{(}\BuiltInTok{len}\NormalTok{(lista))}
\end{Highlighting}
\end{Shaded}

\section{Funciones}

En python la declaración de funciones requiere muy poco código. Basta
con utilizar la palabra clave \texttt{def} para comenzar la definición
de la función.

\begin{Shaded}
\begin{Highlighting}[]
    \KeywordTok{def}\NormalTok{ nombreFuncion(parámetroEntrada):}
\NormalTok{        instrucción }\DecValTok{1}
\NormalTok{        instrucción }\DecValTok{2}
\NormalTok{        ...}
\NormalTok{        instrucción n}
        \ControlFlowTok{return}\NormalTok{ parámetroSalida}

\end{Highlighting}
\end{Shaded}

\begin{code} Escriba una función que calcule el factorial de una
función. Verifique que el número ingresado por el usuario sea positivo y
considere que por definición el factorial de cero es uno.

\begin{Shaded}
\begin{Highlighting}[]
    \KeywordTok{def}\NormalTok{ factorial(x):}
        \ControlFlowTok{if}\NormalTok{ x}\OperatorTok{\textgreater{}=}\DecValTok{0}\NormalTok{:}
\NormalTok{            f}\OperatorTok{=}\DecValTok{1}
        \ControlFlowTok{if}\NormalTok{ x}\OperatorTok{\textgreater{}}\DecValTok{0}\NormalTok{:}
\NormalTok{            f}\OperatorTok{=}\DecValTok{1}
            \ControlFlowTok{for}\NormalTok{ i }\KeywordTok{in} \BuiltInTok{range}\NormalTok{(}\DecValTok{1}\NormalTok{,x}\OperatorTok{+}\DecValTok{1}\NormalTok{):}
\NormalTok{            f}\OperatorTok{=}\NormalTok{f}\OperatorTok{*}\NormalTok{i}
            \ControlFlowTok{return}\NormalTok{ f}
        \ControlFlowTok{else}\NormalTok{:}
            \BuiltInTok{print}\NormalTok{(}\StringTok{"El factorial no está definido"}\NormalTok{)}

\NormalTok{    x }\OperatorTok{=} \BuiltInTok{int}\NormalTok{(}\BuiltInTok{input}\NormalTok{(}\StringTok{"Introduzca un número: "}\NormalTok{))}
    \BuiltInTok{print}\NormalTok{(}\SpecialStringTok{f\textquotesingle{}}\SpecialCharTok{\{}\NormalTok{x}\SpecialCharTok{\}}\SpecialStringTok{! = }\SpecialCharTok{\{}\NormalTok{factorial(x)}\SpecialCharTok{\}}\SpecialStringTok{\textquotesingle{}}\NormalTok{)      }

\end{Highlighting}
\end{Shaded}
\end{code}

\begin{code} Escriba una función que calcule el factorial de una
función. Considere la definción recursiva del factorial. Verifique que
el número ingresado por el usuario sea positivo y considere que por
definición el factorial de cero es uno.

\begin{Shaded}
\begin{Highlighting}[]
    \KeywordTok{def}\NormalTok{ factorial(x):}
        \ControlFlowTok{if}\NormalTok{ x}\OperatorTok{==}\DecValTok{0}\NormalTok{:}
\NormalTok{            f}\OperatorTok{=}\DecValTok{1}
        \ControlFlowTok{else}\NormalTok{:}
\NormalTok{            f}\OperatorTok{=}\NormalTok{x}\OperatorTok{*}\NormalTok{factorial(x}\OperatorTok{{-}}\DecValTok{1}\NormalTok{)}
        \ControlFlowTok{return}\NormalTok{(f)}

\NormalTok{    x }\OperatorTok{=} \BuiltInTok{int}\NormalTok{(}\BuiltInTok{input}\NormalTok{(}\StringTok{"Ingrese un número"}\NormalTok{))}
    \ControlFlowTok{if}\NormalTok{ x}\OperatorTok{\textgreater{}=}\DecValTok{0}\NormalTok{:}
\NormalTok{        fac}\OperatorTok{=}\NormalTok{factorial(x)}
        \BuiltInTok{print}\NormalTok{(}\SpecialStringTok{f\textquotesingle{}}\SpecialCharTok{\{}\NormalTok{x}\SpecialCharTok{\}}\SpecialStringTok{! = }\SpecialCharTok{\{}\NormalTok{fac}\SpecialCharTok{\}}\SpecialStringTok{\textquotesingle{}}\NormalTok{)}
    \ControlFlowTok{else}\NormalTok{:}
        \BuiltInTok{print}\NormalTok{(}\StringTok{"El factorial no está definido en los negativos"}\NormalTok{)    }

\end{Highlighting}
\end{Shaded}
\end{code}

\begin{code} Escriba una función recursiva que imprima los elementos de una lista anidada en varios niveles.

\begin{Shaded}
\begin{Highlighting}[]
\NormalTok{    lista1 }\OperatorTok{=}\NormalTok{ [}\StringTok{"manzana"}\NormalTok{, }\StringTok{"plátano"}\NormalTok{, }\StringTok{"naranja"}\NormalTok{, }\StringTok{"mandarina"}\NormalTok{, }\StringTok{"maracuyá"}\NormalTok{, }\StringTok{"toronja"}\NormalTok{, }
            \StringTok{"mango"}\NormalTok{, }\StringTok{"guayaba"}\NormalTok{]}
\NormalTok{    lista2 }\OperatorTok{=}\NormalTok{ [}\DecValTok{1}\NormalTok{, }\DecValTok{2}\NormalTok{, }\DecValTok{3}\NormalTok{, }\DecValTok{4}\NormalTok{, }\DecValTok{5}\NormalTok{]}
\NormalTok{    lista3 }\OperatorTok{=}\NormalTok{ [}\StringTok{"A"}\NormalTok{, }\StringTok{"B"}\NormalTok{, }\StringTok{"C"}\NormalTok{, }\StringTok{"D"}\NormalTok{]}
\NormalTok{    lista4 }\OperatorTok{=}\NormalTok{ [}\StringTok{"a"}\NormalTok{, }\DecValTok{1}\NormalTok{, }\StringTok{"b"}\NormalTok{, }\DecValTok{2}\NormalTok{, }\StringTok{"c"}\NormalTok{, }\DecValTok{3}\NormalTok{, }\StringTok{"d"}\NormalTok{, }\DecValTok{4}\NormalTok{]}
\NormalTok{    superLista }\OperatorTok{=}\NormalTok{ [lista1 }\OperatorTok{+}\NormalTok{ lista2 }\OperatorTok{+}\NormalTok{ lista3] }\OperatorTok{+}\NormalTok{ [[lista4]] }\OperatorTok{+}\NormalTok{ lista1}

    \KeywordTok{def}\NormalTok{ impresionRecursiva(listaAnidada):}
    \ControlFlowTok{for}\NormalTok{ elemento }\KeywordTok{in}\NormalTok{ listaAnidada:}
        \ControlFlowTok{if} \BuiltInTok{isinstance}\NormalTok{(elemento, }\BuiltInTok{list}\NormalTok{):}
\NormalTok{            impresionRecursiva(elemento)}
        \ControlFlowTok{else}\NormalTok{: }
            \BuiltInTok{print}\NormalTok{(elemento)}

\NormalTok{    impresionRecursiva(superLista)}
\end{Highlighting}
\end{Shaded}
\end{code}

\section{Arreglos}

Python no tiene de forma nativa soporte para arreglos, en su lugar opta
por usar listas anidadas. Sin embargo, es posible utilizar el paquete
\href{https://numpy.org/}{NumPy} para utilizar los arreglos de manera
semejante a la existente en otros lenguajes. \textbf{NumPy} además
resulta ser más eficiente en el manejo de datos que su contraparte
nativa de python mediante listas anidadas. Adicionalmente,
\textbf{NumPy} incluye más herramientas que extienden la funcionalidad
de Python.

\subsection{Declaración}

Sintaxis:

\begin{Shaded}
\begin{Highlighting}[]
    \ImportTok{import}\NormalTok{ numpy}
\NormalTok{    arreglo }\OperatorTok{=}\NormalTok{ numpy.array([}\DecValTok{1}\NormalTok{, }\DecValTok{2}\NormalTok{, }\DecValTok{3}\NormalTok{, }\DecValTok{4}\NormalTok{, }\DecValTok{5}\NormalTok{])}
    \BuiltInTok{print}\NormalTok{(arreglo)}
\end{Highlighting}
\end{Shaded}

Esta sintaxis puede resultar incómoda porque será necesario escribirla
todas las veces que necesite declarar un arreglo. Una alternativa para
simplificar un poco esta declaración es mediante la creación de un
alias, esto de la siguiente manera:

\begin{Shaded}
\begin{Highlighting}[]
    \ImportTok{import}\NormalTok{ numpy }\ImportTok{as}\NormalTok{ np}
\NormalTok{    arreglo }\OperatorTok{=}\NormalTok{ np.array([}\DecValTok{1}\NormalTok{, }\DecValTok{2}\NormalTok{, }\DecValTok{3}\NormalTok{, }\DecValTok{4}\NormalTok{, }\DecValTok{5}\NormalTok{])}
    \BuiltInTok{print}\NormalTok{(arreglo)}
\end{Highlighting}
\end{Shaded}

Para declarar un arreglo bidimensional se utiliza la siguiente sintáxis:

\begin{Shaded}
\begin{Highlighting}[]
    \ImportTok{import}\NormalTok{ numpy }\ImportTok{as}\NormalTok{ np}
\NormalTok{    arreglo }\OperatorTok{=}\NormalTok{ np.array([[}\DecValTok{1}\NormalTok{, }\DecValTok{2}\NormalTok{, }\DecValTok{3}\NormalTok{], [}\DecValTok{4}\NormalTok{, }\DecValTok{5}\NormalTok{, }\DecValTok{6}\NormalTok{]])}
    \BuiltInTok{print}\NormalTok{(arreglo) }
\end{Highlighting}
\end{Shaded}

El atributo \texttt{ndim} devuelve la cantidad de dimensiones que tiene
un arreglo.

\begin{Shaded}
\begin{Highlighting}[]
    \ImportTok{import}\NormalTok{ numpy }\ImportTok{as}\NormalTok{ np}
\NormalTok{    arreglo }\OperatorTok{=}\NormalTok{ np.array([[}\DecValTok{1}\NormalTok{, }\DecValTok{2}\NormalTok{, }\DecValTok{3}\NormalTok{], [}\DecValTok{4}\NormalTok{, }\DecValTok{5}\NormalTok{, }\DecValTok{6}\NormalTok{]])}
    \BuiltInTok{print}\NormalTok{(arreglo.ndim) }
\end{Highlighting}
\end{Shaded}

\subsection{Acceso a elementos de un arreglo}

El acceso a elementos dentro de un arreglo en \texttt{numpy} es similar
a la forma que se utiliza para las listas. Recuerde que elos índice para
los elementos dentro del arreglo comienza en 0.

\begin{Shaded}
\begin{Highlighting}[]
    \ImportTok{import}\NormalTok{ numpy }\ImportTok{as}\NormalTok{ np}
\NormalTok{    arreglo }\OperatorTok{=}\NormalTok{ np.array([}\DecValTok{1}\NormalTok{, }\DecValTok{2}\NormalTok{, }\DecValTok{3}\NormalTok{, }\DecValTok{4}\NormalTok{, }\DecValTok{5}\NormalTok{])}
    \BuiltInTok{print}\NormalTok{(arreglo[}\DecValTok{0}\NormalTok{])}
\end{Highlighting}
\end{Shaded}

Para el caso de un arreglo bidimensional se utiliza una coma para separar la posición de las dimensiones.

\begin{Shaded}
\begin{Highlighting}[]
    \ImportTok{import}\NormalTok{ numpy }\ImportTok{as}\NormalTok{ np}
\NormalTok{    arreglo }\OperatorTok{=}\NormalTok{ np.array([[}\DecValTok{1}\NormalTok{, }\DecValTok{2}\NormalTok{, }\DecValTok{3}\NormalTok{], [}\DecValTok{4}\NormalTok{, }\DecValTok{5}\NormalTok{, }\DecValTok{6}\NormalTok{]])}
    \BuiltInTok{print}\NormalTok{(arreglo[}\DecValTok{1}\NormalTok{, }\DecValTok{2}\NormalTok{]) }
\end{Highlighting}
\end{Shaded}

Si el arreglo tiene más dimensiones se utiliza la misma idea para cada una de ellas.

\begin{Shaded}
\begin{Highlighting}[]
    \ImportTok{import}\NormalTok{ numpy }\ImportTok{as}\NormalTok{ np}
\NormalTok{    arreglo }\OperatorTok{=}\NormalTok{ np.array([[[}\DecValTok{1}\NormalTok{, }\DecValTok{2}\NormalTok{, }\DecValTok{3}\NormalTok{], [}\DecValTok{4}\NormalTok{, }\DecValTok{5}\NormalTok{, }\DecValTok{6}\NormalTok{]],[[}\DecValTok{7}\NormalTok{, }\DecValTok{8}\NormalTok{, }\DecValTok{9}\NormalTok{], [}\DecValTok{10}\NormalTok{, }\DecValTok{11}\NormalTok{, }\DecValTok{12}\NormalTok{]])}
    \BuiltInTok{print}\NormalTok{(arreglo[}\DecValTok{0}\NormalTok{, }\DecValTok{1}\NormalTok{, }\DecValTok{2}\NormalTok{]) }
\end{Highlighting}
\end{Shaded}

De la misma forma que con las listas, también es posible utilizar índices negativos.

\begin{Shaded}
\begin{Highlighting}[]
\NormalTok{    arreglo }\OperatorTok{=}\NormalTok{ np.array([[}\DecValTok{1}\NormalTok{,}\DecValTok{2}\NormalTok{,}\DecValTok{3}\NormalTok{,}\DecValTok{4}\NormalTok{,}\DecValTok{5}\NormalTok{], [}\DecValTok{6}\NormalTok{,}\DecValTok{7}\NormalTok{,}\DecValTok{8}\NormalTok{,}\DecValTok{9}\NormalTok{,}\DecValTok{10}\NormalTok{]])}
    \BuiltInTok{print}\NormalTok{(}\StringTok{\textquotesingle{}El último elemento en el arreglo bidimensional\textquotesingle{}}\NormalTok{, arreglo[}\DecValTok{1}\NormalTok{, }\OperatorTok{{-}}\DecValTok{1}\NormalTok{]) }
\end{Highlighting}
\end{Shaded}

\subsection{Cortes de arreglos}

Es posible \emph{cortar} un subconjunto de un arreglo para definir uno
nuevo. Esto es de especial utilidad para extraer vectores de una matriz
existente, ya sea para definir un nuevo vector o bien realizar
operaciones con el.

El corte (o rebanada) de la matriz se hace indicando un rango de
posiciones, es decir \texttt{{[}inicio\ :\ fin{]}}. Además se puede
especificar un incremento \texttt{{[}inicio\ :\ fin\ :\ incremento{]}}.
Si no se especifica un inicio, de asume como 0, y si no se especifica un
final se asume el último elemento de la matriz. Si no se especifica un
incremento, se asume como 1.

\begin{Shaded}
\begin{Highlighting}[]
    \ImportTok{import}\NormalTok{ numpy }\ImportTok{as}\NormalTok{ np}
\NormalTok{    arreglo }\OperatorTok{=}\NormalTok{ np.array([}\DecValTok{1}\NormalTok{, }\DecValTok{2}\NormalTok{, }\DecValTok{3}\NormalTok{, }\DecValTok{4}\NormalTok{, }\DecValTok{5}\NormalTok{])}
\NormalTok{    arreglo2 }\OperatorTok{=}\NormalTok{ arreglo[}\DecValTok{1}\NormalTok{:}\DecValTok{4}\NormalTok{]}
    \BuiltInTok{print}\NormalTok{(arreglo2)}
\end{Highlighting}
\end{Shaded}

En el siguiente ejemplo, se hace una rebanada de la matriz especificando
un incremento diferente a uno:

\begin{Shaded}
\begin{Highlighting}[]
    \ImportTok{import}\NormalTok{ numpy }\ImportTok{as}\NormalTok{ np}
\NormalTok{    arreglo }\OperatorTok{=}\NormalTok{ np.array([}\DecValTok{1}\NormalTok{, }\DecValTok{2}\NormalTok{, }\DecValTok{3}\NormalTok{, }\DecValTok{4}\NormalTok{, }\DecValTok{5}\NormalTok{, }\DecValTok{6}\NormalTok{, }\DecValTok{7}\NormalTok{])}
    \BuiltInTok{print}\NormalTok{(arreglo[}\DecValTok{1}\NormalTok{:}\DecValTok{5}\NormalTok{:}\DecValTok{2}\NormalTok{])}
\end{Highlighting}
\end{Shaded}

\subsection{Cortes de arreglos bidimensionales}

Es posible realizar rebanadas de arreglos de 2 ó más dimensiones,
resultando un vector o una matriz según sea el caso.

El siguiente ejemplo realiza una rebanada de un arreglo bidimensional y
el resultado es un vector. Observe que el vector es una rebanada de la
segunda dimensión de la matriz.

\begin{Shaded}
\begin{Highlighting}[]
    \ImportTok{import}\NormalTok{ numpy }\ImportTok{as}\NormalTok{ np}
\NormalTok{    arreglo }\OperatorTok{=}\NormalTok{ np.array([[}\DecValTok{1}\NormalTok{, }\DecValTok{2}\NormalTok{, }\DecValTok{3}\NormalTok{, }\DecValTok{4}\NormalTok{, }\DecValTok{5}\NormalTok{], [}\DecValTok{6}\NormalTok{, }\DecValTok{7}\NormalTok{, }\DecValTok{8}\NormalTok{, }\DecValTok{9}\NormalTok{, }\DecValTok{10}\NormalTok{]])}
    \BuiltInTok{print}\NormalTok{(arreglo[}\DecValTok{1}\NormalTok{, }\DecValTok{1}\NormalTok{:}\DecValTok{4}\NormalTok{])}
\end{Highlighting}
\end{Shaded}

En este ejemplo se hace una rebanada de un arreglo bidimensional y el
resultado es nuevamente un vector. En esta ocasión el vector es una
rebanada vertical por lo que el resultado contiene elementos de ambas
dimensiones de la matriz.

\begin{Shaded}
\begin{Highlighting}[]
    \ImportTok{import}\NormalTok{ numpy }\ImportTok{as}\NormalTok{ np}
\NormalTok{    arreglo }\OperatorTok{=}\NormalTok{ np.array([[}\DecValTok{1}\NormalTok{, }\DecValTok{2}\NormalTok{, }\DecValTok{3}\NormalTok{, }\DecValTok{4}\NormalTok{, }\DecValTok{5}\NormalTok{], [}\DecValTok{6}\NormalTok{, }\DecValTok{7}\NormalTok{, }\DecValTok{8}\NormalTok{, }\DecValTok{9}\NormalTok{, }\DecValTok{10}\NormalTok{]])}
    \BuiltInTok{print}\NormalTok{(arreglo[}\DecValTok{0}\NormalTok{:}\DecValTok{2}\NormalTok{, }\DecValTok{2}\NormalTok{])}
\end{Highlighting}
\end{Shaded}

En este último caso, se hace una rebanada que resulta una matriz que
contiene elementos de ambas dimensiones de la matriz original.

\begin{Shaded}
\begin{Highlighting}[]
    \ImportTok{import}\NormalTok{ numpy }\ImportTok{as}\NormalTok{ np}
\NormalTok{    arreglo }\OperatorTok{=}\NormalTok{ np.array([[}\DecValTok{1}\NormalTok{, }\DecValTok{2}\NormalTok{, }\DecValTok{3}\NormalTok{, }\DecValTok{4}\NormalTok{, }\DecValTok{5}\NormalTok{], [}\DecValTok{6}\NormalTok{, }\DecValTok{7}\NormalTok{, }\DecValTok{8}\NormalTok{, }\DecValTok{9}\NormalTok{, }\DecValTok{10}\NormalTok{]])}
    \BuiltInTok{print}\NormalTok{(arreglo[}\DecValTok{0}\NormalTok{:}\DecValTok{2}\NormalTok{, }\DecValTok{1}\NormalTok{:}\DecValTok{4}\NormalTok{])}
\end{Highlighting}
\end{Shaded}

\subsection{Arreglos aleatorios}

Es posible generar un arreglo o matriz lleno de números (pseudo)
aleatorios, para ello se puede utilizar el método \texttt{rand}. Para la
generación, basta con especificar las dimensiones del arreglo que se
desea generar.\\


\begin{code} Generación de un vector unidimensional de 20 elementos aleatorios.

\begin{Shaded}
\begin{Highlighting}[]
    \ImportTok{import}\NormalTok{ numpy }\ImportTok{as}\NormalTok{ np}
\NormalTok{    arreglo }\OperatorTok{=}\NormalTok{ np.random.rand(}\DecValTok{20}\NormalTok{) }
    \BuiltInTok{print}\NormalTok{(arreglo)}

\end{Highlighting}
\end{Shaded}
\end{code}

\begin{code} Generación de una matriz bidimensional de tamaño \texttt{5x3}, es decir 5 renglones y 3 columnas.

\begin{Shaded}
\begin{Highlighting}[]
    \ImportTok{import}\NormalTok{ numpy }\ImportTok{as}\NormalTok{ np}
\NormalTok{    arreglo }\OperatorTok{=}\NormalTok{ np.random.rand(}\DecValTok{5}\NormalTok{, }\DecValTok{3}\NormalTok{) }
    \BuiltInTok{print}\NormalTok{(arreglo)}

\end{Highlighting}
\end{Shaded}
\end{code}

\begin{code} Generación de una matriz tridimensional de tamaño \texttt{5x3x2}, es decir 5 matrices de 3 renglones y 
2 columnas cada una de ellas.

\begin{Shaded}
\begin{Highlighting}[]
    \ImportTok{import}\NormalTok{ numpy }\ImportTok{as}\NormalTok{ np}
\NormalTok{    arreglo }\OperatorTok{=}\NormalTok{ np.random.rand(}\DecValTok{5}\NormalTok{, }\DecValTok{3}\NormalTok{, }\DecValTok{2}\NormalTok{) }
    \BuiltInTok{print}\NormalTok{(arreglo)}

\end{Highlighting}
\end{Shaded}
\end{code}

\subsection{Añadir elementos a un arreglo}

Para añadir elementos al final de un arreglo, se puede utilizar el
método \texttt{append} contenido en la biblioteca \texttt{Numpy}. El
argumento \texttt{axis} permite especificar el lugar donde serán
añadidos los elementos al arreglo.\\

\begin{code}
Añadir una matriz al final de una matriz, justo debajo de la matriz inicial (\texttt{axis=0}).

\begin{Shaded}
\begin{Highlighting}[]
    \ImportTok{import}\NormalTok{ numpy }\ImportTok{as}\NormalTok{ np}
\NormalTok{    arreglo }\OperatorTok{=}\NormalTok{ np.append([[}\DecValTok{1}\NormalTok{, }\DecValTok{2}\NormalTok{], [}\DecValTok{3}\NormalTok{, }\DecValTok{4}\NormalTok{]], [[}\DecValTok{10}\NormalTok{, }\DecValTok{20}\NormalTok{], [}\DecValTok{30}\NormalTok{, }\DecValTok{40}\NormalTok{]], axis}\OperatorTok{=}\DecValTok{0}\NormalTok{)}
    \BuiltInTok{print}\NormalTok{(arreglo)}

\end{Highlighting}
\end{Shaded}
\end{code}

\begin{code}
Añadir una matriz al final de la matriz, a la derehca de la matriz inicial (\texttt{axis=1}).

\begin{Shaded}
\begin{Highlighting}[]
    \ImportTok{import}\NormalTok{ numpy }\ImportTok{as}\NormalTok{ np}
\NormalTok{    arreglo }\OperatorTok{=}\NormalTok{ np.append([[}\DecValTok{1}\NormalTok{, }\DecValTok{2}\NormalTok{], [}\DecValTok{3}\NormalTok{, }\DecValTok{4}\NormalTok{]], [[}\DecValTok{10}\NormalTok{, }\DecValTok{20}\NormalTok{], [}\DecValTok{30}\NormalTok{, }\DecValTok{40}\NormalTok{]], axis}\OperatorTok{=}\DecValTok{1}\NormalTok{)}
    \BuiltInTok{print}\NormalTok{(arreglo)}

\end{Highlighting}
\end{Shaded}
\end{code}

\subsection{Formateo de impresión de un arreglo}

Si se desea controlar la forma en la que los números contenidos en un
arreglo serán impresos, se puede utilizar el método
\texttt{printoptions} de la biblioteca \texttt{Numpy}.\\

\begin{code}
Generar una matriz bidimensional de tamaño 10x5 llena
con números aleatorios no enteros con valores entre 0 y 1. Imprimir la
matriz mostrando únicamente 4 cifras significativas no enteras.

\begin{Shaded}
\begin{Highlighting}[]
    \ImportTok{import}\NormalTok{ numpy }\ImportTok{as}\NormalTok{ np}
\NormalTok{    x }\OperatorTok{=}\NormalTok{ np.random.rand(}\DecValTok{10}\NormalTok{,}\DecValTok{5}\NormalTok{)}
        \ControlFlowTok{with}\NormalTok{ np.printoptions(precision}\OperatorTok{=}\DecValTok{4}\NormalTok{, suppress}\OperatorTok{=}\VariableTok{True}\NormalTok{):}
\NormalTok{            np.set\_printoptions(formatter}\OperatorTok{=}\NormalTok{\{}\StringTok{\textquotesingle{}float\textquotesingle{}}\NormalTok{: }\StringTok{\textquotesingle{}}\SpecialCharTok{\{: 0.4f\}}\StringTok{\textquotesingle{}}\NormalTok{.}\BuiltInTok{format}\NormalTok{\})}
            \BuiltInTok{print}\NormalTok{(x)}

\end{Highlighting}
\end{Shaded}
\end{code}

La función \texttt{with} permite especificar el formato mediante el cual
serán impresos unicamente los números en el arreglo, dejando intactos
los parámetros de los demás \texttt{print} que pudiesen existir. El
parámetro \texttt{suppress=True} indica que los números deben ser
impresos en la forma de punto flotante y evitando la notación
científica. 
