\chapter{Manejo de archivos}

El manejo de archivos es una parte importante de cualquier aplicación
web.

Python tiene varias funciones para crear, leer, actualizar y eliminar
archivos.

\section{\texorpdfstring{Función
\texttt{open}}{Función open}}\label{funciuxf3n-open}

La función clave para trabajar con archivos en Python es
\texttt{open()}. La función \texttt{open()} toma dos parámetros:
\emph{nombre\_archivo} y \emph{modo} .

Hay cuatro métodos (modos) diferentes para abrir un archivo:

\begin{itemize}
  \item ``r'' - Leer - Valor predeterminado. Abre un archivo para leerlo. Se
    produce un error si el archivo no existe.
  \item ``a'' - Anexar - Abre un archivo para anexarlo, crea el archivo si no existe.
  \item ``w'' - Escribir - Abre un archivo para escribir, crea el archivo si no existe.
  \item ``x'' - Crear - Crea el archivo especificado, devuelve un error si el archivo existe.
\end{itemize}

Además, puede especificar si el archivo debe manejarse en modo binario o
de texto.

\begin{itemize}
  \item ``t'' - Texto - Valor predeterminado. Modo texto.
  \item ``b'' - Binario - Modo binario (por ejemplo, imágenes).
\end{itemize}

\section{Sintaxis}

La función \texttt{open()} abre un archivo y lo devuelve como un objeto
de archivo.

\begin{Shaded}
\begin{Highlighting}[]
\BuiltInTok{open}\NormalTok{(}\BuiltInTok{file}\NormalTok{, mode)}
\end{Highlighting}
\end{Shaded}

\section{Valores predeterminados}

\begin{longtable}[]{@{}
  >{\raggedright\arraybackslash}p{(\columnwidth - 2\tabcolsep) * \real{0.4583}}
  >{\raggedright\arraybackslash}p{(\columnwidth - 2\tabcolsep) * \real{0.5417}}@{}}
\toprule\noalign{}
\begin{minipage}[b]{\linewidth}\raggedright
Parámetro
\end{minipage} & \begin{minipage}[b]{\linewidth}\raggedright
Descripción
\end{minipage} \\
\midrule\noalign{}
\endhead
\bottomrule\noalign{}
\endlastfoot
\emph{file} & La ubicación y nombre del archivo \\
\emph{mode} & Una cadena que define el modo como se abre el archivo \\
& \texttt{``r''} - \textbf{Lectura} - Valor por defecto. Abre un archivo
para lectura, envía error si el archivo no existe. \\
& \texttt{``a''} - \textbf{Añadir} - Abre un archivo para añadir, crea el
archivo si no existe. \\
& \texttt{``w''} - \textbf{Escritura} - Abre un archivo para escritura,
crea el archivo si no existe. \\
& \texttt{``x''} - \textbf{Creación} - Crea un archivo, devuelve un error
si el archivo ya existe. \\
& \\
& Adicionalmente se puede especificar si el archivo será manejado como
texto o binario \\
& \texttt{``t''} - \textbf{Texto} - Modo texto, valor por defecto. \\
& \texttt{``b''} - \textbf{Binario} - Modo binario (i.e. imágenes) \\
\end{longtable}

Para abrir un archivo para su lectura es suficiente especificar el
nombre del archivo:

\begin{Shaded}
\begin{Highlighting}[]
\NormalTok{f }\OperatorTok{=} \BuiltInTok{open}\NormalTok{(}\StringTok{"demofile.txt"}\NormalTok{)}
\end{Highlighting}
\end{Shaded}

Este código es equivalente al siguiente:

\begin{Shaded}
\begin{Highlighting}[]
\NormalTok{f }\OperatorTok{=} \BuiltInTok{open}\NormalTok{(}\StringTok{"demofile.txt"}\NormalTok{, }\StringTok{"rt"}\NormalTok{)}
\end{Highlighting}
\end{Shaded}

Dado que \texttt{\textquotesingle{}r\textquotesingle{}} para lectura y
\texttt{\textquotesingle{}t\textquotesingle{}} para texto son los
valores por defecto, no es necesario incluirlos.

\section{Abrir un archivo localmente}

Supongamos que tenemos el archivo \texttt{demofile.txt}, ubicado en la
misma carpeta que el script de Python:

\textbf{Archivo demofile.txt}

\begin{Shaded}
\begin{Highlighting}[]
\NormalTok{Hello! Welcome to demofile.txt}
\NormalTok{This file is for testing purposes.}
\NormalTok{Good Luck!}
\end{Highlighting}
\end{Shaded}

Para abrir el archivo, se utiliza la función incorporada
\texttt{open()}. Esta función devuelve un objeto tipo archivo, que tiene
un método \texttt{read()} para leer el contenido del archivo:\\

\begin{code} Apertura y lectura de un archivo.
\begin{Shaded}
\begin{Highlighting}[]
\NormalTok{f }\OperatorTok{=} \BuiltInTok{open}\NormalTok{(}\StringTok{"demofile.txt"}\NormalTok{, }\StringTok{\textquotesingle{}r\textquotesingle{}}\NormalTok{)}
\BuiltInTok{print}\NormalTok{(f.read())}
\end{Highlighting}
\end{Shaded}

\begin{verbatim}
Hello! Welcome to demofile.txt
This file is for testing purposes.
Good Luck!Now the file has more content!
\end{verbatim}
\end{code}

\section{Abrir un archivo en Google Drive}

Supongamos que tenemos el archivo \texttt{demofile.txt}, ubicado en Google Drive:

\textbf{Archivo demofile.txt}

\begin{Shaded}
\begin{Highlighting}[]
\NormalTok{Hello! Welcome to demofile.txt}
\NormalTok{This file is for testing purposes.}
\NormalTok{Good Luck!}
\end{Highlighting}
\end{Shaded}

Además un notebook de Python se encuentra en la misma carpeta. Para
abrir un archivo es necesario utilizar la función \texttt{google.colab}
dentro del paquete \texttt{drive}. Una vez que se haya incluído el
paquete, ahora es necesario \emph{montar} la unidad de almacenamiento de
Google Drive en una ubicación dentro del sistema de archivos virtual:
\texttt{/drive}. Cabe mencionar que esta dirección apuntará a la raíz de
la unidad de almacenamiento en Google Drive. \\

\begin{code} Código en Colaboratory para abrir archivos.
\begin{Shaded}
\begin{Highlighting}[]
\ImportTok{from}\NormalTok{ google.colab }\ImportTok{import}\NormalTok{ drive}
\NormalTok{drive.mount(}\StringTok{"/drive"}\NormalTok{)}
\end{Highlighting}
\end{Shaded}
\end{code}

Al ejecutar la función \texttt{drive.mount}, Google Drive preguntará por
autorización para tener acceso a su unidad.

Posterior a esto, sólo es necesario utilizar la función open tal y como
lo hicimos anteriormente.

\begin{Shaded}
\begin{Highlighting}[]
\NormalTok{f }\OperatorTok{=} \BuiltInTok{open}\NormalTok{(}\StringTok{"/drive/My Drive/demofile.txt"}\NormalTok{, }\StringTok{\textquotesingle{}r\textquotesingle{}}\NormalTok{)}
\BuiltInTok{print}\NormalTok{(f.read())}
\end{Highlighting}
\end{Shaded}

Observe que se precede al nombre del archivo la ubicación
\texttt{/drive/My\ Drive/}, que es donde se montó virtualmente la unidad
de almacenamiento de Google Drive.

\section{Leer partes de un archivo}

Por defecto el método \texttt{read()} devuelve todo el texto, pero se
puede especificar cuantos caracteres se desea leer.\\

\begin{code} Lectura parcial del archivo.

\begin{Shaded}
\begin{Highlighting}[]
\NormalTok{f }\OperatorTok{=} \BuiltInTok{open}\NormalTok{(}\StringTok{"demofile.txt"}\NormalTok{, }\StringTok{"r"}\NormalTok{)}
\BuiltInTok{print}\NormalTok{(f.read(}\DecValTok{5}\NormalTok{))}
\end{Highlighting}
\end{Shaded}

\begin{verbatim}
Hello
\end{verbatim}
\end{code}

\section{Leer líneas del archivo}
Es posible leer una línea de un archivo mediante el método \texttt{readline()}. \\

\begin{code} Lectura de archivo mediante \texttt{readline()}.
\begin{Shaded}
\begin{Highlighting}[]
\NormalTok{f }\OperatorTok{=} \BuiltInTok{open}\NormalTok{(}\StringTok{"demofile.txt"}\NormalTok{, }\StringTok{"r"}\NormalTok{)}
\BuiltInTok{print}\NormalTok{(f.readline())}
\end{Highlighting}
\end{Shaded}

\begin{verbatim}
Hello! Welcome to demofile.txt

\end{verbatim}
\end{code}
\begin{Shaded}
\begin{Highlighting}[]
\NormalTok{f }\OperatorTok{=} \BuiltInTok{open}\NormalTok{(}\StringTok{"demofile.txt"}\NormalTok{, }\StringTok{"r"}\NormalTok{)}
\BuiltInTok{print}\NormalTok{(f.readline())}
\BuiltInTok{print}\NormalTok{(f.readline())}
\end{Highlighting}
\end{Shaded}

\begin{verbatim}
Hello! Welcome to demofile.txt

This file is for testing purposes.

\end{verbatim}

Si se utiliza un ciclo \texttt{for}, se puede leer todo el archivo línea por línea.
\\

\begin{code} Lectura mediante ciclo \texttt{for}.
\begin{Shaded}
\begin{Highlighting}[]
\NormalTok{f }\OperatorTok{=} \BuiltInTok{open}\NormalTok{(}\StringTok{"demofile.txt"}\NormalTok{, }\StringTok{"r"}\NormalTok{)}
\ControlFlowTok{for}\NormalTok{ x }\KeywordTok{in}\NormalTok{ f:}
    \BuiltInTok{print}\NormalTok{(x)}
\end{Highlighting}
\end{Shaded}

\begin{verbatim}
Hello! Welcome to demofile.txt

This file is for testing purposes.

Good Luck!Now the file has more content!
\end{verbatim}
\end{code}

\section{Cerrar un archivo}

Es una biena práctica cerrar un archivo después de utilizarse. Para ello
se debe utilizar el método \texttt{close()}.\\

\begin{code} Cerrar archivo después de su apertura.
\begin{Shaded}
\begin{Highlighting}[]
\NormalTok{f }\OperatorTok{=} \BuiltInTok{open}\NormalTok{(}\StringTok{"demofile.txt"}\NormalTok{, }\StringTok{"r"}\NormalTok{)}
\BuiltInTok{print}\NormalTok{(f.readline())}
\NormalTok{f.close()}
\end{Highlighting}
\end{Shaded}

\begin{verbatim}
Hello! Welcome to demofile.txt

\end{verbatim}

Se puede verificar si un archivo está cerrado mediante la propiedad
booleana \texttt{closed}.

\begin{Shaded}
\begin{Highlighting}[]
\BuiltInTok{print}\NormalTok{(}\SpecialStringTok{f\textquotesingle{}El archivo está cerrado?: }\SpecialCharTok{\{}\NormalTok{f}\SpecialCharTok{.}\NormalTok{closed}\SpecialCharTok{\}}\SpecialStringTok{\textquotesingle{}}\NormalTok{)}
\end{Highlighting}
\end{Shaded}

\begin{verbatim}
El archivo está cerrado?: True
\end{verbatim}
\end{code}

\section{Escribir en un archivo existente}

Para escribir en un archivo existente se debe agregar un parámetro al
método \texttt{open()}.

\begin{itemize}
  \item \texttt{\textquotesingle{}a\textquotesingle{}}. Añadir contenido al final del archivo
  \item \texttt{\textquotesingle{}w\textquotesingle{}}. Escribir en el archivo. Sobreescribe cualquier contenido existente.
\end{itemize}

\begin{code} Abrir el archivo \texttt{demofile2.txt} y añadir contenido.

\begin{Shaded}
\begin{Highlighting}[]
\NormalTok{f }\OperatorTok{=} \BuiltInTok{open}\NormalTok{(}\StringTok{"demofile2.txt"}\NormalTok{, }\StringTok{"a"}\NormalTok{)}
\NormalTok{f.write(}\StringTok{"Now the file has more content!"}\NormalTok{)}
\NormalTok{f.close()}

\CommentTok{\#open and read the file after the appending:}
\NormalTok{f }\OperatorTok{=} \BuiltInTok{open}\NormalTok{(}\StringTok{"demofile2.txt"}\NormalTok{, }\StringTok{"r"}\NormalTok{)}
\BuiltInTok{print}\NormalTok{(f.read())}
\end{Highlighting}
\end{Shaded}

\begin{verbatim}
Hello! Welcome to demofile.txt
This file is for testing purposes.
Good Luck!Now the file has more content!Now the file has more content!

\end{verbatim}
\end{code}

\begin{code} Abrir el archivo \texttt{demofile2.txt} y sobrescribir contenido.

\begin{Shaded}
\begin{Highlighting}[]
\NormalTok{f }\OperatorTok{=} \BuiltInTok{open}\NormalTok{(}\StringTok{"demofile3.txt"}\NormalTok{, }\StringTok{"w"}\NormalTok{)}
\NormalTok{f.write(}\StringTok{"Woops! I have deleted the content!"}\NormalTok{)}
\NormalTok{f.close()}

\CommentTok{\#open and read the file after the overwriting:}
\NormalTok{f }\OperatorTok{=} \BuiltInTok{open}\NormalTok{(}\StringTok{"demofile3.txt"}\NormalTok{, }\StringTok{"r"}\NormalTok{)}
\BuiltInTok{print}\NormalTok{(f.read())}
\end{Highlighting}
\end{Shaded}

\begin{verbatim}
Woops! I have deleted the content!
\end{verbatim}
\end{code}

\section{Crear un nuevo archivo}

Para crear un nuevo archivo en Python se debe utilizar el método \texttt{open()} con uno de los siguientes parámetros.

\begin{itemize}
  \item \textquotesingle x\textquotesingle. Crear - crea el archivo y devuelve un error si el archivo existe.
  \item \textquotesingle a\textquotesingle. Añadir - crea el archivo si es que no existe previamente.
  \item \textquotesingle w\textquotesingle. Escribir - crea el archivo si es que no existe previamente.
\end{itemize}

\begin{code} Crea un archivo llamado \texttt{myfile.txt}.
\begin{Shaded}
\begin{Highlighting}[]
\NormalTok{f }\OperatorTok{=} \BuiltInTok{open}\NormalTok{(}\StringTok{"myfile.txt"}\NormalTok{, }\StringTok{"x"}\NormalTok{)}
\end{Highlighting}
\end{Shaded}
\end{code}

\begin{code} Crea un nuevo archivo \texttt{myfile.txt} si es que no existe previamente.

\begin{Shaded}
\begin{Highlighting}[]
\NormalTok{f }\OperatorTok{=} \BuiltInTok{open}\NormalTok{(}\StringTok{"myfile.txt"}\NormalTok{, }\StringTok{"w"}\NormalTok{)}
\end{Highlighting}
\end{Shaded}
\end{code}

\section{Borrar un archivo}

Para borrar archivos en Python se debe importar el módulo \texttt{os} y
utlizar el módulo \texttt{os.remove()}. \\

\begin{code} Borrar archivos desde Python.
\begin{Shaded}
\begin{Highlighting}[]
\ImportTok{import}\NormalTok{ os}
\NormalTok{os.remove(}\StringTok{"myfile.txt"}\NormalTok{)}
\end{Highlighting}
\end{Shaded}

\begin{Shaded}
\begin{Highlighting}[]
\ImportTok{import}\NormalTok{ os}
\ControlFlowTok{if}\NormalTok{ os.path.exists(}\StringTok{"myfile.txt"}\NormalTok{):}
\NormalTok{  os.remove(}\StringTok{"myfile.txt"}\NormalTok{)}
\ControlFlowTok{else}\NormalTok{:}
  \BuiltInTok{print}\NormalTok{(}\StringTok{"The file does not exist"}\NormalTok{)}
\end{Highlighting}
\end{Shaded}

\begin{verbatim}
The file does not exist
\end{verbatim}
\end{code}

\section{Referencias}

\begin{itemize}
  \item \href{https://www.w3schools.com/python/python_file_handling.asp}{Manejo de archivos en Python.}
  \item Lutz M., Learning Python, O\textquotesingle Reilly. 2009
\end{itemize}
