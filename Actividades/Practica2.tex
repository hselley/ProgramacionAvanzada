\documentclass{article}
\usepackage[spanish,mexico]{babel}
\usepackage{amsmath}
\usepackage[margin=2.5cm]{geometry}

\title{Práctica 2}
\date{\today}
\author{Programación Avanzada}

\begin{document}
\maketitle

Realice los siguientes ejercicios.

\begin{enumerate}
    \item Desarrolle un programa en Python que genere una arreglo NumPy tridimensional de 
        tamaño $5\times 4\times 3$ con valores aleatorios entre 0 y 100. Posteriormente
        el programa debe encontrar el elemento más pequeño y el más grande e indicar la 
        ubicación de dichos elementos dentro del arreglo. Imprima la matriz, los valores 
        menor y mayor, así como sus ubicaciones. Guarde su programa en un archivo con 
        extensión \textit{.py}.

    \item 
        Dada la matriz tridimensional $H$: 
        \[
            H = 
                \begin{pmatrix}
                    \begin{pmatrix}
                        1 & 2 & 3 \\
                        4 & 5 & 6
                    \end{pmatrix},
                    \begin{pmatrix}
                        7 & 8 & 9 \\
                        10 & 11 & 12
                    \end{pmatrix}    
                \end{pmatrix}
        \]
        Calcula la transpuesta de cada "plano"\footnote[1]{cada matriz bidimensional 
        dentro de la tridimensional} dentro de la matriz tridimensional $H$. Guarde su 
        programa en un archivo con extensión \textit{.py}.

    \item Haga repositorio local con \textit{git} dentro de la carpeta donde almacenó 
        sus programas. Haga al menos un par de \textit{commits} dentro del repositorio.
        Agregue a su carpeta un archivo llamado \textit{README.md} con una breve 
        descripción de sus programas en texto plano.

    \item Haga una cuenta de usuario en GitHub, haga también un repositorio dentro de 
        GitHub para sus programas. Enlace su repositorio local con el de GitHub y 
        realice al menos un \textit{push} para subir sus programas.

\end{enumerate}


\end{document}